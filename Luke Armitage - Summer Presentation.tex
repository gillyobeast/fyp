\documentclass[11pt]{beamer}
\usetheme{Singapore}
\usepackage[utf8]{inputenc}
\usepackage{amsmath}
\usepackage{mathrsfs}
\usepackage{graphicx}
\usepackage{amsfonts}
\usepackage{amssymb}
\usepackage{amsthm}

\usefonttheme[onlymath]{serif}

\let\emph\relax 
\DeclareTextFontCommand{\emph}{\bfseries\itshape}

\author{Luke Armitage}
\title{$C^\ast$-Algebras, and the Gelfand-Naimark Theorems}
\setbeamercovered{invisible} 
\setbeamertemplate{navigation symbols}{} 
\logo{\includegraphics[scale=0.15]{UOY-Logo.pdf}} 
\institute{University of York} 
\date{June 8, 2017} 
%\subject{} 

\theoremstyle{definition}
\newtheorem{defn}{Definition}
\theoremstyle{plain}
\newtheorem{thm}{Theorem}

\renewcommand{\P}[1]{\mathscr{P}(#1)}
\newcommand{\K}{\mathbb{K}}

\begin{document}


\begin{frame}
\titlepage
\end{frame}
				% write C* axiom on board next to slides..
				% '' just gonna write this here so we can remember that it's an innocuous
				% seeming statement causing all these structures.''
%% just gonna say some cool stuff i found out about C*-Algebras	
\begin{frame} {Definitions}
	A \emph{$C^\ast$-algebra} $A$ is a Banach algebra with norm $\|\cdot\|$ and an involution map 
	$a \mapsto a^\ast$ satisfying the following: 
	\begin{enumerate}
		\item $a^{\ast\ast} = a$
		\item $(\alpha a + b)^\ast = \bar\alpha a^\ast + b^\ast$ 
		\item $(ab)^\ast = b^\ast a^\ast$
		\item $\|a^\ast a\| = \|a\|^2$	~ ($C^\ast$ axiom) 
			%% we will see how the $C^\ast$-axiom leads to lots of structure in these algebras.
	\end{enumerate}
\end{frame}

\begin{frame} {Definitions}
	A \emph{state} is a positive linear functional $\rho : A \to \K$ such that $\rho(a) \geq 0$
	for all positive $a \in A$.
	
	spectrum, spectral radius, 
	state, pure state,
	* homo/isomorphism,
	representation, faithful representation,
\end{frame}

\begin{frame} {Examples}
	
\end{frame}

\begin{frame} {Cool Asides}
	%how these are forced by C* axiom:
	\emph{Uniqueness of norm:} % lemma2
	$\|a\|^2 = \|a^\ast a\| = r(a^\ast a)$. 
	Requires spectral theory. The spectral radius of a normal element is equal to its norm. From this, and the C* axiom, we get that the norm of each element is given by the spectral radius, which is defined in terms of the spectrum which does not use the norm. \\
	\emph{$\ast$-homomorphisms are continuous:} % prop2
	homomorphisms do not increase norm, so are bounded and hence continuous. isomorphisms are isometric. again uses spectral theory, this time to show that spectral radius is not increased / is preserved.
\end{frame}

\begin{frame} {Gelfand-Naimark Theorems}
	\begin{thm}
		Every Abelian $C^\ast$-algebra $A$ is $\ast$-isomorphic to $C(\P A)$, the 
		algebra of continuous functions on the compact Hausdorff space $\P A$ of pure 
		states on $A$.
	\end{thm}
	\pause
	\begin{thm}
		Every $C^\ast$-algebra has a faithful representation.
	\end{thm}
\end{frame}

\begin{frame} {The Gelfand-Naimark-Segal Construction}
	Used to prove the GN theorem. \\
	Given a state on a C* algebra, we can construct a Hilbert space and a representation on that space.
	Given $a$ and $b$ in $A$, define $\langle a, b \rangle = \rho(b^\ast a)$. This is a semi-inner product -- basically an inner product, but there exist $a \neq 0$ such that $\langle a,a \rangle = 0$.
	However, if we consider the quotient vector space of $A$ by the collection of such elements, this space completes to a Hilbert space with $\langle \cdot , \cdot \rangle$ as the inner product.
\end{frame}

\begin{frame} {References}
	
\end{frame}

\end{document}