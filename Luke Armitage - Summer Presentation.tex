\documentclass[11pt]{beamer}
\usetheme{Singapore}
\usepackage[utf8]{inputenc}
\usepackage{amsmath}
\usepackage{mathrsfs}
\usepackage{graphicx}
\usepackage{amsfonts}
\usepackage{amssymb}
\usepackage{amsthm}

\usefonttheme[onlymath]{serif}

\setlength{\parskip}{0.5em}

\let\emph\relax 
\DeclareTextFontCommand{\emph}{\bfseries\itshape}

\author{Luke Armitage}
\title{$C^\ast$-Algebras, and the Gelfand-Naimark Theorems}
\setbeamercovered{invisible} 
\setbeamertemplate{navigation symbols}{} 
\logo{\includegraphics[scale=0.15]{UOY-Logo.pdf}} 
\institute{University of York} 
\date{June 8, 2017} 
%\subject{} 

\theoremstyle{definition}
\newtheorem{thm}{Theorem}

\renewcommand{\P}[1]{\mathscr{P}(#1)}
\newcommand{\K}{\mathbb{K}}

\begin{document}


%%%% TODO
%% write flashcards.
%% practice a couple times

\begin{frame}
\titlepage
\end{frame}
%% thank for coming. 
%% my project is about some objects called C* algebras, which occur in 
%% functional analysis. in my project i give an introduction to C* algebras
%% and examined the proofs of the gelfand naimark theorems.

%% in this talk, i hope to give a very brief idea of what a C* algebra is,
%% talk about a couple of cool things i found out about them,
%% discuss the main idea of the proof of one of the theorems,
%% and hopefully convice you that you ought to read my project.

\begin{frame} {Definitions} {Banach algebra}
	A \emph{Banach algebra} is a complete normed algebra $A$ such that
	\[
		 \left\|ab\right\| \leq \left\|a\right\| \cdot \left\|b\right\| ~~\forall a,b \in A. 
		 %% submultiplicity
	\]
\end{frame}
%% our first definition is that of a Banach algebra. a Banach algebra is a normed algebra,
%% which means it's a normed linear space with a ring structure satifying this 
%% submultiplicity relation, which is complete in the sense that all cauchy sequences converge.
%% basic example, think of the complex numbers.

\begin{frame} {Definitions} {$C^\ast$-algebra}
	A \emph{$C^\ast$-algebra} $A$ is a Banach algebra with \emph{adjoint} map 
	$a \mapsto a^\ast$ on $A$, satisfying the following: 
	\begin{enumerate}
		\item $a^{\ast\ast} = a$ 									%% involution 
		\item $(\alpha a + b)^\ast = \bar\alpha a^\ast + b^\ast$ 	%% anti-linear
		\item $(ab)^\ast = b^\ast a^\ast$							%% anti-homomorphic  
		\item $\left\|a^\ast a\right\| = \left\|a\right\|^2$		~ ($C^\ast$ axiom) 
		
	\end{enumerate}
	
	We assume here that $C^\ast$-algebras have an identity, denoted 1. 
\end{frame}
%% now for the major definition. a c* algebra is a Banach algebra with this funny star 
%% operation defined on it, called the adjoint operation, which satisfies these axioms:
%% it must be an involution, anti linear, anti homomorphic (or anti multiplicative if you like)
%% and the all-important C* axiom. This ensures the adjoint map and norm behave in a nice way
%% -- gives a lot of structure to C* algebras, which we will see shortly. [write c* axiom]
%%
%% for a basic example, again think of the complex numbers, with complex conjugation.
%%
%% we note that we assume an identity for purposes of this talk, tho it's not generally assumed.
%% we also generally don't assume commutativity.

\begin{frame} {Examples}
	Continuous complex-valued functions on a compact Hausdorff space.
	
	Bounded operators on a Hilbert space, $\mathcal{B(H)}$. 
	
	Ideal of compact operators, $\mathcal{K(H)}$.   
	
	Calkin algebra, the quotient algebra $\mathcal{B(H)/K(H)}$.
\end{frame}
%% these are important, fairly general examples. the first one will come up later,
%% the second is the prototypical example, the bounded operators on some Hilbert space,
%% which students who took functional analysis this year will be somewhat familiar with.
%% two other examples which are important in funcitonal analysis are the compact operators,
%% and the Calkin algebra, which is the quotient of the previous two examples taken on 
%% the same Hilbert space.

\begin{frame} {Definitions} {Spectrum}
	\emph{Spectrum} of $a \in A$ is 
	\[
		\sigma(a) = \left\{\lambda \in \mathbb C \mid a - \lambda 1  \mbox{ is not invertible} \right\}.
	\]
	\emph{Spectral radius} of $a \in A$ is 
	\[
		 r(a) = \sup_{ \lambda \in \sigma(a)} \left| \lambda \right|.
	\]
\end{frame}
%% we move on to some important further definitions. those familiar with functional analysis
%% will be aware of the importance of the spectrum in the study of bounded operators, and 
%% the spectrum is also important here. it turns out that the spectrum of an element is always
%% a nonempty closed subset of the unit disk of radius ||a||.
%% thus it makes sense to talk about the supremum of the spectral values, which we call the 
%% spectral radius.

\begin{frame}{Definitions} {Normal elements}
	Say that $a \in A$ is \emph{self-adjoint} if $a^\ast = a$, and \emph{normal} 
	if $a^\ast a = aa^\ast$.
	
	Say that a self-adjoint $a$ is \emph{positive} if $\sigma(a) \subset \mathbb R^+$.
\end{frame}
%% there are some distinguished elements of C* algberas we sould like to single out, 
%% namely self-adjoint elements,
%% normal elements and positive elements. self adjoint elements are exactly what they 
%% sound like, normal elements commute with their adjoint and positive elements are 
%% self adjoint with real, nonnegative spectrum.
%% we note that self adjoint elements are normal, and that elements of the 
%% form a*a are self adjoint by the third axiom. [write the a*a]

\begin{frame}{Definitions} {States}
	A \emph{state} is a linear map $\rho : A \to \mathbb C$ such that $\rho(a) \geq 0$
	for all positive $a \in A$, and $\rho(1) = 1$.
	
	The \emph{state space}, $\mathcal S (A)$, is a convex subset of the dual space of $A$. 
	Call the extreme points of the state space \emph{pure states}.
\end{frame}
%% linear functionals are, unsurprisingly, an important thing in functional analysis.
%% we call a linear functional, that is a linear map A to C, a state if it is posisive 
%% on positive elements, and maps the identity to 1.
%% the state space is the collection of all states on A, and  it is a convex subset 
%% of the dual space. the extreme points of the state space are called pure states.

\begin{frame} {Definitions} {Maps between $C^\ast$-algebras}
	A \emph{$\ast$-homomorphism} is an algebra homomorphism $\varphi : A \to B$ such that $\varphi (a^\ast) = \varphi(a)^\ast$, for all $a \in A$.
	
	A \emph{$\ast$-isomorphism} is a bijective $\ast$-homomorphism.
\end{frame}
%% the definitions of *homomorphisms between C* algebras is pretty much exactly what you
%% would expect it to be: preserves the adjoint operation. here A and B are C* algebras
%% similarly we define a *-isomorphism in the way you would expect.

\begin{frame} {Cool Results} {Uniqueness of norm} % lemma2
	The norm of each element is given by the spectral radius, so the norm is unique.
	
	%% the spectral radius of a normal element is equal to its norm.
		%% proved by showing that one of +-||a|| is in spec(a)
	\begin{align*}
		\|a\|^2 &= \|a^\ast a\| = r(a^\ast a) \\ 
		\|a\| &= r(a^\ast a)^{1/2}
	\end{align*}
\end{frame}		
%% we noted before that a*a is normal, and it turns out that the spectral radius
%% of a normal element is equal to its norm. this allows us to show that the norm of a
%% depends only on the spectral radius, i.e. there is only 
%% one norm on the C* algebra for which the C* axiom will hold.
		
\begin{frame} {Cool Results} {$\ast$-homomorphisms are continuous} % prop2
	Let $\varphi:A \to B$ be a $\ast$-homomorphism, then
	\begin{align*}
		\|\varphi(a)\| \leq \|a\| 
	\end{align*}
	for all $a \in A$. 
	
	Equality if $\varphi$ is a $\ast$-isomorphism.
\end{frame}
%% the second of our cool results is that *homomorphisms are continuous and that
%% *isomorphisms are isometric. we prove this inequality by showing that spec phi(a) is 
%% contained in spec a, and then since phi is a bounded linear map, it is continuous.
%% the second part comes from showing that the spectra are equal in the bijective case.

\begin{frame}{Definitions} {Representation}
	A \emph{representation} of $A$ on a Hilbert Space $\mathcal H$ is a $\ast$-homomorphism 
	$A \to \mathcal{B(H)}$.
	
	A bijective representation is called \emph{faithful}.
\end{frame}
%% a representation is a star homomorphisms into the bounded operators on some Hilbert space
%% and we call it faithful if it's a star isomorphism

\begin{frame} {Gelfand-Naimark Theorem} 
	\begin{thm}
		Every $C^\ast$-algebra has a faithful representation. 
	\end{thm}
	\begin{proof}
	Uses the Gelfand-Naimark-Segal construction.\renewcommand{\qedsymbol}{}
	\end{proof}

\end{frame}
%% this is the general gelfand naimark theorem, it means that these C* algebras
%% we are studying are the same thing as an algebra of bounded operators.
%% the proof uses the gelfand-naimark-segal construction 
%% to find a representation of A from each element of some collection of 
%% states containing all pure states. use something to stitch these together, 
%% and the result is a faithful representation of A on some Hilbert space.

\begin{frame} {Gelfand-Naimark Theorem} {Proof: Gelfand-Naimark-Segal Construction}
	Let $L= \left\{a \in A \mid \rho (a^\ast a) = 0 \right\}$.
	
	$\mathcal H $ is the Hilbert space completion of $A / L$.
	
	Define operators 
	\[
		\pi_a : A/L \to A/L : b+L \mapsto ab + L,
	\]
	and extend to $\pi_a : \mathcal H \to \mathcal H$.
	
	Representation is given by
	\[
		\pi: A \to \mathcal{B(H)} : a \mapsto \pi_a.
	\]
\end{frame}
%% here is a brief outline of the GNS construction.
%% Given a state rho on A, we can construct a Hilbert space H and a representation
%% of A on that Hilbert space.
%% The Hilbert space comes from considering the quotient space of A by the 
%% ideal L, defined here. Taking this quotient means we can then define an inner 
%% product by <a,b> = rho(b*a), and completing this quotient space to a Hilbert
%% space gives us what we need.
%% now we define operators like these for a, extending them to H, and the representation
%% is given by sending each element of A to its corresponding 'extended
%% left-multiplication' operator on H.	

\begin{frame} {Gelfand-Naimark Theorem} {Proof: Direct Sum}
	Proof concludes by taking `direct sum' representation over the representations 
	given by doing GNS construction to a subset of state space containing all pure 
	states. This gives a faithful representation.
\end{frame}
%%
%%

\begin{frame} {Gelfand-Naimark Theorem} {For Commutative $C^\ast$-algebras}
	\begin{thm}
		Every commutative $C^\ast$-algebra $A$ is $\ast$-isomorphic to $C(\P A)$, the 
		algebra of continuous functions on the compact Hausdorff space $\P A$ containing 
		all pure states on $A$. 
	\end{thm}
\end{frame} 
%% the title of this talk refers to multiple theorems, and here's the other one.
%% this theorem says that all commutative C* algebras have a related compact, Hausdorff
%% topological space.
%% This theorem allows us to think about the study of C* algebras in general as
%% some sort of non commutative analogue to topology -- with the 'topological space'
%% given by the pure state space, and the 'topology' given by the preimages of certain 
%% maps on that space.

	%% if asked: proved by showing that all pure states are multiplicative functionals and
	%% that something called the Gelfand representation separates the points of P(A)

\begin{frame} {References -- Any Questions?}
	Kadison, R.V. and Ringrose, J.R., 1983. 
	\textit{Fundamentals of the Theory of Operator Algebras, Vol. I. Elementary Theory}. 
	Springer.
	
	Blackadar, B., 2006. 
	\textit{Algebras: Theory of $C^\ast$-Algebras and Von~Neumann 
	Algebras} (Vol. 122). 
	Springer Science \& Business Media.

	\vfill

	My project report can be downloaded from \texttt{goo.gl/Qv1zas}.
\end{frame}
%% thank you for listening, hopefully i have convinced you that my project is
%% worth reading. i have included a URL from which you can download my project.
%% finally -- any questions?

\end{document}



