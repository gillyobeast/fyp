\documentclass[11pt]{beamer}
\usetheme{Singapore}
\usepackage[utf8]{inputenc}
\usepackage{amsmath}
\usepackage{mathrsfs}
\usepackage{graphicx}
\usepackage{amsfonts}
\usepackage{amssymb}
\usepackage{amsthm}

\usefonttheme[onlymath]{serif}

\setlength{\parskip}{0.5em}

\let\emph\relax 
\DeclareTextFontCommand{\emph}{\bfseries\itshape}

\author{Luke Armitage}
\title{$C^\ast$-Algebras, and the Gelfand-Naimark Theorems}
\setbeamercovered{invisible} 
\setbeamertemplate{navigation symbols}{} 
\logo{\includegraphics[scale=0.15]{UOY-Logo.pdf}} 
\institute{University of York} 
\date{June 8, 2017} 
%\subject{} 

\theoremstyle{definition}
\newtheorem{thm}{Theorem}

\renewcommand{\P}[1]{\mathscr{P}(#1)}
\newcommand{\K}{\mathbb{K}}

\begin{document}


\begin{frame}
\titlepage
\end{frame}
%% just gonna say some cool stuff i found out about C*-Algebras	

\begin{frame} {Definitions} {Banach Algebra}
	A \emph{Banach algebra} is a complete normed algebra $A$ such that
	\[
		 \left\|ab\right\| \leq \left\|a\right\| \cdot \left\|b\right\| ~~\forall a,b \in A. 
		 %% submultiplicity
	\]
	
	%% complete in the sense that all Cauchy sequences converge to a limit
	%% normed algebra means normed linear space with ring structure
\end{frame}

\begin{frame} {Definitions} {$C^\ast$-Algebra}
	A \emph{$C^\ast$-algebra} $A$ is a Banach algebra with involution map $a \mapsto a^\ast$ 
	satisfying the following: 
	\begin{enumerate}
		\item $a^{\ast\ast} = a$ 									%% involution 
		\item $(\alpha a + b)^\ast = \bar\alpha a^\ast + b^\ast$ 	%% anti-linear
		\item $(ab)^\ast = b^\ast a^\ast$							%% anti-homomorphic  
		\item $\left\|a^\ast a\right\| = \left\|a\right\|^2$		~ ($C^\ast$ axiom) 
			%% we will see how the $C^\ast$-axiom leads to lots of structure in these algebras.
	\end{enumerate}
	%% define unit here
\end{frame}

\begin{frame} {Examples}
	Continuous linear functionals on a compact, Hausdorff space.
	
	Bounded operators on a Hilbert space, $\mathcal{B(H)}$. %% the prototypical example
	
	Ideal of compact operators, $\mathcal{K(H)}$.   % brief mention these two
	
	Calkin algebra, the quotient algebra $\mathcal{B(H)/K(H)}$.
	
\end{frame}

\begin{frame} {Definitions} {Spectrum}
	\emph{Spectrum} of $a \in A$ is 
	\[
		\sigma(a) = \left\{\lambda \in \mathbb C \mid a - \lambda 1  \mbox{ is not invertible} \right\}.
	\]
	\emph{Spectral radius} of $a \in A$ is 
	\[
		 r(a) = \sup_{ \lambda \in \sigma(a)} \left| \lambda \right|.
	\]
	Say that $a \in A$ is \emph{positive} if $a^\ast = a$ and $\sigma(a) \subset \mathbb R$.
\end{frame}
	
\begin{frame}{Definitions} {States}
	A \emph{state} is a linear map $\rho : A \to \mathbb C$ such that $\rho(a) \geq 0$
	for all positive $a \in A$, and $\rho(1) = 1$.
	
	The \emph{state space}, $\mathcal S (A)$, is a convex subset of the dual space of $A$. 
	Call the extreme points of the state space \emph{pure states}.
\end{frame}

\begin{frame} {Definitions} {Maps between $C^\ast$-algebras}
	A \emph{$\ast$-homomorphism} is an algebra homomorphism such that $\varphi (a^\ast) = \varphi(a)^\ast$.
	
	A \emph{$\ast$-isomorphism} is a bijective $\ast$-homomorphism.
\end{frame}

\begin{frame} {Cool Results} {Uniqueness of norm}
	% lemma2
	$\|a\|^2 = \|a^\ast a\| = r(a^\ast a)$. 
	The spectral radius of a normal element is equal to its norm. 
	From this, and the C* axiom, we get that the norm of each element is given by 
	the spectral radius, which is defined in terms of the spectrum which is not 
	defined in terms of the norm. 
\end{frame}		
		
\begin{frame} {Cool Results} {$\ast$-homomorphisms are continuous} 
	% prop2
	homomorphisms do not increase norm, so are bounded and hence continuous. isomorphisms are isometric. again uses spectral theory, this time to show that spectral radius is not increased / is preserved.
\end{frame}

\begin{frame}{Definitions} {Representation}
	A \emph{representation} of $A$ on a Hilbert Space $\mathcal H$ is a $\ast$-homomorphism $A \to \mathcal{B(H)}$.
	
	A bijective representation is called \emph{faithful}.
\end{frame}

\begin{frame} {Gelfand-Naimark Theorem} 
	\begin{thm}
		Every $C^\ast$-algebra has a faithful representation. 
	\end{thm}
	\begin{proof}
	Uses the Gelfand-Naimark-Segal construction.\renewcommand{\qedsymbol}{}
	\end{proof}
	%% to find a representation of A from each element of some collection of 
	%% states containing all pure states. use something called the direct sum 
	%% representation to stitch these together, and the result is a faithful 
	%% representation of A on the direct sum of the resulting Hilbert spaces.
	
\end{frame}

\begin{frame} {Gelfand-Naimark Theorem} {Proof: Gelfand-Naimark-Segal Construction}
	%% Used to prove the Gelfand-Naimark theorem.
	%% Given a state rho on A, we can construct a Hilbert space H and a representation
	%% of A on that Hilbert space. 
	
	Let $L= \left\{a \in A \mid \rho (a^\ast a) = 0 \right\}$.
	
	$\mathcal H $ is the Hilbert space completion of $A / L$.
	
	%% the Hilbert space comes from considering the quotient space of A by the 
	%% ideal L, defined here. Taking this quotient means we can then define an inner 
	%% product by <a,b> = rho(b*a), and completing this quotient space to a Hilbert
	%% space gives us what we need.
	
	Define operators 
	\[
		\pi_a : A/L \to A/L : b+L \mapsto ab + L,
	\]
	and extend to $\pi_a : \mathcal H \to \mathcal H$.
	
	%% now we define operators like this, extending them to H, and the representation
	%% is given by sending each element of A to its corresponding 'extended
	%% left-multiplication' operator on H.	
	
	Representation is given by
	\[
		\pi: A \to \mathcal{B(H)} : a \mapsto \pi_a.
	\]
\end{frame}

\begin{frame} {Gelfand-Naimark Theorem} {Proof: Direct Sum}
	Proof concludes by taking `direct sum' representation over the representations 
	given by doing GNS construction to a subset of state space containing all pure 
	states. This gives a faithful representation.
\end{frame}

\begin{frame} {Gelfand-Naimark Theorem} {For Commutative $C^\ast$-algebras}
	\begin{thm}
		Every Abelian $C^\ast$-algebra $A$ is $\ast$-isomorphic to $C(\P A)$, the 
		algebra of continuous functions on the compact Hausdorff space $\P A$ containing 
		all pure states on $A$. 
	\end{thm}
\end{frame}

\begin{frame} {References -- Any Questions?}

	%% thank you for listening, hopefully i have convinced you that my project is at least 
	%% remotely interesting. i have included a URL from which you can download my project.
	%% finally -- any questions?
	
	Kadison, R.V. and Ringrose, J.R., 1983. 
	\textit{Fundamentals of the Theory of Operator Algebras, Vol. I. Elementary Theory}. 
	Springer.
	
	Blackadar, B., 2006. 
	\textit{Algebras: Theory of $C^\ast$-Algebras and Von~Neumann 
	Algebras} (Vol. 122). 
	Springer Science \& Business Media.

	\vfill

	My project report can be downloaded from \texttt{goo.gl/Qv1zas}.
	
\end{frame}

\end{document}



