\documentclass[11pt]{beamer}
\usetheme{Singapore}
\usepackage[utf8]{inputenc}
\usepackage{amsmath}
\usepackage{mathrsfs}
\usepackage{graphicx}
\usepackage{amsfonts}
\usepackage{amssymb}
\usepackage{amsthm}

\usefonttheme[onlymath]{serif}

\setlength{\parskip}{0.5em}

\let\emph\relax 
\DeclareTextFontCommand{\emph}{\bfseries\itshape}

\author{Luke Armitage}
\title{$C^\ast$-Algebras, and the Gelfand-Naimark Theorems}
\setbeamercovered{invisible} 
\setbeamertemplate{navigation symbols}{} 
\logo{\includegraphics[scale=0.15]{UOY-Logo.pdf}} 
\institute{University of York} 
\date{June 8, 2017} 
%\subject{} 


\newtheorem{thm}{Theorem}

\renewcommand{\P}[1]{\mathscr{P}(#1)}
\newcommand{\K}{\mathbb{K}}

\begin{document}


\begin{frame}
\titlepage
\end{frame}
				% write C* axiom on board next to slides..
				% '' just gonna write this here so we can remember that it's an innocuous
				% seeming statement causing all these structures.''
%% just gonna say some cool stuff i found out about C*-Algebras	
\begin{frame} {Definitions}
	A \emph{$C^\ast$-algebra} $A$ is a Banach algebra with norm $\left\|\cdot\right\|$ 
	and an involution map $a \mapsto a^\ast$ satisfying the following: 
	\begin{enumerate}
		\item $a^{\ast\ast} = a$
		\item $(\alpha a + b)^\ast = \bar\alpha a^\ast + b^\ast$ 
		\item $(ab)^\ast = b^\ast a^\ast$
		\item $\left\|a^\ast a\right\| = \left\|a\right\|^2$	~ ($C^\ast$ axiom) 
			%% we will see how the $C^\ast$-axiom leads to lots of structure in these algebras.
	\end{enumerate}
\end{frame}
	
\begin{frame} {Definitions}
	\emph{Spectrum} of $a \in A$ is 
	\[
		\sigma(a) = \left\{\lambda \in \mathbb C \mid a - \lambda 1  \mbox{ is not invertible} \right\}.
	\]
	\emph{Spectral radius} of $a \in A$ is 
	\[
		 r(a) = \sup_{ \lambda \in \sigma(a)} \left| \lambda \right|.
	\]
	Say that $a \in A$ is \emph{positive} if $a^\ast = a$ and $\sigma(a) \subset \mathbb R$.
\end{frame}
	
\begin{frame}{Definitions}
	A \emph{state} is a linear map $\rho : A \to \mathbb C$ such that $\rho(a) \geq 0$
	for all positive $a \in A$, and $\rho(1) = 1$.
	
	The \emph{state space}, $\mathcal S (A)$, is a convex subset of the dual space of $A$. 
	Call the extreme points of the state space \emph{pure states}.
\end{frame}

\begin{frame} {Examples}
	\begin{itemize}
		\item Continuous linear functionals on a compact, Hausdorff space.
		\item Bounded operators on a Hilbert space, $\mathcal{B(H)}$. %% the prototypical example
		\item Ideal of compact operators, $\mathcal{K(H)}$.   % brief mention these two
		\item Calkin algebra, the quotient algebra $\mathcal{B(H)/K(H)}$.
	\end{itemize}
\end{frame}

\begin{frame} {Definitions}
	A \emph{$\ast$-homomorphism} is an algebra homomorphism such that $\varphi (a^\ast) = \varphi(a)^\ast$.
	
	A \emph{$\ast$-isomorphism} is a bijective $\ast$-homomorphism.
\end{frame}

\begin{frame} {Cool Results}
	%how these are forced by C* axiom:
	\emph{Uniqueness of norm:} % lemma2
	$\|a\|^2 = \|a^\ast a\| = r(a^\ast a)$. 
	Requires spectral theory. The spectral radius of a normal element is equal to its norm. From this, and the C* axiom, we get that the norm of each element is given by the spectral radius, which is defined in terms of the spectrum which does not use the norm. 
	
	\emph{$\ast$-homomorphisms are continuous:} % prop2
	homomorphisms do not increase norm, so are bounded and hence continuous. isomorphisms are isometric. again uses spectral theory, this time to show that spectral radius is not increased / is preserved.
\end{frame}

\begin{frame}{Definitions}
	A \emph{representation} of $A$ on a Hilbert Space $\mathcal H$ is a $\ast$-homomorphism $A \to \mathcal{B(H)}$.
	
	A bijective representation is called \emph{faithful}.
\end{frame}

\begin{frame} {Gelfand-Naimark Theorems}
	\begin{thm}
		Every Abelian $C^\ast$-algebra $A$ is $\ast$-isomorphic to $C(\P A)$, the 
		algebra of continuous functions on the compact Hausdorff space $\P A$ of pure 
		states on $A$.
	\end{thm}
	\pause
	\begin{thm}
		Every $C^\ast$-algebra has a faithful representation.
	\end{thm}
\end{frame}

\begin{frame} {The Gelfand-Naimark-Segal Construction}
	Used to prove the GN theorem. \\
	Given a state on a C* algebra, we can construct a Hilbert space and a representation on that space.
	Given $a$ and $b$ in $A$, define $\langle a, b \rangle = \rho(b^\ast a)$. This is a semi-inner product -- basically an inner product, but there exist $a \neq 0$ such that $\langle a,a \rangle = 0$.
	However, if we consider the quotient vector space of $A$ by the collection of such elements, this space completes to a Hilbert space with $\langle \cdot , \cdot \rangle$ as the inner product.
\end{frame}

\begin{frame} {References -- Questions?}
	My project report can be found at \texttt{goo.gl/[link]}
\end{frame}

\end{document}