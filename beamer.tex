\documentclass{beamer}

\mode<presentation>{ \usetheme{Frankfurt} \setbeamercovered{invisible}}

\usepackage[english]{babel}

\usepackage[utf8]{inputenc}
\usepackage{times}
\usepackage[T1]{fontenc}
\usepackage{hyperref}

\title[Hypergraphs]
{C$^\ast$-algebras, and The~Gelfand-Naimark~theorem}
 
\author[L. Armitage] 
{Luke Armitage}

\date{Friday 2nd December 2016}

\beamertemplatenavigationsymbolsempty
%\beamerdefaultoverlayspecification{<+->}


\begin{document}
\begin{frame}
  	\titlepage
\end{frame}


\section{Introduction}
\subsection{}
\begin{frame}{A Brief History}{}
 \begin{itemize}
	\item 1925 -- Heisenberg, 					
				%% authored paper
		\emph{\"{U}ber quantentheoretische...}\\
				%% presenting a
		 new QM model.						
		 		%% which turned out to have noncommutative properties, 
		\[PQ-QP= \frac{h} {2 \pi i}.\]			
				%% for example
											
	\item 1925 -- Born \& Jordan,
		\emph{Zur Quantenmechanik}\\			
				%% took the noncommutative properties and 
				%% developed in terms of matrices
		 developed matrix mechanics.		
			
	\item 1935-1943 -- Murray \& von Neumann,
		\emph{On rings of operators}\\			
				%% developed a more general framework
		 a general framework. 					
		 		%% for this matrix mechanics
 \end{itemize}
\end{frame}

\subsection{}
\begin{frame}{A Brief History}{}
 \begin{itemize}
  	\item 1943 -- Gelfand \& Naimark,
  		\emph{On the embedding of normed rings...}\\  
  				%% into the ring of operators on a Hilbert Space
  		 abstract C$^\ast$-algebras. 				  
  		 		%% free from dependence on operators 
				%% on a Hilbert space
  		
 \end{itemize}
\end{frame}


\section{Aims}
\subsection{}
\begin{frame}{Aims} 
 In my project
 		% i aim to give
 \begin{itemize}
 	\item Background understanding on C$^\ast$-algebras, standard results,	
 		% including geometric and topological discussion of results where possible
 	\item Representation theory, considering GNS construction, 				
 		% which i will discuss next, and
 	\item Commutative and general GN theorems and proofs.					
 		% giving a link between the operator algebras and abstract C* algbebras
 \end{itemize}
\end{frame}

%% define (quickly) a state and a *-rep here


\section{Gelfand-Naimark Theorems}
\subsection{•}
\begin{frame}{Gelfand-Naimark-Segal construction}
 Two theorems:
 \begin{itemize}
 	\item Existence, for every state $\rho$ of a C$^\ast$-algebra, of a cyclic $\ast$-representation $\pi$ and a unit cyclic vector $x$, such that 
 		\[
 			\rho (A) = \langle \pi _{\rho} (A) x_{\rho}, x_{\rho} \rangle.
 		\]
 	\item Uniqueness of these representations and vectors.
 \end{itemize}
\end{frame}
 
\begin{frame}{Gelfand-Naimark theorems}
 
\end{frame}


\appendix{}
\section{References}
\begin{frame}{References}{}
 
\end{frame}




\end{document}




