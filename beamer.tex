\documentclass{beamer}

\mode<presentation>{ \usetheme{Frankfurt} \setbeamercovered{invisible}}

\usepackage[english]{babel}

\usepackage[utf8]{inputenc}
\usepackage{times}
\usepackage[T1]{fontenc}
\usepackage{hyperref}

\title[Hypergraphs]
{C$^\ast$-algebras, and The~Gelfand-Naimark~theorem}
 
\author[L. Armitage] 
{Luke Armitage}

\date{Friday 2nd December 2016}

\beamertemplatenavigationsymbolsempty
%\beamerdefaultoverlayspecification{<+->}


\begin{document}
\begin{frame}
  	\titlepage
\end{frame}


\section{Introduction}
\subsection{}
\begin{frame}{A Brief History}{}
 \begin{itemize}
	\item 1925 -- Heisenberg, 					%% authored paper
		\emph{\"{U}ber quantentheoretische...}\\%% presenting a
		-- new QM model.						%% which turned out to have noncommutative properties, 
		\[PQ-QP= \frac{h} {2 \pi i}.\]			%% for example
											
	\item 1925 -- Born \& Jordan,
		\emph{Zur Quantenmechanik}\\			%% took the noncommutative properties and 
												%% developed in terms of matrices
		-- developed matrix mechanics.		
			
	\item 1935-1943 -- Murray \& von Neumann,
		\emph{On rings of operators}\\			%% developed a more general framework
		-- a general framework. 				%% for this matrix mechanics
 \end{itemize}
\end{frame}

\subsection{}
\begin{frame}{A Brief History}{}
 \begin{itemize}
  	\item 1943 -- Gelfand \& Naimark,
  		\emph{On the embedding of normed rings...}\\  %% into the ring of operators on a Hilbert Space
  		-- abstract C$^\ast$-algebras. 				  %% free from dependence on operators 
  												      %% on a Hilbert space
  		
 \end{itemize}
\end{frame}


\section{Stuff}
\subsection{}
\begin{frame}{} 
	
\end{frame}

\begin{frame}{}
 
\end{frame}

\begin{frame}{ }
  
\end{frame}

\begin{frame}{}

\end{frame}


\section{ }
\begin{frame}{}

\end{frame}


\section{}
\begin{frame}{}


\end{frame}

\appendix{}
\section{References}
\begin{frame}{References}{}
 
\end{frame}




\end{document}




