\documentclass{beamer}

\mode<presentation>{ \usetheme{} \setbeamercovered{invisible}}

\usepackage[english]{babel}

\usepackage[utf8]{inputenc}
\usepackage{times}
\usepackage[T1]{fontenc}
\usepackage{hyperref}

\title[Hypergraphs]
{C$^\ast$-algebras, and the Gelfand-Naimark theorems}
 
\author[L. Armitage] 
{Luke Armitage}

\date{Friday 2nd December 2016}

\beamertemplatenavigationsymbolsempty
%\beamerdefaultoverlayspecification{<+->}


\begin{document}
\begin{frame}
  	\titlepage
\end{frame}


\section{Introduction}
\subsection{History}
\begin{frame}{A Brief History}{} %% why C* algebras are an interesting topic of study
 \begin{itemize}
	\item 1925 -- Heisenberg, 					
				%% authored paper
		\emph{\"{U}ber quantentheoretische...}.
				%% presenting a
		 New QM model.						
		 		%% which turned out to have noncommutative properties, 
		\[pq-qp= \frac{h} {2 \pi i}.\]			
				%% for example. q = displacement, p = momentum
										
	\item 1925 -- Born \& Jordan,
		\emph{Zur Quantenmechanik}.		
				%% took the noncommutative properties and 
				%% developed in terms of matrices
		 Developed matrix mechanics.		
		 		
	\item 1935-1943 -- Murray \& von Neumann,
		\emph{On rings of operators}.
				%% developed a more general framework	
			A general framework. 				
		 		%% for this matrix mechanics
 	\item 1943 -- Gelfand \& Naimark,
  		\emph{On the embedding of normed rings...}.
  				%% into the ring of operators on a Hilbert Space
  		 Abstract C$^\ast$-algebras. 				  
  		 		%% free from dependence on operators 
				%% on a Hilbert space
  		
 \end{itemize}
\end{frame}


\subsection{Aims}
\begin{frame}{Aims} 
 In my project
 		% i aim to give
 \begin{itemize}
 	\item Background understanding of C$^\ast$-algebras, standard results,	
 		% including geometric and topological discussion of results where possible.
 	\item Representation theory, considering the Gelfand-Naimark-Segal construction, 				
 		% which is an important step in the proof of:
 	\item Commutative and general GN theorems and their proofs.					
 		% giving a link between the operator algebras and abstract C* algbebras
 \end{itemize}
\end{frame}

%% define C* algebras, and (quickly) states and *-reps here


\section{Gelfand-Naimark theorems}
\subsection{•}
\begin{frame}{C$^\ast$-algebras}
 \begin{itemize}
 	\item A \emph{C$^\ast$-algebra} is a Banach algebra $(A, \| \cdot \|)$ with involution map 
	$a \mapsto a^\ast$, with the condition that
	\begin{align*}
		\|a ^\ast a\| = \|a\|^2 \mbox{ for all } a \in A.
	\end{align*}
	
	%% verbally expand on banach algebra and involution:
	%%		a banach algebra is a banach space which forms an algebra, such that ||a.b|| <= ||a||.||b||
	%%		involution is a linear, antisymmetric map 
	%%			such that if you do it twice, then you get back what you started with
 \end{itemize}
\end{frame}


\begin{frame}{Gelfand-Naimark theorem}{Commutative}
 \begin{itemize}
 	\item \emph{Every commutative, unital C$^\ast$-algebra $A$ is isometrically $\ast$-isomorphic to the algebra of continuous functions on the algebra of characters on A.} 
 \end{itemize}
 A relation between commutative C$^\ast$-algebras, and the space of continuous functions on a compact topological space.  %% that space being the character space of A
 %% give defn of isometric *-isomorphism f:
 %% 	||f(x)|| = ||x|| for x in A
 %% 	algebra isomorphism such that f(x*) = f(x)*, all x in a 
 
  Gives us a way to explore non-commutative analogues to geometry and topology.
\end{frame}




\begin{frame}{Gelfand-Naimark theorem}{General}
 \begin{itemize}
 	\item \emph{Every C$^\ast$-algebra $A$ is isometrically $\ast$-isomorphic to the algebra of bounded operators on a Hilbert space.} 
 \end{itemize} 

\end{frame}


\appendix{}
\section{References}
\begin{frame}{References}{}
\begin{itemize}
\item{}
	Kadison, R. V. \& Ringrose, J. R.,
	\emph{Fundamentals of the theory of operator algebras: Vol. I. Elementary theory.}
	
\item{}
	MacKinnon, E.,
	\emph{Heisenberg, Models, and the Rise of Matrix Mechanics.}

\item{}
	Schroer, B.,
	\emph{Pascual Jordan, Glory and Demise and his legacy in contemporary local quantum physics.}
\end{itemize}
\end{frame}




\end{document}