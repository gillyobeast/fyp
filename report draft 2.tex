\documentclass[12pt,a4paper]{amsart}

%\usepackage{fullpage}
\usepackage{graphicx}

\theoremstyle{plain}
\newtheorem{thm}{Theorem}
\newtheorem*{thm*}{Theorem}
\theoremstyle{definition}
\newtheorem{defn}{Definition}
\newtheorem*{defn*}{Definition}


\author{Luke Armitage}
\title{C$^\ast$-Algebras, and the~Gelfand-Naimark~Theorem}

%%% note on style: use C$^\ast$ unless italicised

%%% TODO:

%%% definition section - how quick an overview? what should i include?
%%% 
%%% which statement of [thm] do i want use?
%%% 
%%% 
%%%	
%%% 
%%%	
%%% 
%%% 

\begin{document}
\maketitle
\section{Introduction}
\subsection{History of C$^\ast$-Algebras}
	The noncommutative nature of Werner Heisenberg's work in 1925 on a new quantum mechanics \cite{heisenberg25} lead to Born and Jordan \cite{bornjordan25}, together with Heisenberg \cite{bornjordanheisenberg25}, developing the matrix mechanics required to concisely summarise the new quantum mechanical model. 
	From 1935-1943, John von Neumann, together with F. J. Murray, developed the theory of \emph{rings of operators} acting on a Hilbert space \cite{vonneumann35,vonneumann37,vonneumann43,vonneumann40}, in an attempt to establish a general framework for this matrix mechanics.
	These rings of operators are now considered part of the theory of \emph{von Neumann algebras}, a subsection of C$^\ast$-algebra theory. 
	Discussion of the seminal quantum mechanical works of Heisenberg can be found in \cite{mackinnon77}, and likewise in \cite{schroer03}, for the works of Jordan expanding on this.
	
	In 1943 \cite{gelfand43}, Gelfand and Naimark established an abstract characterisation of C$^\ast$-algebras, free from dependence on the operators acting on a Hilbert space.
	The Gelfand-Naimark theorem, which we will be considering here at length, gives the link between these abstract C$^\ast$-algebras and the rings of operators previously studied.
	Used in the proof of the G-N theorem is the Gelfand-Naimark-Segal construction, a pair of results relating cyclic $\ast$-representations of C$^\ast$-algebras to certain linear functionals on that algebra. 
	
	
\subsection{Background Mathematics and Resources.}	
	The following is some mathematics which may prove useful throughout the project, with relevant resources; we will of course be making definitions as needed, this is just for further background and related theory.

	We will be assuming some familiarity with the following theory, giving some explanation as necessary as we go:
\begin{itemize}
	\item Rings, algebras and linear spaces.
	\item Normed spaces, inner product spaces, Banach and Hilbert spaces.
	\item Point-set topology.
\end{itemize}
	A good broad background on all of these can be found in \cite{simmons83}.

	Some texts which cover C$^\ast$-algebras: 
	Dixmier \cite{dixmier77} presents a summary of the general theory up to that time (1977), with \cite{dixmier81} focussing on reworking and developing the theory of von Neumann algebras. 
	Sakai \cite{sakai71} gives a treatment of C$^\ast$- and von Neumann algebras from a more topological point of view.
	In \cite{kadison83,kadison86}, the authors aim to make accessible the ``vast recent research literature'' in this subsection of functional analysis.
	Blackadar \cite{blackadar06} gives a much faster, more encyclopaedic coverage of the theory of operator algebras, and covering more specialised material and applications.
	

\subsection{Aims}
The aims for this project are:
\begin{itemize}
	\item Give a good background understanding on C$^\ast$-algebras, including topological and geometric interpretation of results where possible.
	\item Consider the representation theory of C$^\ast$-algebras, using the Gelfand-Naimark-Segal construction as a starting point.
	\item Consider the commutative and general versions of the Gelfand-Naimark theorem, and understand their contents and proof.
\end{itemize}
	
\subsection{The Gelfand-Naimark theorem}


We can define C$^\ast$-algebras in two ways -- either as an algebra of bounded linear operators acting on a Hilbert space satisfying two conditions, or abstractly as a normed algebra together with an involution map satisfying four axioms.

	To get to the abstract definition of a C$^\ast$-algebra, we need to know about the following.
\begin{itemize}
	\item Banach spaces.
	\item Banach algebras.
	\item $\ast$-algebras.
\end{itemize}

A Banach space is a complete normed linear space. 
A Banach algebra is a Banach space $A$ which forms an algebra, such that 
\begin{align*}
	\|ab\| \leq \|a\| \|b\| \mbox{ for all } a,b \in A.
\end{align*}
A $\ast$-algebra is an algebra $A$ with an \emph{involution} map $a \mapsto a^\ast$ on $A$ such that, for all $a,b \in A$,
\begin{align*}
	 a^{\ast\ast} = (a^\ast)^\ast = a,\\
	(a+b)^\ast = a^\ast + b^\ast,\\	
	(ab)^\ast = b^\ast a^\ast.
\end{align*}


\begin{defn*}
A \emph{C$^\ast$-algebra} is a Banach algebra $(A, \| \cdot \|)$ with involution map $a \mapsto a^\ast$, with the condition that
\begin{align*}
	\|a ^\ast a\| = \|a\|^2 \mbox{ for all } a \in A.
\end{align*}
\end{defn*}
\noindent This condition is known as the \emph{C$^\ast$ axiom}. 


We are now, finally, in a position to state the theorems we're interested in.

\begin{thm*}[Gelfand-Naimark, Commutative]
	Every commutative, unital C*-algebra $A$ is isometrically $\ast$-isomorphic to the algebra of continuous functions on the algebra of characters on A.
\end{thm*}

This theorem first appeared in \cite{gelfand43} in a form relating normed rings to subrings of the set of bounded operators on a Hilbert space.

A \emph{faithful representation} of a C*-algebra $A$ on a Hilbert space $\mathcal{H}$ is an isomorphism of $\ast$-algebras from $A$ to the algebra of bounded operators on $\mathcal{H}$.
\begin{thm*}[Gelfand-Naimark]
	Each C*-algebra has a faithful representation.
\end{thm*}


\subsection{Further research}
	Research in C*-algebras is still very much active, with much work going into, for example: 
	amenable C* algebras and classification of approximately-finite dimensional (AF) C* algebras; 
	actions of compact groups on C* algebras; 
	classification of separable simple nuclear C* algebras; 
	single operator theory and spectral theory; 
	operator K-theory, K-homology and KK-theory; 
	*-derivations;
	homogeneous C*-algebras.
	
	
	Lin \cite{lin01} gives a summary of the state of the K-theoretical classification of amenable C*-algebras, building on Elliott's summary \cite{elliott94} of the topic. 
	Lin is also prolific in research on classification of C*-algebras in general, for example \cite{lin08,lin11} are considered important papers on the topic.
	The Ph.D. thesis of Gardella \cite{gardella15} concerns itself in part with classifying the actions of compact groups on C*-algebras.
	Pedersen \cite{pedersen79} covers most of the usual basic theory before moving on to cover a great deal of advanced theory, including automorphism groups of C*-algebras and spectral theory for such groups.
	J.M.G. Fell \cite{fell61} uses theory of fibre bundles, homogeneous algebras and algebras of continuous trace, to get at the group C*-algebra for SL$(2,\mathbb{C})$.
	Niemiec \cite{niemiec12} gives an elementary proof of a result \cite[Theorem 3.2]{fell61} from Fell, on $n$-homogeneous C*-algebras, and proposes a spectral theorem for these $n$-homogeneous systems.
	
	


\begin{thebibliography}{00}
\bibitem{blackadar06}
	Blackadar, B.,
	\emph{Operator Algebras: Theory of C*-Algebras and von Neumann Algebras.}
	Encyclopaedia of Mathematical Sciences, 122. Operator Algebras and Non-commutative Geometry, III. Springer-Verlag, Berlin (2006).
	
\bibitem{bornjordan25}
	Born, M. \& Jordan, P.,
	\emph{Zur Quantenmechanik.}
	Z. Physik (1925) 34: 858.
	
\bibitem{bornjordanheisenberg25}
	Born, M.; Heisenberg, W. \& Jordan, P.,
	\emph{Zur Quantenmechanik. II.}
	Z. Physik (1926) 35: 557.

\bibitem{dixmier77}
	Dixmier, J.,
	\emph{C*-algebras.}
	Holland Mathematical Library, Vol. 15. North-Holland Publishing Co., Amsterdam -- New York -- Oxford (1977).

\bibitem{dixmier81}
	Dixmier, J.,
	\emph{von Neumann algebras.}
	North-Holland Publishing Co., Amsterdam -- New York (1981).
		
\bibitem{elliott94}
	Elliott, G. A.,
	\emph{The classification problem for amenable C*-algebras.}
	Proceedings of the International Congress of Mathematicians, Vol. 1, 2 (Z\"{u}rich, 1994), pp.~922-–932, Birkhäuser, Basel (1995).

\bibitem{fell61}
	Fell, J. M. G.,
	\emph{The structure of algebras of operator fields. }
	Acta Math. 106, 1961, 233–280. 


\bibitem{gardella15}
	Gardella, E. E.,
	\emph{Compact group actions on C*-algebras: classification, non-classifiability, and crossed products and rigidity results for L$^p$-operator algebras. }
	Thesis (Ph.D.) -– University of Oregon (2015).

\bibitem{gelfand43}
	Gelfand, I. \& Neumark, M.,
	\emph{On the imbedding of normed rings into the ring of operators in Hilbert space.}
	Rec. Math. [Mat. Sbornik] N.S. 12(54) (1943), pp.~197–-213.

\bibitem{heisenberg25}
	Heisenberg, W.,
	\emph{{\"U}ber quantentheoretische Umdeutung kinematischer und mechanischer Beziehungen.}
	Z. Physik (1925) 33: 879. 
	
\bibitem{kadison83}
	Kadison, R. V. \& Ringrose, J. R.,
	\emph{Fundamentals of the theory of operator algebras: Vol. I. Elementary theory.}
	Pure and Applied Mathematics, 100. Academic Press, Inc. [Harcourt Brace Jovanovich, Publishers], New York (1983).

\bibitem{kadison86}	
	Kadison, R. V. \& Ringrose, J. R.,
	\emph{Fundamentals of the theory of operator algebras: Vol. II. Advanced theory.}
	Pure and Applied Mathematics, 100. Academic Press, Inc., Orlando, FL (1986).
	
\bibitem{lin01}
	Lin, H.,
	\emph{An introduction to the classification of amenable C*-algebras.}
	World Scientific Publishing Co., Inc., River Edge, NJ (2001).
\bibitem{lin08}
	Lin, H. \& Niu, Z.,
	\emph{Lifting KK-elements, asymptotic unitary equivalence and classification of simple C*-algebras.}
	Adv. Math. 219 (2008), no. 5, pp.~1729-–1769. 
	
\bibitem{lin11}
	Lin, H.,
	\emph{Asymptotic unitary equivalence and classification of simple amenable C*-algebras.}
	Invent. Math. 183 (2011), no. 2, pp.~385-–450. 

\bibitem{mackinnon77}
	MacKinnon, E.,
	\emph{Heisenberg, Models, and the Rise of Matrix Mechanics.}
	Hist. Stud. Phys. Sci., Vol. 8 (1977), pp.~137--188
	
\bibitem{vonneumann35}
	Murray, F. J. \& von Neumann, J.,
	\emph{On rings of operators.}
	Ann. of Math. (2) 37 (1936), no. 1, pp.~116–-229.

\bibitem{vonneumann37}
	Murray, F. J. \& von Neumann, J.,
	\emph{On rings of operators. II.}
	Trans. Amer. Math. Soc. 41 (1937), no. 2, pp.~208-–248. 

\bibitem{vonneumann43}
	Murray, F. J. \& von Neumann, J.,
	\emph{On rings of operators. IV.}
	Ann. of Math. (2) 44, (1943), pp.~716-–808.
	
\bibitem{niemiec12}
	Niemiec, P.,
	\emph{Elementary Approach to Homogeneous C*-algebras.}
	Rocky Mountain J. Math. 45 (2015), no. 5, pp~1591--1630.	
	
\bibitem{pedersen79}
	Pedersen, G. K.,
	\emph{C*-algebras and their automorphism groups.}
	Academic Press, Inc. [Harcourt Brace Jovanovich, Publishers], London-New York (1979).

\bibitem{sakai71}
	Sakai, S.,
	\emph{C*-algebras and W*-algebras.}
	Ergebnisse der Mathematik und ihrer Grenzgebiete, Band 60. Springer-Verlag, New York-Heidelberg (1971).

\bibitem{schroer03}
	Schroer, B.,
	\emph{Pascual Jordan, Glory and Demise and his legacy in contemporary local quantum physics.}
	Unpublished manuscript (2003).
	
\bibitem{simmons83}
	Simmons, G.,
	\emph{Introduction to Topology and Modern Analysis.}
	Robert E. Krieger Publishing Co., Inc., Melbourne, Fl. (1983).

\bibitem{vonneumann40}
	von Neumann, J.,
	\emph{On rings of operators. III.}
	Ann. of Math. (2) 41, (1940), pp.~94–-161.

	
\end{thebibliography}

\end{document}