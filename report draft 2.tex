\documentclass[12pt,a4paper]{amsart}

%\usepackage{fullpage}
\usepackage{graphicx}

\theoremstyle{plain}
\newtheorem{thm}{Theorem}
\newtheorem*{thm*}{Theorem}
\theoremstyle{defn}
\newtheorem{defn}{Definition}



\author{Luke Armitage}
\title{C$^\ast$-Algebras, the~Gelfand-Naimark Theorem, and~[other thing]}

%%% note on style: use C$^\ast$

\begin{document}
\maketitle
\section{Introduction}
\subsection{History of C$^\ast$-Algebras}
	A brief timeline of the development on the theory.
The theory of C$^\ast$-algebras has its basis in W. Heisenberg's use of so-called ``matrix mechanics'' in the modelling of algebras of physical observables in quantum mechanical systems. In [1925], Heisenberg 
	
\subsection{C$^\ast$-Algebras}
	A brief overview of what a C$^\ast$-algebra is.
\begin{enumerate}
	\item[$\bullet$] banach space
	\item[$\bullet$] banach algebra
	\item[$\bullet$] *-algebra
	\item[$\bullet$] C* algebra
	\item[$\bullet$] representations
\end{enumerate}
\begin{defn}
A \emph{C$^\ast$-algebra} is a Banach algebra $(A, \| \cdot \|)$ with involution $\ast : A \to A$ with the condition that
\begin{align*}
	\|x ^\ast x\| = \|x\|^2 \mbox{ for all } x \in A.
\end{align*}
\end{defn}
 I HAVE CHANGED A THING
This condition is known as the \emph{C$^\ast$ axiom}. The study of C$^\ast$-algebras started with the consideration of matrix mechanics by H

There are many statements of the theorem; this here comes from [ref].
\begin{thm*}[Gelfand-Naimark, Commutative]
	Every commutative, unital C*-algebra A is isometrically $\ast$-isomorphic to the algebra of continuous functions on the algebra of characters on A.

\end{thm*}
The theorem first appeared in [gelfand-naimark] in a form relating `normed rings' and a closed subrings of the set of bounded operators on a Hilbert space.


\subsection{Aims}
The aims for my project are, provisionally:
\begin{enumerate}

	\item[$\bullet$] Take the Gelfand-Naimark theorem, and understand its contents and proof.
	\item[$\bullet$] Consider the representation theory of C$^\ast$-algebras, using the Gelfand-Naimark-Segal construction as a starting point.
	\item[$\bullet$] (as 'further reading') Give an overview of areas in which operator algebra theory can be taken (for example, Cuntz algebras and operator K-theory, von Neumann algebras and Factors, abstract harmonic analysis).
\end{enumerate}

\begin{thebibliography}{00}
	
	
	
\end{thebibliography}

\end{document}