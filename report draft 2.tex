\documentclass[12pt,a4paper]{amsart}

%\usepackage{fullpage}
\usepackage{graphicx}

\theoremstyle{plain}
\newtheorem{thm}{Theorem}
\newtheorem*{thm*}{Theorem}
\theoremstyle{definition}
\newtheorem{defn}{Definition}



\author{Luke Armitage}
\title{C$^\ast$-Algebras, the~Gelfand-Naimark Theorem, and~[other thing]}

%%% note on style: use C$^\ast$

%%% TODO:

%%% think about [other thing]
%%% add more references to bibliography
%%% figure out how to cite schroer03
%%% find a better word to replace convey, history section
%%% finish sentence in history section
%%%	write definition section - how quick an overview?
%%% write about prereq knowledge
%%%	write about further/recent research
%%% which statement of [thm] do i want use?
%%% write about GNS in first section too

\begin{document}
\maketitle
\section{Introduction}
\subsection{History of C$^\ast$-Algebras}
	The noncommutative nature of W. Heisenberg's work on (finish sentence) \cite{heisenberg25} lead to Born and Jordan \cite{bornjordan25}, together with Heisenberg \cite{bornjordanheisenberg25}, developing the matrix mechanics required to concisely convey the new quantum mechanical model. 
	From 1930-1943, J. von Neumann, together with F. J. Murray, developed the theory of \emph{rings of operators} acting on a Hilbert space \cite{vonneumann35}, \cite{vonneumann37}, \cite{vonneumann40}, \cite{vonneumann43}, in an attempt to establish a general framework for this matrix mechanics.
	These rings of operators are now considered part of the theory of \emph{von Neumann algebras}, a subsection of C$^\ast$-algebra theory. 
	Discussion of the seminal quantum mechanical works of Heisenberg can be found in \cite{mackinnon77}, and likewise in \cite{schroer03}, for the works of Jordan expanding on this.
	
	In \cite{gelfand43}, Gelfand and Naimark established an abstract characterisation of C$^\ast$-algebras, free from dependence on the operators acting on a Hilbert space.
	The Gelfand-Naimark theorem, which we will be considering here at length, gives the link between these abstract C$^\ast$-algebras and the rings of operators previously studied.
	Used in the proof of the G-N theorem is the Gelfand-Naimark-Segal construction, a pair of results relating cyclic $\ast$-representations of C$^\ast$-algebras to certain linear functionals on that algebra. 
	
	
\subsection{C$^\ast$-Algebras}
	A brief overview of what a C$^\ast$-algebra is.
\begin{enumerate}
	\item[$\bullet$] banach space
	\item[$\bullet$] banach algebra
	\item[$\bullet$] *-algebra
	\item[$\bullet$] C* algebra
	\item[$\bullet$] representations
\end{enumerate}
\begin{defn}
A \emph{C$^\ast$-algebra} is a Banach algebra $(A, \| \cdot \|)$ with involution $\ast : A \to A$ with the condition that
\begin{align*}
	\|x ^\ast x\| = \|x\|^2 \mbox{ for all } x \in A.
\end{align*}
\end{defn}
 
This condition is known as the \emph{C$^\ast$ axiom}. The study of C$^\ast$-algebras started with the consideration of matrix mechanics by H

\clearpage

\begin{thm*}[Gelfand-Naimark, Commutative]
	Every commutative, unital C*-algebra A is isometrically $\ast$-isomorphic to the algebra of continuous functions on the algebra of characters on A.

\end{thm*}
The theorem first appeared in [gelfand-naimark] in a form relating `normed rings' and a closed subrings of the set of bounded operators on a Hilbert space.


\subsection{Aims}
The aims for my project are, provisionally:
\begin{enumerate}

	\item[$\bullet$] Take the Gelfand-Naimark theorem, and understand its contents and proof.
	\item[$\bullet$] Consider the representation theory of C$^\ast$-algebras, using the Gelfand-Naimark-Segal construction as a starting point.
	\item[$\bullet$] (as 'further reading') Give an overview of areas in which operator algebra theory can be taken (for example, Cuntz algebras and operator K-theory, von Neumann algebras and Factors, abstract harmonic analysis).
\end{enumerate}

\begin{thebibliography}{00}
\bibitem{bornjordan25}
	Born, M. \& Jordan, P.,
	\emph{Zur Quantenmechanik.}
	Z. Physik (1925) 34: 858.
	
\bibitem{bornjordanheisenberg25}
	Born, M.; Heisenberg, W. \& Jordan, P.,
	\emph{Zur Quantenmechanik. II.}
	Z. Physik (1926) 35: 557.

\bibitem{gelfand43}
	Gelfand, I. \& Neumark, M.,
	\emph{On the imbedding of normed rings into the ring of operators in Hilbert space.}
	Rec. Math. [Mat. Sbornik] N.S. 12(54), (1943), pp.~197–-213.

\bibitem{heisenberg25}
	Heisenberg, W.,
	\emph{{\"U}ber quantentheoretische Umdeutung kinematischer und mechanischer Beziehungen.}
	Z. Physik (1925) 33: 879. 

\bibitem{mackinnon77}
	MacKinnon, E.,
	\emph{Heisenberg, Models, and the Rise of Matrix Mechanics.}
	Hist. Stud. Phys. Sci., Vol. 8 (1977), pp.~137--188


\bibitem{schroer03}
	Schroer, B.,
	\emph{Pascual Jordan, Glory and Demise and his legacy in contemporary local quantum physics.}
	(no idea where published)

\bibitem{vonneumann35}
	Murray, F. J. \& von Neumann, J.,
	\emph{On rings of operators.}
	Ann. of Math. (2) 37 (1936), no. 1, pp.~116–-229.

\bibitem{vonneumann37}
	Murray, F. J. \& von Neumann, J.,
	\emph{On rings of operators. II.}
	Trans. Amer. Math. Soc. 41 (1937), no. 2, pp.~208-–248. 

\bibitem{vonneumann43}
	Murray, F. J. \& von Neumann, J.,
	\emph{On rings of operators. IV.}
	Ann. of Math. (2) 44, (1943), pp.~716-–808.

\bibitem{vonneumann40}
	von Neumann, J.,
	\emph{On rings of operators. III.}
	Ann. of Math. (2) 41, (1940), pp.~94–-161.

	
\end{thebibliography}

\end{document}