\documentclass[12pt,a4paper]{report}
\usepackage{graphicx}
\usepackage[color=purple!80]{todonotes} %% does the to do stuff
\usepackage{enumitem}
%%\usepackage{fullpage}
\usepackage{bbm}
\usepackage{hyperref}
\usepackage{mathrsfs}
\usepackage{amsmath}
\usepackage{amsthm}
\usepackage{amsfonts}





\theoremstyle{plain}

\newtheorem{thm}{Theorem}
\newtheorem*{thm*}{Theorem}
\newtheorem*{claim}{Claim}
\newtheorem{prop}{Proposition}
\newtheorem*{prop*}{Proposition}
\theoremstyle{definition}
\newtheorem{defn}{Definition}
\newtheorem*{defn*}{Definition}

\renewcommand{\H}{\mathcal{H}}
\newcommand{\B}{\mathcal{B}}
\newcommand{\BH}{\mathcal{\B(\H)}}
\newcommand{\Hr}{\mathcal{H}_\rho}
\newcommand{\1}{\mathbbm{1}}
\newcommand{\C}{\mathbb{C}}
\newcommand{\R}{\mathbb{R}}
\newcommand{\N}{\mathbb{N}}
\newcommand{\Vr}{V_\rho}
\newcommand{\Lr}{L_\rho}
\newcommand{\xr}{x_\rho}
\renewcommand{\phi}{\varphi}
\newcommand{\CX}{C(X)}
\renewcommand{\S}[1]{\mathscr{S}(#1)}
\renewcommand{\P}[1]{\mathscr{P}(#1)}
\newcommand{\spec}[1]{\mbox{sp}(#1)}
\renewcommand{\labelenumi}{(\roman{enumi})}

\author{Luke Armitage}
\title{$C^\ast$-Algebras, and the~Gelfand-Naimark~Theorem}

\begin{document}
\maketitle
\makeatletter  						%%todonotes magic
    \providecommand\@dotsep{5}
  \makeatother
\listoftodos\relax
\newcommand\Item[1][]{ 				%% enumerate magic
  \ifx\relax#1\relax  \item \else \item[#1] \fi
  \abovedisplayskip=0pt\abovedisplayshortskip=0pt~\vspace*{-\baselineskip}}
\tableofcontents

\section{Preliminaries}

\subsection{History of C$^\ast$-Algebras}
	The noncommutative nature of Werner Heisenberg's work in 1925 on a new quantum mechanics \cite{heisenberg25} 
	led to Born and Jordan \cite{bornjordan25}, together with Heisenberg \cite{bornjordanheisenberg25}, 
	developing the matrix mechanics required to concisely summarise the new quantum mechanical model. 
	From 1935-1943, John von Neumann, together with F.J. Murray, developed the theory of \emph{rings of 
	operators} acting on a Hilbert space \cite{vonneumann35,vonneumann37,vonneumann43,vonneumann40}, in an 
	attempt to establish a general framework for this matrix mechanics.
	These rings of operators are now considered part of the theory of \emph{von Neumann algebras}, a subsection 
	of C$^\ast$-algebra theory. 
	Discussion of the seminal quantum mechanical works of Heisenberg can be found in \cite{mackinnon77}, and 
	similarly \cite{schroer03} gives a summary of the works of Jordan expanding on this.
	
	In 1943 \cite{gelfand43}, Gelfand and Naimark established an abstract characterisation of C$^\ast$-algebras, 
	free from dependence on the operators acting on a Hilbert space.
	The Gelfand-Naimark theorem, which we will be considering here at length, gives the link between these 
	abstract C$^\ast$-algebras and the rings of operators previously studied.
	Used in the proof of the GN theorem is the Gelfand-Naimark-Segal construction, a pair of results relating 
	cyclic $\ast$-representations of C$^\ast$-algebras to certain linear functionals on that algebra. 
	
\subsection{Background Mathematics and Resources.}	
	The following is some mathematics which may prove useful throughout the project, with relevant resources; 
	we will of course be making definitions as needed, this is for further background and related theory.

	We will be assuming some familiarity with the following theory, giving some explanation as necessary:
\begin{itemize}
	\item Rings, algebras and linear spaces.
	\item Normed spaces, inner product spaces, Banach and Hilbert spaces.
	\item Point-set topology.
\end{itemize}
	A good broad background on all of these can be found in \cite{simmons83}.

	Some texts which cover C$^\ast$-algebras: 
	Dixmier \cite{dixmier77} presents a summary of the general theory up to that time (1977), with 
	\cite{dixmier81} focussing on reworking and developing the theory of von Neumann algebras. 
	Sakai \cite{sakai71} gives a treatment of C$^\ast$- and von Neumann algebras from a more topological point 
	of view. In \cite{kadison83,kadison86}, the authors aim to make accessible the ``vast recent research 
	literature'' in this subsection of functional analysis.
	Blackadar \cite{blackadar06} gives a much faster, more encyclopaedic coverage of the theory of operator 
	algebras, and covering more specialised material and applications.
	
\subsection{Aims}
	The aims for this project are:
\begin{itemize}
	\item Give a good background understanding on C$^\ast$-algebras, including topological and geometric interpretation of results where possible.
	\item Consider the representation theory of C$^\ast$-algebras, using the Gelfand-Naimark-Segal construction as a starting point.
	\item Consider the commutative and general versions of the Gelfand-Naimark theorem, and understand their contents and proof.
\end{itemize}

List of stuff we're gonna go right ahead an assume:
\begin{itemize}
\item	Familiarity with algebras, Banach spaces, Hilbert spaces and other guff.
\end{itemize}


we will assume knowledge on... (algebra homos map 1 to 1,hausdorff/compact spaces)\\
brief(er than asst 3) history \\
most texts start from $\BH$ to justify the whole thing. we're algebraists, who don't need no justification. we jump right in at the deep (abstract) end.
\todo[inline]{write this!} %%


\section{Definitions}
We begin this section with some definitions and an example, then give some results we will need later and finish 
with a fundamental example.
\begin{defn}
	A \emph{Banach algebra} is a complex Banach space $(A,\|\cdot\|)$ which forms an 
	algebra, such that 
	\begin{align*}
		\|ab\| \leq \|a\| \|b\| \mbox{ for all } a,b \in A.
	\end{align*}
	A \emph{$\ast$-algebra} is an algebra $A$ with an \emph{involution} map 
	$a \mapsto a^\ast$ on $A$ such that, for all $a,b \in A$ and for $\alpha \in \C$,
	
	\begin{enumerate}
		\item $a^{\ast\ast} = (a^\ast)^\ast = a$,
		\item $(\alpha a+b)^\ast = \overline{\alpha} a^\ast + b^\ast$,
		\item $(ab)^\ast = b^\ast a^\ast$.
	\end{enumerate}
	The element $a^\ast$ is referred to as the \emph{adjoint} of $a$. 		\\
	A \emph{C*-algebra} is a Banach algebra $(A, \| \cdot \|)$ with involution 
	map $a \mapsto a^\ast$ making it a $\ast$-algebra, with the condition that
	\begin{align*}
		\|a ^\ast a\| = \|a\|^2 \mbox{ for all } a \in A.
	\end{align*}
	This condition is known as the \emph{$C^\ast$ axiom}. There is a weaker, but 
	ultimately equivalent, axiom called the \emph{$B^\ast$ axiom}.\todo{state and prove} %%
\end{defn}
Here we consider complex $C^\ast$-algebras. The theory of real $C^\ast$-algebras has advanced....
\todo[inline]{remark on work in real $C^\ast$-algebras?} %%

Unless specified otherwise, by an \emph{ideal} of a Banach algebra, we mean a two-sided ideal.
Given a subset $S$ of a $C^\ast$-algebra $A$, let $C^\ast (S)$ denote the \emph{$C^\ast$-subalgebra of
$A$ generated by $S$}, which is the smallest $S^\ast$-subalgebra of $A$ containing $S$.
\todo[inline]{cauchy-schwarz, C* subalgebra generated by a set}


\subsection{$\CX$ - an example.}\todo{maybe move this to as early as we can?}%%
Given a locally compact Hausdorff space $X$, let $\CX$ be the algebra of continuous functions
$f:X\to\C$, with addition and multiplication defined pointwise. Define $\|\cdot\|$ on $\CX$ by
\[
	\|f\|:= \sup_{x\in X}{|f(x)|},
\]
that is the norm inherited from the Banach space $\ell^2(X,\C)$.





\todo [inline]{example of $C(X)$?} %%


\subsection{Unitization}
If a $C^\ast$-algebra $A$ contains an identity element $\1$ such that 
$a\cdot \1 = a = \1 \cdot a$ for all $a \in A$, call $\1$ the \emph{unit} 
in $A$, and $A$ is then a \emph{unital} $C^\ast$-algebra.

\begin{prop}
	Any non-unital $C^\ast$-algebra $A$ can be isometrically embedded 
	in a unital $C^\ast$-algebra $\tilde{A}$ as a maximal ideal.
\end{prop}
\todo[inline]{tidy up proof}%%
\begin{proof}
	Let $\tilde{A} = A \oplus \C$ with pointwise addition, and define
	\begin{align*}
		(a,\lambda) (b,\mu) &:= (ab+\lambda b + \mu a, \lambda \mu),		\\
		(a,\lambda)^\ast &:= (a^\ast,\overline{\lambda}),					\\
		\|(a,\lambda)\| &:= \sup_{\|b\|=1}{\|ab+\lambda b\|}.
	\end{align*}
	Then $\tilde{A}$ is a $\ast$-algebra. The norm $\|(a,\lambda)\|$ is the norm 
	in $\B(A)$ of left-multiplication by $a$ on something iunno???
	Thus $\tilde{A}$ is a Banach $\ast$-algebra with unit $(0,1)$. 
	By design, $A$ is a maximal ideal of codimension 1. 
	The embedding $a\mapsto(a,\lambda)$ is isometric as 
	\[
		\|a\| = \|a\cdot\frac{a}{\|a\|}\| \leq \|(a,0)\| \leq \sup_{\|b\|=1}{\|ab\|} \leq \|a\|.
	\]
	It remains to verify the $C^\ast$-axiom:
	\begin{align*}
				\|(a,\lambda)\|^2 
		&=		\sup_{\|b\|=1}{\|ab+\lambda b\|^2}								\\
		&=		\sup_{\|b\|=1}{\|b^\ast a^\ast ab 
								+\lambda b^\ast a^\ast b
								+\overline{\lambda}b^\ast a b
								+|\lambda|^2 b^\ast b}							\\
		&\leq	\sup_{\|b\|=1}{\|a^\ast ab 
								+\lambda a^\ast b
								+\overline{\lambda}a b
								+|\lambda|^2 b\|}									\\
		&=		\|(a^\ast a + \lambda a^\ast +\overline{\lambda}a,|\lambda|^2)	\\
		&= 		\|(a,\lambda)^\ast(a,\lambda)\|									\\
		&\leq	\|(a,\lambda)^\ast\| \|(a,\lambda)\|.
	\end{align*}
	By symmetry of $\ast$, $\|(a,\lambda)^\ast\| = \|(a,\lambda)\|$. 
	Hence, the above inequality becomes equality and we have that
	\begin{align*}
		\|(a,\lambda)^\ast(a,\lambda)\| = \|(a,\lambda)\|^2. 
	\end{align*}		
	
\end{proof}

In light of this result, we take all $C^\ast$-algebras from here to be unital unless
specified otherwise.
For the results we will consider, we can simply consider the unital case. 
However, there are important circumstances in advanced theory in which one 
needs to relax the unital condition.


\subsection{the spectrum}
Given an element $a \in A$ of a $C^\ast$-algebra, define its spectrum $\spec{a}$:
\[
	\spec{a}:=\{\lambda\in \C ~|~ a-\lambda \1 \mbox{ is not invertible in }A\}.
\]

state without proof results?
AAAAAAAAAAH I NEED SO MUCH SPECTRAL THEORY AND I DON'T KNOW ANY

\subsection{more definitions}
An element $a\in A$ of a $C^\ast$-algebra is called \todo{normal?} %%
\begin{itemize}
	\item \emph{self-adjoint} if $a^\ast=a$;
	\item \emph{unitary} if $aa^\ast=a^\ast a = \1$;
	\item \emph{positive} if it is self-adjoint and sp$(a)$ $\subseteq \R^+$.
\end{itemize}
Denote the set of self-adjoint elements in $A$ by $A_{sa}$, and the subset of positive 
elements in $A_{sa}$ by $A^+$. The set of positive elements $A_{sa}$ forms a \emph{partially ordered 
(real) vector space}, with \emph{positive cone} $A^+$. That is to say, all $f,g \in A^+$ satisfy
\begin{enumerate}
	\item 	$f,-f\in A^+$ implies $f=0$,
	\item	$\alpha f\in A^+$ for all $\alpha\in \R^+$,
	\item	$f+g \in A^+$.
\end{enumerate}
The unit $\1$ is positive, and for any $a\in A_{sa}$ we have $-\|a\|\1 \leq a \leq \|a\|\1$. 
With commuting elements $a,b\in A_{sa}$, we have $(ab)^\ast = b^\ast a^\ast = ba = ab$, so 
$ab$ is self-adjoint. Since $a,b,ab$ have the same spectrum in $A$ as in the Abelian 
$C^\ast$-subalgebra $C^\ast(\1,a,b)$, by our spectral theory we have \todo{state in 'spectrum' - 4.1.5}
\begin{align*}
	\spec{ab} \subseteq \spec{a}\spec{b}.
\end{align*}

\begin{defn}
	Given Banach $\ast$-algebras $A$ and $B$, a map $\phi:A\to B$ is a \emph{$\ast$-homomorphism} 
	if it is an algebra homomorphism for which $\phi(a^\ast) = \phi(a)^\ast$ for all $a \in A$.
	If $A$ and $B$ are both unital algebras and a homomorphism $\phi$ maps $\1_A$ to $\1_B$, 
	say $\phi$ is a \emph{unital} homomorphism.    \todo{is unital necessary?} %%
	If a $\ast$-homomorphism $\phi$ is one-to-one, call it a \emph{$\ast$-isomorphism}.
\end{defn}

\begin{prop}
	Suppose $A$ and $B$ are $C^\ast$-algebras and $\phi:A\to B$ is a $\ast$-homomorphism. Then
	$\|\phi(a)\| \leq \|a\|$ for all $a \in A$. If $\phi$ is a $\ast$-isomporphism, then
	$\|\phi(a)\| = \|a\|$ for all $a \in A$.
\end{prop}
\begin{proof}
	
\end{proof}
\begin{defn}
	A \emph{linear functional} on a $C^\ast$-algebra $A$ is a linear operator $\rho:A \to \C$.
	A linear functional $\rho$ is \emph{positive} if $\rho(a)\geq 0$ for all $a\in A^+$.
	A \emph{multiplicative} linear functional $\rho$ satisfies $\rho(ab) =\rho(a)\rho(b)$ 
	for all $a,b\in A$.
	A \emph{state} on $A$ is a positive linear functional $\rho$ such that $\|\rho\|=1$ and 
	$\rho (a) \geq 0$  for all positive elements $a \in A^+$. 
	Denote by $\S{A}$ the set of all states on $A$.
	An extreme point of $\S{A}$ is called a \emph{pure} state on $A$, and the set of pure 
	states on $A$ is denoted by $\P{A}$.
\end{defn}
\todo{show that $\|\rho\| = \rho(\1)$. explain extreme point.} %%

It is a simple exercise, using the fact that $A^+\subseteq A_{sa}$, to verify that a linear functional 
$\rho$ is a pure state on $A$ if and only if the restriction $\rho|_{A_{sa}}$ is a pure state on $A_{sa}$.
Every pure state on $A_{sa}$ extends to a pure state on $A$. \todo{justify. 4.3.13.} %%
We will need the following few results on pure states later.
\begin{prop}\label{pure1}
	A state $\rho$ on $A_{sa}$ is pure if and only if, for all positive linear functionals $\tau$ on $A_{sa}$
	such that $0\leq\tau\leq\rho$, we have $\tau =\lambda\rho$ for some $\lambda\in\R$.
\end{prop}
\begin{proof}[Proof. (Adapted from 3.4.6)]
	Suppose that $\tau =\lambda\rho$ for all $0\leq\tau\leq\rho$, and suppose we can write 
	$\rho=\alpha\rho_1+(1-\alpha)\rho_2$ for some $0\leq\alpha\leq1$ and some $\rho_1,\rho_2 \in \S{A_{sa}}$. 
	Then $0\leq\alpha\rho_1\leq\rho$, so $\alpha\rho_0 = \lambda\rho$. Then $\rho_1(\1)=1=\rho(\1)$, so 
	$\alpha = \lambda$ so $\rho_0 =\rho$. Similarly, we can show that $\rho_2=\rho$, and so we conclude that 
	$\rho$ is pure.
	
	Conversely, suppose that $\rho$ is a pure state and $0\leq\tau\leq\rho$. Applying this to $\1$, we get 
	$0\leq\tau(\1)\leq\rho(\1) = 1$. Let $\lambda = \tau(\1)$.
	If $\lambda=0$, then for any $a\in A_{sa}$, applying $\tau$ to $-\|a\|\1\leq a\leq\|a\|\1$ gives
	\begin{align*}
		0=-\|a\|\lambda=\tau(-\|a\|\1)\leq\tau(a)\leq\tau(\|a\|\1)=\|a\|\lambda=0,
	\end{align*}
	so $\tau=0=\lambda\rho$. 
	A similar argument shows that $\lambda=1$ implies $\tau-\rho=0$ so that $\tau=\rho=\lambda\rho$.
	If $0\leq\lambda\leq 1$, we can write $\rho=\lambda\rho_1+(1-\lambda)\rho_2$ for $\rho_1=\lambda^{-1}\tau$ 
	and $\rho_2=(1-\lambda)^{-1}(\rho-\tau)$. $\rho$ is pure so $\tau=\lambda\rho_1=\lambda\rho$.
	
\end{proof}

\begin{prop}\label{pure2}
	The set of pure states on an Abelian $C^\ast$-algebra $A$ is precisely the set 
	of multiplicative linear functionals on $A$.
\end{prop}
\begin{proof}[Proof.(Adapted from K\&R, 4.4.1)]
	Suppose $\rho$ is a pure state on $A$. To show that $\rho(ab)=\rho(a)\rho(b)$ for $a,b,\in A$,
	we restrict attention to the case where $0\leq b\leq\1$. Linearity gives us the general case.\todo{show} ~%%
	In this case, for $h\in A^+$ we have that $0\leq hb\leq h$, so $0\leq\rho(hb)\leq\rho(h)$.
	Hence $\rho_b(a):=\rho(ab)$ for $a\in A$ defines a positive linear functional on $A$ 
	with $\rho_b\leq\rho$. 
	The restriction $\rho|_{A_{sa}}$ is a pure state on $A_{sa}$ and $\rho_b|_{A_{sa}} 
	\leq \rho|_{A_{sa}}$, and it follows from Proposition \ref{pure1} that $\rho_b|_{A_{sa}} = \alpha \rho|_{A_{sa}}$ 
	for some $\alpha \in \R^+$.
	Hence $\rho_b = \alpha\rho$  and so for $a\in A$:
	\begin{align*}
		\rho(ab) = \rho_b(a) = \alpha\rho(a) = 
							\alpha\rho(\1)\rho(a) = \rho_b(\1)\rho(a) = \rho(b)\rho(a)
	\end{align*}
	
	Conversely, suppose $\rho$ is a multiplicative linear functional. \todo{show a state.} %% 
	Suppose we can write $\rho=\alpha\rho_1+\beta\rho_2$ for states $\rho_1,\rho_2$ on $A$ and 
	$\alpha,\beta >0$ such that $\alpha+\beta=1$. For $c\in A_{sa}$, by the Cauchy-Schwarz inequality we have
	for $j=1,2$:
	\begin{align*}
		\big(\rho_j(c)\big)^2 = \big(\rho_j(\1 c)\big)^2 \leq \rho_j(\1)\rho(c^2)=\rho(c^2).
	\end{align*}
	Then:
	\begin{align*}
				0
		&=		\rho(c^2)-\rho(c)^2 											\\
		&=		\alpha\rho_1(c^2)+\beta\rho_2(c^2) 
						- \big(\alpha\rho_1(c)+\beta\rho_2(c)\big)^2			\\
		&\geq	\alpha(\alpha+\beta)\rho_1(c)^2 
						+ \beta(\alpha+\beta)\rho_2(c)^2
						- \big(\alpha\rho_1(c)+\beta\rho_2(c)\big)^2			\\
		&=		\alpha\beta\big(\rho_1(c) - \rho_2(c)\big)^2.
	\end{align*}
	Hence $\rho_1(c)=\rho_2(c)$, for all $c\in A_{sa}$, so $\rho_1=\rho_2$ and we conclude
	that $\rho$ is a pure state. 
\end{proof}




\todo[inline]{define weak* topology on P(S)}  %%


\subsection{$\B(\H)$ - an example.}
This section concerns the fundamental example of a $C^\ast$-algebra - the set $\B(\H)$ of bounded 
linear operators on a Hilbert space $\H$.  
Here we will demonstrate that $\B(\H)$ is a $C^\ast$-algebra and give some basic results.

\begin{claim} $\B(\H)$ is a $C^\ast$-algebra with the operator norm 
\[
	\|T\|:= \sup_{\|x\|=1}{\|Tx\|}
\]
and involution taking $T$ to its adjoint map $T^\ast$. 
The identity map $I:x\mapsto x$ is a unit for $\BH$
\end{claim}
\begin{proof}
	$\|\cdot\|$ is a norm on $\BH$. Let $\{T_n\}_{n\in\N}$ be a Cauchy sequence 
	in $\BH$. Then for any positive $\epsilon$, there is a positive integer $N$ such that 
	\begin{align*}
		\|T_m-T_n\| < \epsilon \mbox{ for all } m,n \geq N.
	\end{align*}
	Applying $T_m-T_n$ to $x \in \H$, we have 
	\begin{align}\label{cauchyTn}
		\|T_mx-T_nx\| \leq \|T_m-T_n\| \|x\| < \epsilon \|x\|,
	\end{align}
	so $\{T_nx\}_{n\in\N}$ is a Cauchy sequence in $\H$, converging to an element in $\H$.
	Define a linear operator $T:\H \to \H$ by 
	\begin{align*}
		Tx:= \lim_{n\to\infty}{T_nx} \mbox{ for } x \in \H.
	\end{align*}
	Taking limits as $m$ tends to infinity in equation \eqref{cauchyTn}, we obtain
	\begin{align*}
		\|Tx-T_nx\| < \epsilon \|x\| \mbox{ for all }n \geq N,
	\end{align*}
	and so we have that $T-T_n$ (and hence $T=(T-T_n)+T_n$) is a bounded operator and  
	\begin{align*}
		\|T-T_n\| <\epsilon \mbox{ for all }n \geq N.
	\end{align*}
	We conclude that $T_n \to T$, and so $\BH$ is complete.
	
	Since boundedness is equivalent to continuity on $\H$, given $S,T\in\BH$, the operator 
	$ST:\H \to \H; x \mapsto (S\circ T)(x)$ is bounded on $\H$.
	Given $x\in\H$ and $\lambda\in\C$, 
	\begin{align*}
			((\lambda S)T)(x)
		&=	((\lambda S)\circ T)(x)												\\
		&=	\lambda S(Tx)														\\
		&=	\lambda (S\circ T)(x)												\\
		&=	\lambda ST(x),
	\end{align*} 
	so that $(\lambda S)T = \lambda ST$ in $\BH$, whence $\BH$ is an algebra.
	We have 
	\begin{align*}
				\|ST\|
		&=		\sup_{\|x\|=1}{\|STx\|} 										\\
		&=		\sup_{\|x\|=1}{\|S(Tx)\|} 										\\
		&\leq	\|S\| \sup_{\|x\|=1}{\|Tx\|} 									\\
		&=		\|S\| \|T\|.
	\end{align*}
	
	To see that $^\ast$ is an involution, use the fact that the adjoint 
	operator is unique for each operator and the following equalities.
	\begin{enumerate}[label=(\roman*)]
	\Item	\begin{align*}
				\langle (\alpha T+S)^\ast x,y\rangle 
			&=	\langle x, \alpha T+S y \rangle									\\
			&=	\overline{\alpha}\langle x,Ty\rangle + \langle x, Sy \rangle	\\
			&=	\overline{\alpha}\langle T^\ast x,y\rangle +
										 \langle S^\ast x, y \rangle			\\
			&=	\langle(\overline{\alpha} T^\ast + S^\ast) x, y \rangle.
			\end{align*}
	\Item 	\begin{align*}
				\langle (T^\ast)^\ast x,y\rangle 
			&=	\langle x, T^\ast y \rangle										\\
			&=	\overline{\langle T^\ast y,x \rangle}							\\
			&=	\overline{\langle y, Tx \rangle}								\\
			&=	\langle Tx,y\rangle.
			\end{align*}
	\Item	\begin{align*}
				\langle (ST)^\ast x,y \rangle
			&=	\langle x, STy \rangle											\\
			&=	\langle	S^\ast x,Ty \rangle										\\
			&=	\langle T^\ast S^\ast x,y \rangle.
		\end{align*}
	\end{enumerate}
	It remains to demonstrate the $C^\ast$-axiom on $\BH$. For all $x\in\H$, we have
	\begin{align*}
		\|Tx\|^2 = \langle Tx,Tx\rangle = \langle T^\ast Tx,x\rangle \leq \|T^\ast T\| \|x\|^2,
	\end{align*}
	so that
	\begin{align*}
		\|T\|^2 \leq \|T^\ast T\| \leq \|T^\ast\| \|T\| = \|T\|^2.
	\end{align*}
	It is clear that $I$ is a unit.
	Hence, the claim.
\end{proof} 




\todo [inline]{example of $\B(\H)$ at end of section to retro-motivate notation.
				discuss nomenclature (state etc) coming from QM}%%




\section{Representations of C*-algebras}
\todo[inline]{to include all representation theory, including GNS, CGN and GN}%%

	
\begin{thm}[Gelfand-Naimark, commutative] 
	Every commutative $C^\ast$-algebra $A$ is $\ast$-isomorphic to $\CX$,
	the algebra of continuous functions a compact Hausdorff space $X$.
\end{thm}
\begin{proof}
	Our compact topological space will be the set $\P{A}$ of pure states, endowed 
	with the weak$^\ast$ topology as defined above. 
\end{proof}

	\todo{remark about nonunital commutative.}%%
	

\begin{defn}
	Given a $C^\ast$-algebra $A$, a \emph{representation of $A$ on a Hilbert space $\H$} 
	is a $\ast$-homomorphism $\phi: A \to \BH$. 
	An isomorphic representation is called \emph{faithful}.
	If there exists an element $x\in\H$ such that the set $\{\phi(a) ~|~ a\in A\}$ is
	everywhere-dense in $\H$, say that $\phi$ is a \emph{cyclic} representation, 
	with \emph{cyclic vector} $x$.
\end{defn}	
	
\begin{thm} [Gelfand-Naimark-Segal construction]
	If $\rho$ is a state on a $C^\ast$-algebra $A$, then there exists a cyclic representation 
	$\pi_\rho$ of $A$ on a Hilbert space ${H}_\rho$, with unit cyclic vector $x_\rho$, such that 
	\[ 
		\rho(a)~=~ \langle \pi_\rho (a) x_\rho, x_\rho \rangle, ~~~~ \forall a \in A.
	\]
\end{thm}
\begin{proof}
	We will construct from $\rho$ the space $\Hr$,  representation $\pi_\rho$, 
	and vector $x_\rho$, and demonstrate the required properties.
	
	Consider the \emph{left kernel} of $\rho$:
	\[
		L_\rho := \{t \in A ~|~ \rho (t ^\ast t) = 0 \}.
	\]	
	For $a,b \in A$, define $\langle a , b \rangle_0 := \rho(b^\ast a)$.
	Then $L_\rho = \{t\in A ~|~ \langle t , t \rangle _0 = 0 \} $, and 
	$\langle \cdot, \cdot \rangle_0$ satisfies
	\begin{enumerate}[label=(\roman*)]
	  \item Linearity in 1st argument: for $a,b\in A$, $\alpha, \beta \in \mathbb{C}$:
		\begin{align*}
		   \langle \alpha a + \beta b, c \rangle_0 
		&= \rho (c^\ast(\alpha a + \beta b)   								\\
		&= \rho (\alpha c^\ast a + \beta c^\ast b)  						\\
		&= \alpha \rho (c^\ast a) + \beta \rho (c^\ast b)					\\
		&= \alpha \langle a , c \rangle_0 + \beta \langle b, c \rangle_0.
		\end{align*}
	  \item Conjugate symmetric: for $a,b \in A$:
	  	\begin{align*}
	  	   \langle b,a \rangle _0 
	  	&= \rho (a^\ast b)													\\
	  	&= \rho ((b^\ast a)^\ast)											\\
	  	&= \overline{\rho (b^\ast a)}										\\
	  	&= \overline{\langle a,b \rangle _0 }.
	  	\end{align*}
	  \item Positive semi-definite. \todo{why?} %%
	  	\begin{align*}
	  	\end{align*}
	\end{enumerate}
	Note that $\langle \cdot, \cdot \rangle$ is not necessarily positive definite on 
	$A$ -- $L_\rho$ is exactly where this fails.
	
	$L_\rho$ is \todo{sentence} a linear subspace of $A$: Consider  %%
	\[
		L:= \{t \in A  ~|~ \langle t,a \rangle _0 = 0, ~\forall a \in A \}\subseteq L_\rho.
	\]
	For $t \in L_\rho$, by Cauchy-Schwarz we have 
	\[ 
		|\langle t,a \rangle_0|^2 \leq \langle t,t \rangle_0 \langle a,a\rangle_0,~~ \forall a \in A;
	\]
	that is,
	\[
		\langle t,a \rangle _0 = 0, ~~~ \forall a \in A,
	\]
	so $t\in L$ and $L_\rho =L$.
	Now, for $a,b \in L$, $\alpha \in \mathbb{C}$ and $c \in A$:
	\[
		\langle \alpha a + b, c \rangle _0 = \alpha \langle a,c \rangle _0 + \langle b,c\rangle _0 = 0,
	\]
	so $\alpha a +b \in L$; also, $\langle 0,c\rangle _0 = 0$ so $0 \in L$. 
	Hence, $L ( =L_\rho ) $ is a linear subspace of $A$.
	
	For $s \in A$, $t\in L_\rho$, by the Cauchy Schwarz inequality [ref] we have 
	\begin{align*}
		|\rho (s^\ast t) |^2 
		&= 		|\langle t,s\rangle_0 |^2 									\\
		&\leq 	\langle t,t\rangle_0 \cdot \langle s,s\rangle_0  			\\
		&= 		\rho (t^\ast t) \cdot \rho (s^\ast s)						\\
		&=		0,
	\end{align*}
	so $\rho (s^\ast t) = 0$. Letting $s = a^\ast a t$ for $a \in A$, then
	\begin{align*}
		\rho ((at)^\ast at) 
		&= 		\rho (at^\ast a^\ast at) 									\\
		&= 		\rho ((a^\ast at)^\ast t) 									\\
		&= 		\rho (s^\ast t) 											\\
		&=		0,
	\end{align*}
	so that $at \in L_\rho$, for all $a \in A$ and $t \in L_\rho$; 
	we conclude that $L_\rho$ is a left ideal in $A$.
	$L_\rho$ is \todo{start sentence properly} the preimage in $A$ of $\{0\}$ under the continuous map%%
	$t \mapsto \rho (t^\ast t)$, so is closed.
	
	Consider now $V_\rho := A / L_\rho$, with $\langle \cdot,\cdot\rangle$ defined by 
	\[
		\langle a+L_\rho,b+L_\rho \rangle := \langle a,b\rangle_0, ~~~~ 
		\mbox{for}~~a+L_\rho,b+L_\rho \in V_\rho.
	\]
	It follows from properties $i),ii)$ and $iii)$ of $\langle \cdot,\cdot  \rangle _0$ that
	$\langle \cdot,\cdot  \rangle$ is an inner product on $V_\rho$ -- with
	\begin{align*}
				\langle a + L^\rho, a + L^\rho \rangle = 0
		&\iff 	\langle a,a\rangle=0										\\
		&\iff 	a \in L_\rho												\\
		&\iff 	a+L_\rho = 0+L_\rho
	\end{align*}
	giving positive definiteness.
	The completion of $V_\rho$ with respect to $\langle \cdot,\cdot  \rangle$ is a Hilbert space -
	this is the Hilbert space $\Hr$ we're looking for.
	
	Now we fix $a \in A$, and consider the map 
	\[
		\pi_a : V_\rho \to V_\rho; b+L_\rho \mapsto ab+L_\rho.
	\]
	Let $b_1, b_2 \in A$ be such that $b_1+L_\rho = b_2+L_\rho$. Then:
	\begin{align*}
		&\implies	b_1-b_2 \in L_\rho										\\	
		&\implies	a(b_1-b_2) \in L_\rho									\\	
		&\implies	ab_1-ab_2 \in L_\rho									\\	
		&\implies	ab_1 + L_\rho = ab_2+L_\rho								\\	
		&\implies	\pi_a(b_1+L_\rho) = \pi_a(b_2+L_\rho).
	\end{align*}
	Hence $\pi_a$ defines a linear operator on $V_\rho$.
	
	For $b+L_\rho \in V_\rho$:
	\begin{align*}
				\|a\|^2 \cdot \|b+L_\rho\| - \| \pi_a(b+L_\rho) \|
		&=		\|a\|^2 \cdot \|b+L_\rho\| - \| ab+L_\rho \|				\\
		&=		\|a\|^2 \cdot \langle b+L_\rho,b+L_\rho \rangle - 
									\langle ab+L_\rho,ab+L_\rho \rangle		\\
		&=		\|a\|^2 \cdot \rho(b^\ast b) - \rho ((ab)^\ast ab)			\\
		&= 		\rho (\|a\|^2 b^\ast b-b^\ast a^\ast ab)					\\
		&=		\rho (b^\ast (\|a\|^{2} \mathbbm{1} - a ^\ast a)b)			\\
		&\geq 	0.
	\end{align*}
	
	Thus $\pi_a$ is a bounded operator, with $\|\pi_a\| \leq \|a\|$. By continuity, \todo {cont of what?} %%
	$\pi_a$ extends to a bounded operator on $\Hr$ -- say $\pi_\rho(a):\Hr \to \Hr$ 
	such that \[ \pi_\rho(a)(v) = \pi_a(v) \] for $v\in \Vr$. 
	Then $\pi_\rho(a) \in \B(\Hr)$ for each $a \in A$, so $\pi_\rho$ defines a map 
	$A \to \B(\Hr)$ such that $ a \mapsto \pi_\rho(a)$. This will be our representation.
	
	Now, for $a,b \in A$, $c+ L_\rho \in \Vr$ and $\alpha \in \mathbb{C}$:
	\begin{align*}
				\pi_{\alpha a+b}(c+\Lr)
		&=		(\alpha a+b) (c+\Lr)										\\
		&=		(\alpha ac +\Lr) + (bc+\Lr)									\\
		&=		\alpha \pi_a (c+\Lr) + \pi_b(c+\Lr),
	\end{align*}
	so that $\pi_{\alpha a + b} = \alpha \pi_a +\pi_b$ on $\Vr$.
	
	For $a,b \in A$ and $c+ L_\rho \in \Vr$:
	\begin{align*}
		\pi_{ab}(c+\Lr)
		&=		abc+\Lr														\\
		&=		\pi_a (bc+\Lr)												\\
		&=		\pi_a (\pi_b (c+\Lr))										\\
		&=		(\pi_a \cdot \pi_b) (c+\Lr),
	\end{align*}
	so that $\pi_{a b} = \pi_a \cdot \pi_b$ on $\Vr$.
	
	
	For $a\in A$ and $b+ L_\rho,~c+ L_\rho \in \Vr$:
	\begin{align*}
				\langle b+\Lr, \pi_a^\ast (c+\Lr) \rangle 
		&=		\langle \pi_a (b+\Lr), c+\Lr \rangle						\\
		&=		\langle ab +Lr, c+\Lr \rangle								\\
		&=		\rho(c^\ast ab)												\\
		&=		\rho((a^\ast c)^\ast b)										\\
		&=		\langle b+\Lr, a^\ast c+\Lr \rangle							\\
		&=		\langle b+\Lr, \pi_{a^\ast}(c+\Lr),
	\end{align*}
	so that $ \pi_a^\ast = \pi_{a^\ast}$ on $\Vr$.
	
	$\Vr \subset \Hr$ is a dense subset, so the three properties above hold on 
	$\Hr$ by continuity. \todo {cont of what?} %%
	Hence, $\pi_\rho: A \to \B(\Hr)$ is a representation of $A$.
	As to the unit vector, consider $\xr := \1 + \Lr \in \Vr$. Then for $a \in A$,
	\begin{align*}
				\langle \pi_\rho(a)\xr ,\xr \rangle 
		&=		\langle \pi_a(\1+\Lr), \1+\Lr \rangle						\\
		&=		\langle a+\Lr \1+\Lr \rangle								\\
		&=		\rho(a);
	\end{align*}
	in particular, $\langle \xr,\xr \rangle = \rho (\1) = 1$, so $\xr$ is a unit vector in $\Hr$.	
\end{proof}
\todo[inline]{example of this construction on C(X)? may just be a short explanation of how B(H) and 
C(X) link together. can then talk about noncommutative topology!} %%

\begin{thm}[Gelfand-Naimark]
	Every $C^\ast$-algebra has a faithful representation.
\end{thm}
\begin{proof}
	for this we just take the direct sum representation of the representations given from GNS
	by some set of states containing all pure states.
\end{proof}
\todo[inline]{further topics: K-theory, group C* algebras, amenable algebras, von neumann algebras,}
\todo[inline]{references!!!!}


\begin{thebibliography}{00}
\bibitem{blackadar06}
	Blackadar, B.,
	\emph{Operator Algebras: Theory of C*-Algebras and von Neumann Algebras.}
	Encyclopaedia of Mathematical Sciences, 122. Operator Algebras and Non-commutative Geometry, III. Springer-Verlag, Berlin (2006).
	
\bibitem{bornjordan25}
	Born, M. \& Jordan, P.,
	\emph{Zur Quantenmechanik.}
	Z. Physik (1925) 34: 858.
	
\bibitem{bornjordanheisenberg25}
	Born, M.; Heisenberg, W. \& Jordan, P.,
	\emph{Zur Quantenmechanik. II.}
	Z. Physik (1926) 35: 557.

\bibitem{dixmier77}
	Dixmier, J.,
	\emph{C*-algebras.}
	Holland Mathematical Library, Vol. 15. North-Holland Publishing Co., Amsterdam -- New York -- Oxford (1977).

\bibitem{dixmier81}
	Dixmier, J.,
	\emph{von Neumann algebras.}
	North-Holland Publishing Co., Amsterdam -- New York (1981).
		
\bibitem{elliott94}
	Elliott, G. A.,
	\emph{The classification problem for amenable C*-algebras.}
	Proceedings of the International Congress of Mathematicians, Vol. 1, 2 (Z\"{u}rich, 1994), pp.~922-–932, Birkhäuser, Basel (1995).

\bibitem{fell61}
	Fell, J. M. G.,
	\emph{The structure of algebras of operator fields. }
	Acta Math. 106, 1961, 233–280. 


\bibitem{gardella15}
	Gardella, E. E.,
	\emph{Compact group actions on C*-algebras: classification, non-classifiability, and crossed products and rigidity results for L$^p$-operator algebras. }
	Thesis (Ph.D.) -– University of Oregon (2015).

\bibitem{gelfand43}
	Gelfand, I. \& Neumark, M.,
	\emph{On the imbedding of normed rings into the ring of operators in Hilbert space.}
	Rec. Math. [Mat. Sbornik] N.S. 12(54) (1943), pp.~197–-213.

\bibitem{heisenberg25}
	Heisenberg, W.,
	\emph{{\"U}ber quantentheoretische Umdeutung kinematischer und mechanischer Beziehungen.}
	Z. Physik (1925) 33: 879. 
	
\bibitem{kadison83}
	Kadison, R. V. \& Ringrose, J. R.,
	\emph{Fundamentals of the theory of operator algebras: Vol. I. Elementary theory.}
	Pure and Applied Mathematics, 100. Academic Press, Inc. [Harcourt Brace Jovanovich, Publishers], New York (1983).

\bibitem{kadison86}	
	Kadison, R. V. \& Ringrose, J. R.,
	\emph{Fundamentals of the theory of operator algebras: Vol. II. Advanced theory.}
	Pure and Applied Mathematics, 100. Academic Press, Inc., Orlando, FL (1986).
	
\bibitem{lin01}
	Lin, H.,
	\emph{An introduction to the classification of amenable C*-algebras.}
	World Scientific Publishing Co., Inc., River Edge, NJ (2001).
\bibitem{lin08}
	Lin, H. \& Niu, Z.,
	\emph{Lifting KK-elements, asymptotic unitary equivalence and classification of simple C*-algebras.}
	Adv. Math. 219 (2008), no. 5, pp.~1729-–1769. 
	
\bibitem{lin11}
	Lin, H.,
	\emph{Asymptotic unitary equivalence and classification of simple amenable C*-algebras.}
	Invent. Math. 183 (2011), no. 2, pp.~385-–450. 

\bibitem{mackinnon77}
	MacKinnon, E.,
	\emph{Heisenberg, Models, and the Rise of Matrix Mechanics.}
	Hist. Stud. Phys. Sci., Vol. 8 (1977), pp.~137--188
	
\bibitem{vonneumann35}
	Murray, F. J. \& von Neumann, J.,
	\emph{On rings of operators.}
	Ann. of Math. (2) 37 (1936), no. 1, pp.~116–-229.

\bibitem{vonneumann37}
	Murray, F. J. \& von Neumann, J.,
	\emph{On rings of operators. II.}
	Trans. Amer. Math. Soc. 41 (1937), no. 2, pp.~208-–248. 

\bibitem{vonneumann43}
	Murray, F. J. \& von Neumann, J.,
	\emph{On rings of operators. IV.}
	Ann. of Math. (2) 44, (1943), pp.~716-–808.
	
\bibitem{niemiec12}
	Niemiec, P.,
	\emph{Elementary Approach to Homogeneous C*-algebras.}
	Rocky Mountain J. Math. 45 (2015), no. 5, pp~1591--1630.	
	
\bibitem{pedersen79}
	Pedersen, G. K.,
	\emph{C*-algebras and their automorphism groups.}
	Academic Press, Inc. [Harcourt Brace Jovanovich, Publishers], London-New York (1979).

\bibitem{sakai71}
	Sakai, S.,
	\emph{C*-algebras and W*-algebras.}
	Ergebnisse der Mathematik und ihrer Grenzgebiete, Band 60. Springer-Verlag, New York-Heidelberg (1971).

\bibitem{schroer03}
	Schroer, B.,
	\emph{Pascual Jordan, Glory and Demise and his legacy in contemporary local quantum physics.}
	Unpublished manuscript (2003).
	
\bibitem{simmons83}
	Simmons, G.,
	\emph{Introduction to Topology and Modern Analysis.}
	Robert E. Krieger Publishing Co., Inc., Melbourne, Fl. (1983).

\bibitem{vonneumann40}
	von Neumann, J.,
	\emph{On rings of operators. III.}
	Ann. of Math. (2) 41, (1940), pp.~94–-161.

	
\end{thebibliography}

\end{document}