\documentclass[12pt,a4paper]{report}
\usepackage{graphicx}
\usepackage{enumitem}
\usepackage{bbm}	%% makes \1 look good
\usepackage{hyperref}
\usepackage{fullpage}
\usepackage{mathrsfs}
\usepackage{amsmath}
\usepackage{amsthm}
\usepackage{amssymb}
\usepackage{amsfonts}
\usepackage{csquotes}




\theoremstyle{plain}
\newtheorem{thm}{Theorem}
\newtheorem*{thm*}{Theorem}
\newtheorem{lemma}{Lemma}
\newtheorem*{lemma*}{Lemma}
\newtheorem*{claim}{Claim}
\newtheorem{prop}{Proposition}
\newtheorem*{prop*}{Proposition}
\theoremstyle{definition}
\newtheorem{defn}{Definition}
\newtheorem*{defn*}{Definition}

\newcommand{\1}{\mathbbm{1}}
\newcommand{\C}{\mathbb{C}}
\newcommand{\R}{\mathbb{R}}
\newcommand{\N}{\mathbb{N}}
\renewcommand{\H}{\mathcal{H}}
\newcommand{\M}{\mathcal{M}}
\newcommand{\B}{\mathcal{B}}
\newcommand{\K}{\mathcal{K}}
\newcommand{\T}{\mathcal{T}}
\newcommand{\cover}{\mathcal{C}}
\newcommand{\BH}{\mathcal{\B(\H)}}
\newcommand{\Hr}{\mathcal{H}_\rho}
\newcommand{\Vr}{V_\rho}
\newcommand{\Lr}{L_\rho}
\newcommand{\xr}{x_\rho}
\renewcommand{\phi}{\varphi}
\newcommand{\CX}{C(X)}
\renewcommand{\S}{\mathscr{S}}
\renewcommand{\P}[1]{\mathscr{P}(#1)}
\newcommand{\spec}[1]{\sigma(#1)}
\renewcommand{\bar}{\overline}
\renewcommand{\labelenumi}{(\roman{enumi})} %% makes top layer enum numbers roman
\newcommand\Item[1][]{ 				%% enumerate magic
  \ifx\relax#1\relax  \item \else \item[#1] \fi
  \abovedisplayskip=0pt\abovedisplayshortskip=0pt~\vspace*{-\baselineskip}}




\begin{document}
\begin{titlepage}
	{\centering
	{\includegraphics[width=0.5\textwidth]{UOY-Logo-Stacked-shield-PMS432.png}\par 
		%% if this won't compile, comment out this line. i don't have permission to upload the 
		%%		university logo with this, sozzle.
		}
	\vspace{0.5cm}
	{\huge\bfseries $C^\ast$-Algebras, and the Gelfand-Naimark Theorems\par}
	\vspace{1.5cm}
	{\Large\itshape Luke Armitage\par}
	\vspace{0.5cm}
	{\large Supervised by Dr. Eli Hawkins\par}
	\vfill
	{\scshape\small Abstract\par}}
	{\small The collection of bounded linear operators from a Hilbert space to itself 
	forms an algebra, complete with respect to the operator norm and posessing an 
	involution map taking an operator to its adjoint operator. These operations 
	interact in a very strong way, leading to a linking of the algebraic and 
	topological structures of such an operator algebra. These operator algebras are 
	the prototype for a $C^\ast$-algebra, which we define as a complete normed 
	algebra with an involution such that the involution and norm interact in the 
	same way as in an operator algebra. In this report, we define an abstract 
	$C^\ast$-algebra and discuss the standard examples -- operators on a Hilbert 
	space and continuous functions on a compact Hausdorff space. We discuss some of 
	the algebraic and topological structure of these spaces and go on to prove the 
	Gelfand-Naimark theorems. The Gelfand-Naimark theorem says that all 
	$C^\ast$-algebras are the same as an operator algebra on some Hilbert space, 
	and the commutative Gelfand-Naimark theorem states that all commutative 
	$C^\ast$-algebras are the same as an algebra of continuous functions on some 
	compact Hausdorff space.}
	\par
	\vspace{1cm}
	{\large\centering \today\par}
\end{titlepage}

\tableofcontents


\chapter{Introduction}
\section{History of The Study of Operator Algebras}
The study of $C^\ast$-algebras started with the study of rings of operators acting on a Hilbert space, 
which was introduced in the 1930s as a framework for John von Neumann's 1932 formulation of quantum 
mechanics in \cite{vonneumann32}. These rings of operators are now considered part of the theory of 
\emph{von Neumann algebras}, a subsection of $C^\ast$-algebra theory. 


{In 1943, Gelfand and Naimark \cite{gelfand43} established an abstract 
characterisation of $C^\ast$-algebras, free from dependence on the operators 
acting on a Hilbert space. The Gelfand-Naimark theorem, which we will be 
considering here at length, gives the link between these abstract 
$C^\ast$-algebras and the rings of operators previously studied. Used in the 
proof of the GN theorem is the Gelfand-Naimark-Segal construction, a technique 
which, given a certain linear function on a $C^\ast$-algebra, yields a 
representation of that $C^\ast$-algebra as an algebra of operators on a Hilbert 
space.} 

\section{Resources}
Throughout this work we will be making extensive use of (the first volume of) \emph{`Fundamentals of the 
Theory of Operator Algebras'} \cite{kadison83,kadison86}, by Kadison and Ringrose; the authors aim to 
make accessible the ``vast recent research literature'', by first introducing and exploring the 
prototypical examples and developing the abstract theory later. 
	
Some other texts which cover $C^\ast$-algebras: 
\begin{itemize}
	\item 	Dixmier \cite{dixmier77} presents a summary of the general theory up to that time (1977), 
	with \cite{dixmier81} focussing on reworking and developing the theory of von Neumann algebras; 
	\item 	Sakai \cite{sakai71} gives a treatment of $C^\ast$- and von Neumann algebras from a more 
	topological point of view; 
	\item	Davidson \cite{davidson96} gives a quick overview of the theory and uses that as a baseline 
	to explore many classes of examples;
	\item 	Blackadar \cite{blackadar06} gives a concise, encyclopaedic coverage of the theory of 
	operator algebras, and covers more specialised material and applications. 
\end{itemize}
	
\section{An Overview}

After this introductory chapter, we give some definitions and results which will 
be used throughout, and establish notation. Then the definition of a 
$C^\ast$-algebra is given, with some discussion of algebras with a 
multiplicative identity, followed by an exploration of an example with some 
previews of concepts later defined. 

We move on to discussing a powerful tool in 
this area, and indeed all of functional analysis, which is the concept of the 
spectrum of an element; then we make some more definitions and discuss how the 
topological and algebraic structures of $C^\ast$-algebras are related. 
We finish that chapter with another example, with a remark on what makes the 
study of abstract $C^\ast$-algebras applicable to quantum mechanics. 

The final chapter 
contains the statements and proofs of the Gelfand-Naimark theorems, and a 
discussion of how they relate to each other. The Gelfand-Naimark theorem for 
commutative $C^\ast$-algebras, discussed in Section \ref{section:abelian}, says 
that an Abelian $C^\ast$-algebra is in some way `the same as' the algebra of 
continuous complex valued functions on a compact, Hausdorff topological space. 
The Gelfand-Naimark theorem for general $C^\ast$-algebras, discussed in Section 
\ref{section:gn}, says that any $C^\ast$-algebra `the same as' the algebra of 
bounded operators on a Hilbert space.




\section{Further topics}
Topics in $C^\ast$-algebra theory which follow on from this include the classification of different 
types of $C^\ast$-algebras, such as \emph{approximately-finite dimensional} $C^\ast$-algebras, which are 
made up in a certain way from finite dimensional $C^\ast$-algebras.

\emph{A von Neumann algebra }is a $C^\ast$-subalgebra of $\BH$ which is unital 
and closed in the \emph{weak operator topology}, the weakest topology such that 
the map $T\mapsto\langle Tx,y\rangle$ is continuous for all $x,y$ in $\H$. This 
enforces an even stricter structure on von Neumann algebras, giving results such 
as the von Neumann Double Commutant theorem.


Another area of research is $K$-theory, which for $C^\ast$-algebras refers to the study of groups 
related to the structure of projective elements. 
For more on all of these topics, see \cite{davidson96,dixmier81,sakai71}.


%% acknowledgements


\chapter{Background Mathematics and Resources} %%needs a better name  -- '... and results'?
The following is some mathematics which may prove useful throughout the project, with relevant 
resources; we will of course be making definitions as needed.
	
We will be assuming some familiarity with the following theory:
\begin{itemize}
	\item Rings, algebras and linear spaces.
	\item Normed spaces, inner product spaces, Banach and Hilbert spaces.
	\item Point-set topology.
\end{itemize}
A good broad background on all of these can be found in \cite{rudin91,simmons83}.


	
	
The following notation applies throughout:
$\mathbb{K}$ denotes a field. $\R^+$ denotes the non-negative reals. 
$\mbox{Re }z$ refers to the real part of the complex number $z$.
An ideal of an algebra is, unless otherwise stated, taken to mean a two-sided ideal. 

\section{Linear spaces}

A \emph{partial order} on a set $X$ is a relation $\leq$ which satisfies the conditions
\begin{itemize}
	\item $x \leq x$ for all $x$ in $X$; 		\hfill (reflexivity)
	\item for all $x$ and $y$ in $X$, if $x \leq y$ and $y\leq x$, then $x=y$; 
												\hfill (antisymmetry)
	\item for all $x,y$ and $z$ in $X$, if $x \leq y$ and $y\leq z$, then $x\leq z$. 
												\hfill (transitivity)
\end{itemize}

A \emph{positive cone} in a real vector space $V$ is a subset $V^+$ that is closed under addition and scaling by positive scalars, and if $v,-v$ are in $V^+$, then $v=0$. A \emph{partially ordered vector space} is a real vector space $V$ with a positive cone $V^+$ -- the partial order on $V$ given by
\begin{align*}
	x\leq y ~~\iff~~ y-x \in V^+,
\end{align*}
for $x,y$ in $V$.
A familiar example is the non-negative real numbers, which form a positive cone in $\R$.


Given a linear space $X$ over a field $\mathbb K$, a \emph{linear functional} on $X$ is a linear map 
$\rho:X\to\mathbb K$. The set $X^\ast$ of linear functionals on $X$ is itself a linear space, called
the \emph{dual space} of $X$. A linear functional $\rho$ is called \emph{multiplicative} if $\rho(xy) = \rho(x)\rho(y)$ for all $x,y$ in $X$.

A normed linear space $(V,\|\cdot\|)$ is a \emph{Banach space} if it is complete with respect to 
$\|\cdot\|$, in the sense that all Cauchy sequences in $V$ converge with respect to the norm. An inner 
product space $(V,\langle\cdot,\cdot\rangle)$ is a \emph{Hilbert space} if it is a Banach space with 
respect to the norm induced by $\langle\cdot,\cdot\rangle$.

\begin{thm*}[Hahn-Banach Extension theorem] 
	If $\rho_0$ is a bounded linear functional on a subspace $X_0$ of a normed linear space $X$,
	then there is a bounded linear functional $\rho$ on $X$ such that $\|\rho\|=\|\rho_0\|$ and 
	$\rho=\rho_0$ on $X_0$.
\end{thm*}
\begin{proof}
	Can be found in \cite[Theorem 1.6.1, p.~44]{kadison83}
\end{proof}

\section{Constructions on Hilbert spaces}\label{section:hscons}
We need some Hilbert spaces constructions; particularly, the direct sum of a collection of Hilbert 
spaces and the direct sum of bounded operators on these Hilbert spaces.
Given a finite collection $\{\H_1,\dots,\H_n\}$ of Hilbert spaces, let~$\H$ denote the set
\[
	\H  = \{(x_1,\dots,x_n) ~|~ x_i \in \H_i \mbox { for } i=1,\dots,n\}.
\]
Define addition and scalar multiplication coordinatewise, and given $x=(x_1,\dots,x_n)$ and 
$y=(y_1,\dots,y_n) \in \H$, the equation
\[
	\langle x,y\rangle = \langle x_1,y_1\rangle + \dots + \langle x_n,y_n \rangle
\]
defines an inner product $\langle\cdot,\cdot\rangle$ on $\H$. The resulting norm $\|\cdot\|$ is 
given by
\[
	\|x\|^2 = \|x_1\|^2 + \dots + \|x_n\|^2.
\]
It is easy to show that $\H$ is a Hilbert space with these operations, and we call it the
\emph{direct sum} of the collection $\{\H_1,\dots,\H_n\}$, denoted $\oplus\H_i$.

Similarly, we can construct a Hilbert space direct sum of an infinite collection 
$\{\H_i ~|~ i\in I\}$ of Hilbert spaces. Let $\H$ be the set
\[
	\H= \{(x_i) ~|~ x_i \in \H_i \mbox{ for each }i 
									\mbox{ and } \sum_{i\in I}{\|x_i\|^2} < \infty \}.
\]
Given $x=(x_i)$ and $y=(y_i)\in\H$, we have that
\begin{align*}
			\Big(\sum_{i\in I}{\|x_i+y_i\|^2}\Big)^{1/2}
	&\leq	\Big(\sum_{i\in I}{(\|x_i\|+\|y_i\|)^2\|}\Big)^{1/2}			\\
	&\leq	\Big(\sum_{i\in I}{\|x_i\|^2}\Big)^{1/2}		+
							\Big(\sum_{i\in I}{\|y_i\|^2}\Big)^{1/2}		\\
	&< \infty.
\end{align*}
Hence, the sequence $(x_i+y_i)$ is in $\H$, and we can define addition and scalar multiplication 
coordinatewise on $\H$:
\begin{align*}
	(x_i)+(y_i)=(x_i+y_i) && \alpha(x_i) = (\alpha x_i).
\end{align*}
We also have
\begin{align*}
			\sum_{i\in I}{|\langle x,y\rangle|} 
	&\leq	\sum_{i\in I}{\|x_i\|\|y_i\|}									\\
	&\leq 	\Big(\sum_{i\in I}{\|x_i\|^2}\Big)^{1/2}
							\Big(\sum_{i\in I}{\|y_i\|^2}\Big)^{1/2}		\\
	&< \infty,
\end{align*}
so that we can define an inner product $\langle\cdot,\cdot\rangle$, with induced norm $\|\cdot\|$, 
on $\H$ by
\begin{align*}
	\langle x,y\rangle = \sum_{i\in I}{|\langle x,y\rangle|}, 
					&& \|x\| = \Big(\sum_{i\in I}{\|x_i\|^2}\Big)^{1/2}.
\end{align*}
To see that $\H$ is complete with respect to $\|\cdot\|$, suppose that $(x^{n})_{n\in\N}$ is a 
Cauchy sequence in $\H$, where $x^{n} = (x^n_i)_{i\in I}$ for each $n$. Then given any positive 
$\epsilon$, there exists a positive integer $N$ such that
\[
	\|x^m-x^n\| < \epsilon \mbox{ for all } m,n\geq N,
\]
that is,
\begin{align}\label{eqn:sum1}
	\sum_{i\in I}{\|x^m_i-x^n_i\|^2} < \epsilon^2 \mbox{ for all } m,n\geq N.
\end{align}
Hence for each $i\in I$,
\[
	\|x^m_i-x^n_i\| < \epsilon \mbox{ for all } m,n\geq N
\]
so that $(x^n_i)_{n\in\N}$ is a Cauchy sequence in $\H_i$, having a limit $x_i\in\H_i$.
For any finite subset $J\subset I$, it follows from \eqref{eqn:sum1} that 
\[
	\sum_{j\in J}{\|x^m_j-x^n_j\|^2} < \epsilon^2  \mbox{ for all } m,n\geq N,
\]
and letting $m$ tend to infinity,
\begin{align}\label{eqn:sum2}
	\sum_{j\in J}{\|x_j-x^n_j\|^2} < \epsilon^2  \mbox{ for all } n\geq N.
\end{align}
This holds for any finite subset $J$, so
\begin{align*}
	\sum_{i\in I}{\|x_i-x^n_i\|^2} < \epsilon^2  \mbox{ for all } n\geq N,
\end{align*}
and so $(x_i-x^n_i)$ and $x^n_i$ are in $\H$ for $n\geq N$. 
Then $(x_i)$ is in $\H$ and by \eqref{eqn:sum2}, $x^n$ converges to $(x_i)$ as $n$ tends 
to infinity. We conclude that $\H$ is complete and therefore a Hilbert space. Just like in the finite 
case, we call $\H$ the \emph{direct sum} of the collection $\{\H_i ~|~ i\in I\}$ of Hilbert spaces, 
denoted $\oplus\H_i$.

Suppose now we have a (finite or infinite) collection of bounded operators 
$\{T_i\in\B{(\H_i)} ~|~ i\in I\}$ such that 
\[
	\sup_{i\in I} {\|T_i\|} < \infty
\] 
(by convention this is true when $I$ is a finite set).
For $x=(x_i)$ in $\oplus\H_i$, define an element $Tx$ in $\oplus\H_i$ by $Tx=(T_ix_i)$. 
Then $T:\H\to\H:x\mapsto Tx$ is a bounded linear operator, called the \emph{direct sum} of the 
collection $\{T_i\in\B{(\H_i)} ~|~ i \in I\}$, denoted $\oplus T_i$. For $S_i,T_i$ in 
$\B(\H_i)$, $\alpha,\beta$ in $\C$, we have 
\begin{align*}
		\left(\oplus T_i\right)^\ast &= \oplus T_i^\ast,		\\
		\oplus (\alpha S_i+\beta T_i) &= 
					\alpha \oplus S_i + \beta \oplus T_i, 		\\
		\oplus (S_i T_i) &= \oplus S_i \oplus T_i,				\\
		\|\oplus T_i\| &= \sup_{i\in I} {\|T_i\|}.				\\
\end{align*}

\section{Topological stuff}
By a \emph{topological space} we mean a set $X$ endowed with a collection $\T$ (a \emph{topology}) of 
subsets of $X$ such that $\T$ is closed under finite intersections and arbitrary unions, and contains 
both $X$ and the empty set. A \emph{neighbourhood} of a point $x$ is a subset of $X$ which contains an 
open set which itself contains $x$. An \emph{open cover} of $X$ is a collection $\cover$ of open sets 
whose union contains $X$, and a subset of $\cover$ which still covers $X$ is called a \emph{subcover}. 
A topological space $X$ is said to be:
\begin{itemize}
	\item \emph{Hausdorff} if, for every distinct pair of points $x,y$ of $X$, there is a neighbourhood 
	of $x$ which is disjoint from some neighbourhood of $y$;
	\item \emph{compact} if every open cover of $X$ has a finite subcover;
	\item \emph{locally compact} if every point of $x$ has a compact neighbourhood.
\end{itemize}
Most of the topology we use will be compact and Hausdorff, and locally compact and Hausdorff at the 
least.
A subset $Y$ of $X$ is said to be \emph{dense} in $X$ every non-empty open set in $X$ contains an 
element of $A$ -- equivalently, if $X$ is the largest closed set containing $Y$.
The topology \emph{generated by a collection $\B$} is the collection $\T$ containing $\B$ and all finite 
intersections and arbitrary unions over $\B$. It is the weakest topology on $X$ containing $\B$, in the 
sense that any topology containing $\B$ contains all of $\T$.

A \emph{topological vector space} is a vector space $V$ over $\mathbb K$, together with a topology on 
$V$ such that the vector space operations $V\times V\to V:(x,y)\mapsto x+y$ and $\mathbb{K}\times V\to 
V:(\lambda,x)\mapsto \lambda x$ are continuous. 
Recall that an \emph{extreme point} of a convex subset $X_0$ of a topological vector space is a point 
$x$ for which an expression
\begin{align*}
	x = \alpha x_1 +(1-\alpha)x_2,
\end{align*}
for $0\leq\alpha\leq1$ and $x_1,x_2\in X_0$, implies that $x_1=x=x_2$. 
The convex set $X_0$ is then equal to the set of all linear combinations
\begin{align*}
	\alpha_1 x_1+\dots +\alpha_n x_n
\end{align*}
of its extreme points, where $\alpha_1,\dots,\alpha_n$ are positive scalars summing to 1.
For example, for a polygon embedded in $\R^2$, the vertices of a polygon are its extreme points, and
every point within the polygon can be written as a linear combination of the vertices.

\begin{lemma}[{\cite[1.4.4]{kadison83}}]\label{lemma:144}
	If $X$ is a non-empty, compact, convex subset of a locally convex space $V$, and $\rho$ is a 
	continuous linear functional on $V$, then there is an extreme point $x_0$ of $X$ such that, for 
	every $x$ in $X$,
	\begin{align*}
		\mbox{Re }\rho(x) \leq \mbox{Re }\rho(x_0).
	\end{align*}
\end{lemma}

\begin{defn*}[{\cite[3.14]{rudin91}}]
	Let $X$ be a topological vector space (over a field $\mathbb K$), with dual space $X^\ast$. Every 
	$x$ in $X$ induces a linear functional $f_x$ on $X^\ast$ defined by $f_x(\rho)=\rho(x)$. The 
	\emph{weak$^\ast$ topology} on $X^\ast$ is the topology generated by the sets 
	\begin{align*}
		\{f^{-1}_x(V) ~|~ x \in X, V\subseteq \mathbb K \mbox{ open}\}.
	\end{align*}
	This is the weakest topology on $X$ such that each functional $f_x$ is continuous, and the 
	collection $\{f_x ~|~ x\in X\}$ separate the points of $X^\ast$, so the weak$^\ast$ topology is 
	Hausdorff.
\end{defn*}

\begin{thm*}[{The Banach-Alaoglu theorem \cite{rudin91}}]
	If $V$ is a neighbourhood of $0$ in a topological vector space $X$, and
	\begin{align*}
		K=\{\rho\in X^\ast ~|~ |\rho(x)|\geq 1 \mbox{ for all } x \in V \},
	\end{align*}
	then $K$ is compact in the weak$^\ast$ topology.
\end{thm*}

\chapter{$C^\ast$-algebras}
We begin this chapter with some definitions and an important example, then give some results we 
will need later and finish with another fundamental example.
\section{Definition}
\begin{defn}
	A \emph{Banach algebra} is a complex Banach space $A$, with norm $\|\cdot\|$, which forms an 
	algebra, such that 
	\begin{align*}
		\|ab\| \leq \|a\| \|b\| \mbox{ for all } a,b \in A.
	\end{align*}
	A \emph{$\ast$-algebra} is an algebra $A$ with an \emph{involution} map 
	$a \mapsto a^\ast$ on $A$ such that, for all $a,b \in A$ and for $\alpha \in \C$,
	
	\begin{enumerate}
		\item $a^{\ast\ast} = (a^\ast)^\ast = a$,
		\item $(\alpha a+b)^\ast = \overline{\alpha} a^\ast + b^\ast$,
		\item $(ab)^\ast = b^\ast a^\ast$.
	\end{enumerate}
	The element $a^\ast$ is referred to as the \emph{adjoint} of $a$. 		\\
	A \emph{$C^\ast$-algebra} is a Banach algebra $A$ with involution $a \mapsto a^\ast$ making it a 
	$\ast$-algebra, with the condition that
	\begin{align*}
		\|a ^\ast a\| = \|a\|^2 \mbox{ for all } a \in A.
	\end{align*}
	This condition is known as the \emph{$C^\ast$ axiom}.\\
	A subalgebra $B$ of a $C^\ast$-algebra $A$ is a \emph{$C^\ast$-subalgebra} if it is closed under the 
	adjoint and complete with respect to the norm; equivalently, if $B$ is itself a $C^\ast$-algebra.\\
	An element $a\in A$ of a $C^\ast$-algebra is called
	\begin{itemize}
		\item	\emph{normal} if $a^\ast a=aa^\ast$;
		\item 	\emph{self-adjoint} if $a^\ast=a$;
		\item 	\emph{unitary} if $a^\ast a = aa^\ast =\1$. 
	\end{itemize}
\end{defn}
The set of self-adjoint elements $A_{sa}$ forms a vector space over $\R$.
Note that an element being self-adjoint, or unitary, implies that it is normal.

As a simple example, we can consider $\C$ as a $C^\ast$-algebra: We know from complex analysis that 
Cauchy sequences converge in $\C$ and that complex conjugation is an involutive operation. In this way 
we can interpret self-adjoint elements of a $C^\ast$-algebra $A$ as `reals' -- in fact, for $a$ in $A$, 
let $h=\tfrac{1}{2}(a+a^\ast)$ and $k=\tfrac{i}{2}(a-a^\ast)$. Then $h$ and $k$ are self-adjoint and we 
can write $a=h+ik$ -- call $h$ and $k$ the \emph{real} and \emph{imaginary} part of $a$, respectively. 


\section{Unitization}
If a $C^\ast$-algebra $A$ contains a multiplicative identity element $\1$, such that 
$a \1 = a = \1  a$ for all $a \in A$, call $\1$ the \emph{unit} 
in $A$, and $A$ is then a \emph{unital} $C^\ast$-algebra.

\begin{prop*}[{\cite[I.1.3]{davidson96}}]
	Any non-unital $C^\ast$-algebra $A$ can be isometrically embedded in a unital $C^\ast$-algebra 
	$\tilde{A}$.
\end{prop*}
\begin{proof}
	Let $\tilde{A} = A \oplus \C$ with pointwise addition, and define
	\begin{align*}
		(a,\lambda) (b,\mu) &= (ab+\lambda b + \mu a, \lambda \mu),				\\
		(a,\lambda)^\ast &= (a^\ast,\overline{\lambda}).
	\end{align*}
	Then $\tilde{A}$ is a $\ast$-algebra. 
	Consider $(a,\lambda)$ as an operator acting on $A$ via $b\mapsto ab+\lambda b$ for $b$ in $A$. It 
	is easy to see that this operator is linear, and 
	\begin{align*}
				\sup_{\|b\|=1}\|ab+\lambda b\| 
		&\leq 	\sup_{\|b\|=1} (\|ab\|+ \lambda\|b\|)							\\
		&=		\sup_{\|b\|=1} \|ab\|+ \lambda < \infty
	\end{align*}
	so that the operator $(a,\lambda)$ is bounded. It follows that
	\begin{align*}
		\|(a,\lambda)\| = \sup_{\|b\|=1}\|ab+\lambda b\|
	\end{align*}
	is a norm on $\tilde{A}$, and so $\tilde{A}$ is a Banach $\ast$-algebra, with unit $(0,1)$.
	The embedding $a\mapsto(a,\lambda)$ is isometric because 
	\[
		\|a\| = \|a\cdot\frac{a}{\|a\|}\| \leq \|(a,0)\| \leq \sup_{\|b\|=1}{\|ab\|} \leq \|a\|.
	\]
	It remains to verify the $C^\ast$-axiom:
	\begin{align*}
				\|(a,\lambda)\|^2 
		&=		\sup_{\|b\|=1}{\|ab+\lambda b\|^2}								\\
		&=		\sup_{\|b\|=1}{\|b^\ast a^\ast ab 
								+\lambda b^\ast a^\ast b
								+\overline{\lambda}b^\ast a b
								+|\lambda|^2 b^\ast b}							\\
		&\leq	\sup_{\|b\|=1}{\|a^\ast ab 
								+\lambda a^\ast b
								+\overline{\lambda}a b
								+|\lambda|^2 b\|}								\\
		&=		\|(a^\ast a + \lambda a^\ast +\overline{\lambda}a,|\lambda|^2)	\\
		&= 		\|(a,\lambda)^\ast(a,\lambda)\|									\\
		&\leq	\|(a,\lambda)^\ast\| \|(a,\lambda)\|.
	\end{align*}
	By symmetry of $\ast$, $\|(a,\lambda)^\ast\| = \|(a,\lambda)\|$. 
	Hence, the above inequality becomes equality and we have that
	\begin{align*}
		\|(a,\lambda)^\ast(a,\lambda)\| = \|(a,\lambda)\|^2,
	\end{align*}
	and $\tilde A$ is a $C^\ast$-algebra.
\end{proof}

In light of this result, we take all $C^\ast$-algebras from here to be unital unless
specified otherwise. For the results we will consider, we can simply consider the unital case. 
However, there are circumstances in advanced theory in which one needs to relax the unital 
condition, where properties do not pass from a $C^\ast$-algebra to its maximal ideals, so it is by no 
means a universal assumption that a $C^\ast$-algebra has a unit element.


\section{$\CX$ -- an example} \label{section:CX}
Given a compact Hausdorff space $X$, let $\CX$ be the algebra of continuous functions
$f:X\to\C$, with addition and multiplication defined pointwise. Define $\|\cdot\|$ on $\CX$ by
\[
	\|f\|= \sup_{x\in X}{|f(x)|},
\]
and let $\overline f$ be the function $\overline f(x) = \overline{f(x)}$ for $x$ in $X$.
We will demonstrate that $\CX$ is an Abelian $C^\ast$-algebra and preview some of the theory
we will build up in later sections. Later, in Section \ref{section:abelian}, we will show that $\CX$ is 
essentially \emph{the} Abelian $C^\ast$-algebra.

\begin{claim}
	The set $\CX$, of continuous complex valued functions on a compact Hausdorff space $X$, is an 
	Abelian $C^\ast$-algebra.
\end{claim}
\begin{proof}
	We take $\CX$ as an algebra with operations defined pointwise and norm defined as above.
	Multiplication in $\CX$ is pointwise, so analogous to multiplication in $\C$, hence commutative.
	Showing that $\CX$ is a Banach space with this norm is a simple exercise. We will demonstrate that $
	\CX$ is a Banach algebra, that $f\mapsto\overline{f}$ is an involution, and that the norm satisfies 
	the $C^\ast$ axiom.
	Given $f$ and $g$ in $\CX$, we have 
	\begin{align*}
				\|fg\| 
		&= 		\sup_{x\in X}|(fg)(x)|											\\
		&= 		\sup_{x\in X}|f(x)g(x)|											\\
		&\leq	\sup_{x\in X}|f(x)||g(x)|										\\
		&=		\|f\|\|g\|.
	\end{align*}
	Since $f\mapsto\overline{f}$ is simply pointwise complex conjugation, and the algebraic operations are defined pointwise, it is clear that it is an involution. For the $C^\ast$ axiom: given $f$ in $\CX$,
	\begin{align*}
				\|\bar f f\| 
		&= 		\sup_{x\in X}|(\bar f f)(x)|									\\
		&= 		\sup_{x\in X}|\bar{f(x)} f(x)|									\\
		&=		\sup_{x\in X}|f(x)|^2											\\
		&=		\|f\|^2.
	\end{align*}
	The constant function $\1:X\to\C:x\mapsto 1$ is the unit in $\CX$, with $\|\1\| = 1$.
\end{proof}

Suppose that $f$ is a self-adjoint element of $\CX$ -- that is to say, $\bar f = f$. Then $\bar{f(x)} = 
f(x)$ for all $x$ in $X$, so $f$ is real-valued on $X$. The set of all self-adjoints in $\CX$ is the 
algebra $C(X,\R)$ of all real valued functions on $X$. Call $f$ \emph{positive} if $f(x)$ is positive 
for all $x$ in $X$. 
A linear functional $\rho$ on $\CX$ is \emph{positive} if $\rho(f)\geq 0$ for all positive elements $f$, 
and a \emph{state} on $\CX$ if $\rho$ is positive and $\rho(\1)=1$. The \emph{state space} $\S$, the set 
of all states on $\CX$, is a convex set, and extreme points of $\S$ are called \emph{pure states} on 
$\CX$.

Given $x$ in $X$, define $\rho_x$ by $\rho_x:\CX\to\C:f\mapsto f(x)$. If $f$ is positive, then 
$\rho_x(f) \geq 0$, and $\rho_x(\1) = \1(x)=1$, so $\rho_x$ is a state for all $x$. We can also see that 
these functionals are multiplicative, and it can be shown (see \cite[Corollary 3.4.2]{kadison83}) that 
every non-zero multiplicative functional on $\CX$ arises in this way, as `evaluation' at some point of 
$X$, and the general fact that the non-zero multiplicative linear functionals are exactly the pure 
states on a $C^\ast$-algebra (which we will show in Proposition \ref{prop:puremult}) shows that these 
are in fact all of the pure states on $\CX$.

The following is a result which is crucial in the proof of the commutative Gelfand-Naimark theorem. We 
state it without proof here as the proof is long and involved, and doesn't contribute to understanding. 
The proof, along with a discussion of other consequences of the theorem, can be found in Section 3.4 of 
\cite{kadison83}.

\begin{thm*}[Stone-Weierstrass theorem {\cite[3.4.15]{kadison83}}]
	Let $A$ be a norm-closed subalgebra of $\CX$, containing the constant function $\1$ and the 
	conjugate function $\bar f$ for each $f$ in $A$. If for each pair of distinct points $p\not=p'$ in 
	$X$, there is an $f$ in $A$ such that $f(p)\not=f(p')$, then $A=\CX$.
\end{thm*}

The term `norm-closed' means that the set is closed in the norm topology, meaning it contains all its 
limit points. If the condition in the statement of the theorem holds, say that \emph{$A$ separates the 
points of $X$}.

We finish this section by remarking that if we take $X$ to be a non-compact, \emph{locally} compact 
Hausdorff space, then we are required to take $\CX$ to be the set of continuous complex valued functions 
which \emph{vanish at infinity} in some sense\footnote
{We say $f:X\to\C$ \emph{vanishes at infinity} if for all positive $\epsilon$ there exists a compact 
subset $K$ of $X$ such that $|f(x)|<\epsilon$ for all $x$ outside of $K$.}. This restriction ensures 
that the supremum used to define the norm on $\CX$ exists -- however, we lose the constant function 
$\1$, so $\CX$ is no longer unital. 
It turns out that the unitization of $\CX$ corresponds to the continuous functions on the one-point 
compactification of $X$ -- see \cite[II.1.2.2]{blackadar06}.



\section{The Spectrum}
We will just state the necessary definitions and results here, and defer proofs to sources.

\begin{defn}
	Given an element $a$ of a Banach algebra $A$, define its spectrum $\sigma_A(a)$:
	\[
		\sigma_A(a)=\{\lambda\in \C ~|~ a-\lambda\1 \mbox{ is not invertible in }A\},
	\]
	where $a$ is \emph{invertible in $A$} if there exists an element $b\in A$ such that $ab=\1=ba$. 
	Say that $\lambda\in\spec a$ is a \emph{spectral value} of $a$ in $A$. If it is clear from context 
	where we are taking the spectrum, then we write $\spec{a}$ for $\sigma_A(a)$. 
\end{defn}

\begin{thm*}[{\cite[3.2.3]{kadison83}}]
	If $a$ is an element of a Banach algebra $A$ then $\spec a$ is a non-empty closed subset of the
	closed disk in $\C$ with center $0$ and radius $\|a\|$.
\end{thm*}

The \emph{spectral radius}, $r(a)$, of $a$ is the radius of the smallest disk in $\C$ containing 
$\spec a$; that is, 
\begin{align*}
	r(a)=\sup_{\lambda\in\spec a}{|\lambda|}.
\end{align*}
By the above theorem, $r(a)\leq\|a\|$. 

\begin{thm*}[Spectral radius formula {\cite[3.3.3]{kadison83}}]
	The spectral radius of an element $a$ of a Banach algebra is given by the formula
	\begin{align*}
		r(a)= \lim_{n\to\infty} \|a^n\|^{1/n}.
	\end{align*}
\end{thm*}

\begin{lemma}[{\cite[4.1.1(i)]{kadison83}}]\label{lemma:411}
	If $a$ is a normal element of a $C^\ast$-algebra, then $r(a)=\|a\|$.
\end{lemma}

From this it can be shown that the norm on a $C^\ast$ algebra $A$ is unique, in the sense that only one norm on $A$ can satisfy the $C^\ast$-axiom. \cite[II.1.6.5]{blackadar06}



\begin{thm*}[Continuous functional calculus, {\cite[4.1.3]{kadison83}}]
	Given a self-adjoint element $a$, we denote by $C(\spec a)$ the $C^\ast$-algebra of continuous, 
	complex valued functions on the spectrum of $a$. There is a unique continuous mapping $C(\spec a):f
	\mapsto f(a)$ such that:
	\begin{enumerate}
		\item	if $f$ is a polynomial, $f(z)=\alpha_1+\alpha_2z+\dots+\alpha_mz^m$, then 
						\[f(a)=\alpha_1\1+\alpha_2a+\dots+\alpha_ma^m;\]
		\item	$\|f(a)\| = \|f\|$;
		\item	$(\alpha f+\beta G)(a) = \alpha f(a) + \beta g(a)$;
		\item	$fg(a)=f(a)g(a)$;
		\item	$\overline{f}(a) = (f(a))^\ast$;
		\item	$f(a)$ is normal;
		\item	$f(a)b=bf(a)$ for all $b$ in $A$ such that $ab=ba$.
	\end{enumerate}
	Moreover, each $f$ in $C(\spec a)$ is the limit of a sequence of polynomials over $\C$.
\end{thm*}

\begin{thm*}[Spectral mapping theorem, {\cite[4.1.6]{kadison83}}]
	If $a$ is a self adjoint element of a $C^\ast$-algebra $A$, and $f\in C(\spec a)$, then
	\begin{align*}
		\spec{f(a)} = \{f(t) ~|~ t \in \spec a\}.
	\end{align*}
\end{thm*}

\begin{prop}[{\cite[4.2.3(i)]{kadison83}}]\label{prop:423}
	Given $f$ in $C(\spec a)$, $f(a)$ is positive if and only if $f(t)\geq0$ for all $t$ in $\spec a$.
\end{prop}

\begin{thm*}[{\cite[4.1.5]{kadison83}}]
	If $B$ is a $C^\ast$-subalgebra of a $C^\ast$-algebra $A$, and $b$ is an element of $B$, then 
	\[
		\sigma_B(b) =\sigma_A(b).
	\]
\end{thm*}


As an aside, this theorem allows us to make the convention that the spectrum of an element $a$ in a 
non-unital $C^\ast$-algebra $A$ is defined to be the~spectrum of $a$ in the unitization $\tilde A$ 
of $A$.




\section{Further Definitions and results}
In this section we will set out some further definitions, some of which we have already seen in the context of $\CX$, and prove some cool results.

\begin{defn}
	An element $a$ of a $C^\ast$-algebra $A$ is \emph{positive} if it is self-adjoint and $\spec a
	\subseteq \R^+$. 
	Denote the set of positive elements in $A$ by $A^+$. Then $A_{sa}$ is a partially ordered vector space 
	with $\leq$ given by 
	\begin{align*}
		a\leq b ~~\iff~~ b-a \mbox{ is positive}
	\end{align*}
	for $a,b$ in $A_{sa}$.
	The set of positive elements form a positive cone in $A_{sa}$, which means that 
	\begin{enumerate}
		\item 	$\alpha a+b\in A^+$ for all $a,b \in A^+$ and $\alpha\in \R^+$,
		\item	$a\in A^+$ and $-a\in A^+$ implies that $a=0$.
	\end{enumerate}
\end{defn}

The unit $\1$ is positive (with $\spec \1 = \{1\}$). 

\begin{lemma}[{\cite[4.2.3(ii)]{kadison83}}]\label{lemma:423}
	For a self-adjoint element $a$ in a $C^\ast$-algebra $A$, 
	\begin{align*}
		-\|a\|\1\leq a\leq\|a\|\1.
	\end{align*}
\end{lemma}
\begin{proof}
	Let $f$ in $C(\spec a)$ defined by $f(t) = \|a\| \pm t$, f takes non-negative values on $\spec a$, so by Proposition \ref{prop:423}, $f(a)$ is positive; that is, $\|a\|\1 \pm a$ is positive.
\end{proof}
	
\begin{defn}
	Given Banach $\ast$-algebras $A$ and $B$, a map $\phi:A\to B$ is a \emph{$\ast$-homomorphism} 
	if it is an algebra homomorphism for which $\phi(a^\ast) = \phi(a)^\ast$ for all $a \in A$.
	If a $\ast$-homomorphism is one-to-one, call it a \emph{$\ast$-isomorphism}.
\end{defn}

\begin{prop}[{\cite[4.1.8]{kadison83}}] \label{prop:homo}
	Suppose $A$ and $B$ are $C^\ast$-algebras and $\phi:A\to B$ is a $\ast$-homomorphism. Then
	$\|\phi(a)\| \leq \|a\|$ for all $a \in A$. If $\phi$ is a $\ast$-isomorphism, then
	$\|\phi(a)\| = \|a\|$ for all $a \in A$.
\end{prop}
\begin{proof}
	With $a$ in $A$, if $\alpha$ is not a spectral value for $a$, $\alpha\1_A-a$ has an inverse $s$ in 
	$A$. Since $\phi(\1_A) =\1_b$, $\alpha \1_A-\phi(a)$ has inverse $\phi(s)$ in $B$, so $\alpha$ is 
	not a spectral value of $\phi(a)$; hence $\spec {\phi(a)}\subseteq \spec a$, and in particular, 
	$r(\phi(a))\leq r(a)$. By Lemma \ref{lemma:411}, since $a^\ast a$ is normal, we have 
	\begin{align*}
		\|a\|^2			&=	\|a^\ast a\| 	= 	r(a^\ast a); 	\mbox{ and }	\\
		\|\phi(a)\|^2 &=\|\phi(a)^\ast\phi(a)\| = 
							 \|\phi(a^\ast a)\| = r(\phi(a^\ast a)).
	\end{align*}
	so $\|\phi(a)\|\leq \|a\|$.
	
	If $a$ is a self-adjoint element of $A$ and $f \in C(\spec a)$, then $\phi(f(a))=f(\phi(a))$. This 
	follows from the final assertion of the continuous functional calculus.
	
	Now suppose $\phi$ is a $\ast$-isomorphism, and for a self-adjoint $a$ in $A$, suppose $
	\spec{\phi(a)} \subsetneq \spec a$. Then there is a non-zero $f$ in $C(\spec a)$ whose restriction 
	to $\spec{\phi(a)}$ is 0. From the previous paragraph, however, $\phi(f(a))=f(\phi(a))=0$, so that 
	$f(a)\not= 0$ is in the kernel of $\phi$; thus contradicting the assumption that $\phi$ is 
	one-to-one, and we conclude that $\spec{\phi(a)} = \spec a$. A similar argument to the one in the 
	first half of this proof shows that $\|\phi(a)\|=\|a\|$ and completes the proof.
\end{proof}
This result shows that $\ast$-homomorphisms between $C^\ast$-algebras are 
automatically bounded (hence continuous), and $\ast$-isomorphisms are 
isometries, demonstrating how the algebraic and topological structure of these 
spaces is tied together by the $C^\ast$ axiom.


\begin{defn}
	A linear functional $\rho$ on a $C^\ast$-algebra $A$ is \emph{positive} if $\rho(a)\geq 0$ for all 
	$a\in A^+$. A \emph{state} on $A$ is a positive linear functional $\rho$ such that $\rho(\1)=1$. 
	Denote the set of all states on $A$ by $\S(A)$. 
	The equation $\rho^\ast(a) =\bar{\rho(a^\ast)}$ defines a functional $\rho$, and call $\rho$ 
	\emph{Hermitian} if $\rho=\rho^\ast$.
\end{defn}
Hermitian functionals take real values on self-adjoint elements, and all positive linear functionals are 
Hermitian.
We can extend the partial order notation to linear functionals. Given linear functionals $\rho,\tau$ on 
$A$, define $\leq$ by
\begin{align*}
	\rho \leq \tau ~~\iff~~ \tau - \rho \mbox{ is positive}.
\end{align*}

When we take $\C$ as a $C^\ast$-algebra, the positive elements are exactly the 
positive reals, and the linear functionals are given by multiplication by a 
complex number. Positive linear functionals correspond again to the positive 
reals, and the only state on $\C$ is `multiplication by $1$'.
Extending the analogy between $\C$ and a $C^\ast$-algebra $A$, we can even show 
that a positive element has a unique positive square root, and that $a^\ast a$ 
is positive for all $a$ in $A$. (See \cite[4.2.5 and 4.2.6]{kadison83}).



\begin{prop}[{\cite[4.3.2]{kadison83}}]\label{prop:432}
	A linear functional $\rho$ on a $C^\ast$-algebra $A$ is positive if and only if $\rho$ is bounded
	and $\|\rho\| = \rho(\1)$.
\end{prop}
\begin{proof}
	Suppose that $\rho$ is positive. With $a$ in $A$, let $\alpha$ be a scalar of modulus $1$ 
	such that $\alpha\rho(a)\geq 0$, and let $h$ be the real part of $a$.
	Since $h$ is self adjoint, (by 4.2.3(ii)) we have $h \leq \|h\|\1 \leq \|a\|\1$. Thus, $\|a\|\1-h$
	is positive and 
	\[
		\rho(\|a\|\1-h) = \|a\|\rho(\1)-\rho(h) \geq 0.
	\]
	Therefore,
	\begin{align*}
		|\rho(a)| = \rho(\alpha a) = \overline{\rho(\alpha a)} = \rho(\overline{\alpha}a^\ast) = 
					\rho(\tfrac{1}{2}(\alpha a +\overline{\alpha}a^\ast)) =
					\rho(h) \leq \rho(\1) \|a\|,
	\end{align*}
	so $\rho$ is bounded and $\|\rho\| \leq \rho(\1)$. We also have $\|\rho\| = \sup\{\rho(a)~|~\|a\|=
	1\} \geq \rho(\1)$, so $\|\rho\|=\rho(\1)$.
	
	Conversely, suppose $\rho$ is bounded and $\|\rho\| = \rho(\1)$ -- we can assume without loss that 
	$\rho(\1)=1$. With $a$ a positive element of $A$, let $\rho(a)=\alpha+i\beta$ for real $\alpha
	,\beta$. Then $\rho$ is positive if and only if $\alpha \geq 0$ and $\beta =0$.
	For small positive $s$, 
	\begin{align*}
		\spec{\1-sa} = \{1-st ~|~ t\in \spec{a} \subseteq \R^+\} \subseteq [0,1],
	\end{align*}
	so $\|\1-sa\| = r(\1-sa) \leq 1$. Hence
	\begin{align*}
		1-s\alpha \leq |1-s(\alpha+i\beta)| = |\rho(\1 - sa)| \leq 1,
	\end{align*}
	so $\alpha\geq0$. With $b_n$ in $A$ defined by $b_n= a+(in\beta-\alpha)\1$ for each positive
	integer $n$,
	\begin{align*}
				\|b_n\|^2 = \|b_n^\ast b_n\| 
		&= 		\|(a-\alpha\1)^2 + n^2\beta^2\1\| 									\\
		&\leq 	\|a-\alpha\1\|^2 +n^2\beta^2.
	\end{align*}
	Hence for all positive integers $n$, we have
	\begin{align*}
				(n^2+2n+1)\beta 
		&=	 	|\rho(b_n)|^2  														\\
		&\leq 	\|a-\beta\1\|^2 +n^2\beta^2,
	\end{align*}
	so that $\beta=0$. We conclude that $\rho$ is positive.
\end{proof}



The following result follows easily from the Cauchy-Schwarz inequality for inner products.
\begin{prop*}[Cauchy-Schwarz inequality, {\cite[4.3.1]{kadison83}}]
	If $\rho$ is a positive linear functional on $C^\ast$-algebra $A$, then for all $a$ and $b$ in $A$,
	\begin{align*}
		|\rho(b^\ast a)|^2 \leq \rho(a^\ast a)\rho(b^\ast b).
	\end{align*}
\end{prop*}


\begin{lemma}[{\cite[4.3.3]{kadison83}}]\label{lemma:433}
	Let $A$ be a $C^\ast$-algebra. For any $a$ in $A$ and $\alpha\in\spec{a}$, there exists a state
	$\rho$ on $A$ such that $\rho(a)=\alpha$.
\end{lemma}
\begin{proof}
	For all complex numbers $\beta$ and $\gamma$, $\alpha\beta+\gamma$ is a spectral value for the 
	element $\beta a+\gamma\1$ of $A$, so 
	\[
		|\alpha\beta+\gamma|\leq r(\beta a+\gamma\1)=\|\beta a+\gamma\1\|.
	\] 
	Hence the equation $\rho_0(s)=\alpha\beta+\gamma$ defines a bounded linear functional
	$\rho_0$ on the linear subspace $B= \{\beta a+\gamma\1 ~|~ \beta,\gamma\in\C\}$ of $A$, with 
	$\rho_0(a) =\alpha$ and $\rho_0(\1)=1=\|\rho_0\|$. By the Hahn-Banach theorem, 
	$\rho_0$ extends to a bounded linear functional $\rho$ on $A$, with $\|\rho\|=1$, such that 
	$\rho=\rho_0$ on the subspace $B$. In particular, $\rho(\1)=1=\|\rho\|$ so $\rho$ is positive by 
	the previous result, and $\rho(a)=\alpha$.
\end{proof}

\begin{lemma}[{\cite[4.3.4,(i)]{kadison83}}]\label{lemma:state1}
	Let $A$ be a $C^\ast$-algebra. If $\rho(a)=0$ for all states $\rho$ on $A$, then $a=0$.
\end{lemma}
\begin{proof}
	Suppose first that $a$ is self-adjoint, and $\rho(a)=0$ for all states $\rho$. By the previous
	result, $\spec{a}=\{0\}$, so $\|a\|=r(a)=0$ (using Lemma \ref{lemma:411}). Hence $a=0$.			\\
	Now write $a=h+ik$, for $h$ and $k$ the real and imaginary part of $a$ respectively. Then
	\begin{align*}
		0=\rho(a)=\rho(h)+i\rho(k),
	\end{align*}
	and as $h,k$ are self-adjoint, $\rho(h)$ and $\rho(k)$ are real and we must have 
	$\rho(h)=0=\rho(k)$. By previous statement, $h=0=k$, and we conclude that $a=0$.
\end{proof}

\begin{lemma}[{\cite[4.3.4,(iv)]{kadison83}}]\label{lemma:state2}
	If $a$ is a normal element of a $C^\ast$-algebra $A$, there is a state $\rho$ on $A$ such 
	that $|\rho(a)|=\|a\|$.
\end{lemma}
\begin{proof}
	By Lemma \ref{lemma:411}, $r(a)=\|a\|$, so $\spec a$ contains a complex number $\alpha$ such that $|\alpha|=\|a\|$. By Lemma \ref{lemma:433}, $\alpha=\rho(a)$ for some state $\rho$, so 
	\begin{align*}
		|\rho(a)|=|\alpha|=\|a\|.
	\end{align*}
\end{proof}

\begin{defn}
	An extreme point of $\S(A)$ is called a \emph{pure} state on $A$, and the set of pure 
	states on $A$ is denoted by $\P{A}$.
\end{defn}
The state space $\S(A)$ is then the closure, relative to the weak$^\ast$ 
topology, of the set of convex linear combinations of pure states. 
It is a simple exercise, using the fact that $A^+\subseteq A_{sa}$, to verify 
that a linear functional $\rho$ is a pure state on $A$ if and only if the 
restriction $\rho|_{A_{sa}}$ is a purestate on $A_{sa}$.
Conversely, every pure state on $A_{sa}$ extends to a pure state on $A$ with the 
same norm, by the Hahn-Banach theorem.
 

\begin{lemma}[{\cite[4.3.8,(i)]{kadison83}}]\label{lemma:pure1}
	Let $a$ be an element of a $C^\ast$-algebra $A$. If $\rho(a)=0$ for all \emph{pure} states $\rho$ on 
	$A$, then $a=0$.
\end{lemma}
\begin{proof}
	Since every state is the limit of a sequence of linear combinations of pure states, if $\rho(a)=0$ 
	for all pure states $\rho$ on $A$, then $\rho(a)=0$ for \emph{all} states $\rho$ on $A$. The result 
	follows immediately from Lemma \ref{lemma:state1}.
\end{proof}

The following is a useful characterisation of pure states.
\begin{lemma}[{\cite[4.3.8,(iv)]{kadison83}}]\label{lemma:pure2}
	If $a$ is a normal element of a $C^\ast$-algebra $A$, there is a pure state $\rho_0$ on $A$ such 
	that $|\rho_0(a)|=\|a\|$.
\end{lemma}
\begin{proof}
	By Lemma \ref{lemma:state2}, there is a scalar $\gamma$ and a state $\tau$ on $A$ such that $
	\tau(a)=\gamma$ and $|\gamma|=\|a\|$. Define $\hat t :A^\ast\to\C:\rho \mapsto\rho(t)$, and let $
	\alpha$ be a complex number such that $|\alpha|=1$ and $\tau(\alpha a) = |c| = \|a\|$. From 
	Lemma \ref{lemma:144}, there is a $\rho_0$ in $\P A$ such that 
	\begin{align*}
				\|a\|
		&\geq 	|\rho_0(a)| \geq \mbox{Re } \widehat {\alpha a}(\rho_0)					\\
		&\geq 	\sup_{\rho\in \S(A)} \mbox{Re }\widehat {\alpha a}(\rho) 				\\
		&\geq	\mbox{Re }\widehat {\alpha a}(\tau) = \mbox{Re }\tau(\alpha a) = \|a\|,
	\end{align*}
	so that $|\rho_0(a)|=\|a\|$.
\end{proof}

\begin{prop}[{\cite[3.4.6]{kadison83}}]\label{prop:pure2}
	A state $\rho$ on $A_{sa}$ is pure if and only if, for all positive linear functionals $\tau$ on
	 $A_{sa}$ such that $0\leq\tau\leq\rho$, we have $\tau =\lambda\rho$ for some $\lambda\in\R$. 
\end{prop}
\begin{proof}
	Suppose that $\tau =\lambda\rho$ for all $0\leq\tau\leq\rho$, and suppose we can write 
	$\rho=\alpha\rho_1+(1-\alpha)\rho_2$ for some $0\leq\alpha\leq1$ and some $\rho_1,\rho_2 \in
	\S(A_{sa})$. Then $0\leq\alpha\rho_1\leq\rho$, so $\alpha\rho_0 = \lambda\rho$. Then 
	$\rho_1(\1)=1=\rho(\1)$, so $\alpha = \lambda$ and $\rho_0 =\rho$. Similarly, we can show that 
	$\rho_2=\rho$, and so we conclude that $\rho$ is a pure state.
	
	Conversely, suppose that $\rho$ is a pure state and $0\leq\tau\leq\rho$. Applying this to $\1$,
	we get $0\leq\tau(\1)\leq\rho(\1) = 1$. Let $\lambda = \tau(\1)$. We work case-by-case: \\
	If $\lambda=0$, then for any $a\in A_{sa}$, applying $\tau$ to $-\|a\|\1\leq a\leq\|a\|\1$ 
	(from Lemma \ref{lemma:423}) gives
	\begin{align*}
		0=-\|a\|\lambda=\tau(-\|a\|\1)\leq\tau(a)\leq\tau(\|a\|\1)=\|a\|\lambda=0,
	\end{align*}
	so $\tau=0=\lambda\rho$. \\
	A similar argument shows that $\lambda=1$ implies $\tau-\rho=0$, so that $\tau=\rho=\lambda\rho$.\\
	If $0\leq\lambda\leq 1$, we can write $\rho=\lambda\rho_1+(1-\lambda)\rho_2$ for 
	$\rho_1=\lambda^{-1}\tau$ and $\rho_2=(1-\lambda)^{-1}(\rho-\tau)$. $\rho$ is pure so 
	$\tau=\lambda\rho_1=\lambda\rho$.
	
\end{proof}
Recall that a linear functional $\rho$ on a $C^\ast$-algebra $A$ is 
\emph{multiplicative} if $\rho(ab)=\rho(a)\rho(b)$ for all $a,b$ in $A$. The set 
of all non-zero multiplicative linear functionals on $A$ is called the 
\emph{maximal ideal space} of $A$, denoted $\M_A$. This name hints at the fact 
that the kernel of each of these functionals is a maximal ideal of $A$, and all 
maximal ideals of $A$ arise in this way. \cite[Theorem I.2.5]{davidson96}




\begin{prop}[{\cite[4.4.1]{kadison83}}]\label{prop:puremult}
	The set of pure states on an Abelian $C^\ast$-algebra $A$ is precisely the maximal ideal space of 
	$A$.
\end{prop}
\begin{proof}
	Suppose $\rho$ is a pure state on $A$. To show that $\rho(ab)=\rho(a)\rho(b)$ for $a,b,\in A$,
	we restrict attention to the case where $0\leq b\leq\1$. Linearity gives us the general case.
	In this case, for $h\in A^+$ we have that $0\leq hb\leq h$, so $0\leq\rho(hb)\leq\rho(h)$.
	Hence $\rho_b(a)=\rho(ab)$ for $a\in A$ defines a positive linear functional on $A$ with 
	$\rho_b\leq\rho$. The restriction $\rho|_{A_{sa}}$ is a pure state on $A_{sa}$ and 
	$\rho_b|_{A_{sa}} \leq \rho|_{A_{sa}}$, and it follows from Proposition \ref{prop:pure2} that 
	$\rho_b|_{A_{sa}} = \alpha \rho|_{A_{sa}}$ for some $\alpha \in \R^+$.
	Hence $\rho_b = \alpha\rho$  and so for $a\in A$:
	\begin{align*}
		\rho(ab) = \rho_b(a) = \alpha\rho(a) = 
							\alpha\rho(\1)\rho(a) = \rho_b(\1)\rho(a) = \rho(b)\rho(a)
	\end{align*}
	
	Conversely, suppose $\rho$ is a multiplicative linear functional. By Lemma \ref{lemma:3220}, $
	\rho$ is bounded and $\|\rho\|=\rho(\1)=1$, so by Proposition \ref{prop:432}, $\rho$ is a state.
	Suppose we can write $\rho=\alpha\rho_1+\beta\rho_2$ for states $\rho_1,\rho_2$ on $A$ and 
	$\alpha,\beta >0$ such that $\alpha+\beta=1$. For $c\in A_{sa}$, by the Cauchy-Schwarz inequality
	 we have for $j=1,2$:
	\begin{align*}
		\big(\rho_j(c)\big)^2 = \big(\rho_j(\1 c)\big)^2 \leq \rho_j(\1)\rho(c^2)=\rho(c^2).
	\end{align*}
	Then:
	\begin{align*}
				0
		&=		\rho(c^2)-\rho(c)^2 											\\
		&=		\alpha\rho_1(c^2)+\beta\rho_2(c^2) 
						- \big(\alpha\rho_1(c)+\beta\rho_2(c)\big)^2			\\
		&\geq	\alpha(\alpha+\beta)\rho_1(c)^2 
						+ \beta(\alpha+\beta)\rho_2(c)^2
						- \big(\alpha\rho_1(c)+\beta\rho_2(c)\big)^2			\\
		&=		\alpha\beta\big(\rho_1(c) - \rho_2(c)\big)^2.
	\end{align*}
	Hence $\rho_1(c)=\rho_2(c)$, for all $c\in A_{sa}$, so $\rho_1=\rho_2 =\rho$ and we conclude
	that $\rho$ is a pure state. 
\end{proof}


The set $\P A$ of pure states on $A$ is a subset of the dual space $A^\ast$, and when we take $A^\ast$ with the weak$^\ast$ topology, we get a compact Hausdorff space, which is the object of study in the Gelfand-Naimark theorem for commutative algebras.

\begin{lemma}[{\cite[3.2.20]{kadison83}}]\label{lemma:3220}
	The maximal ideal space of an Abelian Banach algebra $A$ form a compact subset of the unit ball of 
	the dual space $A^\ast$, relative to the weak$^\ast$ topology.
\end{lemma}
\begin{proof}
	Follows from the Banach-Alaoglu theorem.
\end{proof}




\section{$\B(\H)$ -- an example}
This section concerns the fundamental example of a $C^\ast$-algebra - the set $\B(\H)$ of bounded 
linear operators on a Hilbert space $\H$, otherwise known as \emph{operator algebras}. Here we will 
demonstrate that $\B(\H)$ is a $C^\ast$-algebra and discuss how the theory of $C^\ast$-algebras relates 
to quantum mechanics, the very subject which started the study of these objects, explaining a good deal 
of the terminology used here. 


%% potentially give a specific example of \H=\C^n

\begin{claim} $\B(\H)$ is a $C^\ast$-algebra with the operator norm 
\[
	\|T\|= \sup_{\|x\|=1}{\|Tx\|}
\]
and involution taking $T$ to its adjoint map $T^\ast$. 
The identity map $I:x\mapsto x$ is a unit for $\BH$
\end{claim}
\begin{proof}
	It is an easy exercise to show that $\|\cdot\|$ is a norm on $\BH$. Let $\{T_n\}_{n\in\N}$ be a 
	Cauchy sequence in $\BH$. Then for any positive $\epsilon$, there is a positive integer $N$ such 
	that 
	\begin{align*}
		\|T_m-T_n\| < \epsilon \mbox{ for all } m,n \geq N.
	\end{align*}
	Applying $T_m-T_n$ to $x \in \H$, we have 
	\begin{align}\label{eqn:cauchyTn}
		\|T_mx-T_nx\| \leq \|T_m-T_n\| \|x\| < \epsilon \|x\|,
	\end{align}
	so $\{T_nx\}_{n\in\N}$ is a Cauchy sequence in $\H$, converging to an element in $\H$.
	Define a linear operator $T:\H \to \H$ by 
	\begin{align*}
		Tx= \lim_{n\to\infty}{T_nx} \mbox{ for } x \in \H.
	\end{align*}
	Taking limits as $m$ tends to infinity in equation \eqref{eqn:cauchyTn}, we obtain
	\begin{align*}
		\|Tx-T_nx\| < \epsilon \|x\| \mbox{ for all }n \geq N,
	\end{align*}
	and so we have that $T-T_n$ (and hence $T=(T-T_n)+T_n$) is a bounded operator and  
	\begin{align*}
		\|T-T_n\| <\epsilon \mbox{ for all }n \geq N.
	\end{align*}
	We conclude that $T_n \to T$, and so $\BH$ is complete.
	
	Since boundedness is equivalent to continuity on $\H$, given $S,T\in\BH$, the operator 
	$ST:\H \to \H: x \mapsto (S\circ T)(x)$ is bounded on $\H$.
	Given $x\in\H$ and $\lambda\in\C$, 
	\begin{align*}
			((\lambda S)T)(x)
		&=	((\lambda S)\circ T)(x)												\\
		&=	\lambda S(Tx)														\\
		&=	\lambda (S\circ T)(x)												\\
		&=	\lambda ST(x),
	\end{align*} 
	so that $(\lambda S)T = \lambda ST$ in $\BH$. We have 
	\begin{align*}
				\|ST\|
		&=		\sup_{\|x\|=1}{\|STx\|} 										\\
		&=		\sup_{\|x\|=1}{\|S(Tx)\|} 										\\
		&\leq	\|S\| \sup_{\|x\|=1}{\|Tx\|} 									\\
		&=		\|S\| \|T\|.
	\end{align*}
	We conclude that $\BH$ is a Banach algebra.
	
	To see that $^\ast$ is an involution, use the fact that the adjoint 
	operator is unique for each operator and the following equalities.
	\begin{enumerate}
	\Item	\begin{align*}
				\langle (\alpha T+S)^\ast x,y\rangle 
			&=	\langle x, \alpha T+S y \rangle									\\
			&=	\overline{\alpha}\langle x,Ty\rangle + \langle x, Sy \rangle	\\
			&=	\overline{\alpha}\langle T^\ast x,y\rangle +
										 \langle S^\ast x, y \rangle			\\
			&=	\langle(\overline{\alpha} T^\ast + S^\ast) x, y \rangle.
			\end{align*}
	\Item 	\begin{align*}
				\langle (T^\ast)^\ast x,y\rangle 
			&=	\langle x, T^\ast y \rangle										\\
			&=	\overline{\langle T^\ast y,x \rangle}							\\
			&=	\overline{\langle y, Tx \rangle}								\\
			&=	\langle Tx,y\rangle.
			\end{align*}
	\Item	\begin{align*}
				\langle (ST)^\ast x,y \rangle
			&=	\langle x, STy \rangle											\\
			&=	\langle	S^\ast x,Ty \rangle										\\
			&=	\langle T^\ast S^\ast x,y \rangle.
		\end{align*}
	\end{enumerate}
	It remains to demonstrate the $C^\ast$-axiom on $\BH$. For all $x\in\H$, we have
	\begin{align*}
		\|Tx\|^2 = \langle Tx,Tx\rangle = \langle T^\ast Tx,x\rangle \leq \|T^\ast T\| \|x\|^2,
	\end{align*}
	so that
	\begin{align*}
		\|T\|^2 \leq \|T^\ast T\| \leq \|T^\ast\| \|T\| = \|T\|^2.
	\end{align*}
	It is clear that $I$ is a unit.
	Hence, the claim.
\end{proof} 



The terminology for normal, positive and unitary elements of a $C^\ast$-algebra comes from the 
corresponding concepts when studying operator algebras. 
Paul Dirac's \emph{`Principles of Quantum Mechanics'} and von Neumann's \emph{`Mathematical Foundations 
of Quantum Mechanics'} jointly established a formulation of Quantum Mechanics in terms of 
Hilbert spaces, and operators thereon. To summarise briefly, a physical quantity in a quantum mechanical 
system has associated with it a self-adjoint operator $T$ on a Hilbert space $\H$,  while the state of 
the system corresponds to a unit vector $\phi$ in $\H$. Then the expected outcome of measuring the 
quantity corresponding to $T$ is given by evaluating $\langle\phi, T\phi\rangle$. In the context of $C^
\ast$-algebras, these translate, in turn, to a self adjoint element $a$ of a $C^\ast$-algebra $A$, a 
normalised positive linear functional $\rho$ (that is, a state) on $A$, and evaluation of $\rho(a)$.

After stating his postulate, which he refers to as `\textbf{P.}', on the expectation value of a system 
of observables $\mathscr{R_1},\dots,\mathscr{R_\ell}$ with corresponding operators $R_1,\dots, R_\ell$, 
von Neumann makes a comment on a consequence in the model of the non-commutativity of operators on a 
Hilbert space:
\textquote[{\cite[p.~211]{vonneumann32}}]{In the case of noncommuting $R_1,\dots, R_\ell$, [...] 
\textbf{P.} gave no information regarding the probability relations of $\mathscr{R_1},\dots,\mathscr{R_
\ell}$. In this case, \textbf{P.} could be used only determine the probability distribution of each of 
these quantities by itself, without consideration of the others.}




\chapter{Representations of $C^\ast$-algebras}
\section{Abelian $C^\ast$-algebras}\label{section:abelian}


Let $A$ be an Abelian $C^\ast$-algebra. For $a$ in $A$, define a complex-valued function $\hat{a}$ on
$\P{A}$ by $\hat{a}(\rho)= \rho(a)$ -- that is, \emph{evaluation} at $a$. The weak$^\ast$-topology is 
the weakest topology on $\P{A}$ for which all of the maps $\hat{a}$ are continuous, so that $\hat{a} \in 
C(\P{A})$ for all $a\in A$. The map
\[
	\Gamma:A \to C(\P{A}) : a\mapsto \hat{a}
\]
is called the \emph{Gelfand transform} on $A$ \cite{davidson96}.
For $a,b \in A$, $\alpha,\beta\in\C$ and $\rho\in\P{A}$:
\begin{align*}
	\widehat{(\alpha a+\beta b)}(\rho) = \rho(\alpha a+\beta b) 
	&=	\alpha\rho(a) +\beta\rho(b) = \alpha\hat{a}(\rho)+ \beta\hat{b}(\rho), 	\\
	\widehat{a^\ast}(\rho) = \rho(a^\ast) &= \overline{\rho(a)} = \overline{\hat{a}(\rho)}.
\end{align*}
Since pure states are multiplicative by Proposition \ref{prop:puremult}, we have that 
\begin{align*}
	\widehat{(ab)} (\rho)= \rho(ab) = \rho(a)\rho(b) = \hat{a}(\rho)\hat{b}(\rho).
\end{align*}
Hence the Gelfand transform is a $\ast$-homomorphism, and the image $\Gamma(A)$ is a $C^\ast$-subalgebra 
of $C(\P A)$. The following theorem gives us $\Gamma$ is in fact a $\ast$-isomorphism.

\begin{thm}[{Gelfand-Naimark, commutative \cite[4.4.3]{kadison83}}] \label{thm:gnc}
	Every Abelian $C^\ast$-algebra $A$ is $\ast$-isomorphic to $C(\P A)$, the algebra of continuous
	functions on the compact Hausdorff space $\P A$ of pure states on $A$.
\end{thm}
\begin{proof}
	Since $A$ is Abelian, every $a$ in $A$ is normal, so by Lemma \ref{lemma:pure2} there is a pure 
	state $\rho_0$ on $A$ such that $|\rho_0(a)|=\|a\|$. From this, 
	\begin{align*}
		\|a\|=|\hat a(\rho_0)|\leq \sup_{\rho\in\P A}|\hat a(\rho)|\leq\|a\|,
	\end{align*}
	so 
	\begin{align*}
		\|a\|=\sup_{\rho\in\P A} |\hat a (\rho)| = \|\hat a\|.
	\end{align*}
	Hence $\Gamma$ is isometric.
	Given any pure state $\rho$, $\hat \1(\rho) = \rho(\1) = 1$ (where $\1$ is the identity of $A$), so
	$\hat\1$ is the constant function equal to $1$ on $\P A$ and thus the unit in $C(\P A)$.
	Given distinct pure states $\rho_1$ and $\rho_2$ on $A$, we can choose $a$ in $A$ such that 
	\begin{align*}
		\hat a(\rho_1) = \rho_1(a) \not= \rho_2(a) = \hat a(\rho_2),
	\end{align*}
	so the image $\Gamma(A)$ separates points of $\P A$, and so by the Stone-Weierstrass theorem (see 
	Section \ref{section:CX}), $\Gamma(A)=C(\P A)$.
\end{proof}

This theorem allows us to think of the study of $C^\ast$-algebras in general as 
a sort of non-commutative topology; the `topological space' corresponds to the 
pure state space on a $C^\ast$-algebra $A$, and the `topology' corresponds to 
the preimages of the `evaluation at $a$' maps on states in the pure state space.


	
\section{The Gelfand-Naimark-Segal construction}
In this section, we show that corresponding to every state of a 
$C^\ast$-algebra, there is a $\ast$-homomorphism into the operator algebra $\B 
(\H)$ on some Hilbert space $\H$. 

\begin{defn}
	Given a $C^\ast$-algebra $A$, a \emph{representation of $A$ on a Hilbert space $\H$} 
	is a $\ast$-homomorphism $\phi: A \to \BH$. 
	A $\ast$-isomorphic representation is called \emph{faithful}.
	If there exists an element $x\in\H$ such that the set $\{\phi(a)x ~|~ a\in A\}$ is dense in $\H$, 
	say that $\phi$ is a \emph{cyclic} representation, with \emph{cyclic vector} $x$.
\end{defn}	
The construction used in the proof of this theorem is known as the Gelfand-Naimark-Segal (GNS)
construction.
\begin{thm}[{\cite[4.5.2]{kadison83}}]\label{thm:gns}
	If $\rho$ is a state on a $C^\ast$-algebra $A$, then there exists a cyclic representation 
	$\pi_\rho$ of $A$ on a Hilbert space ${H}_\rho$, with unit cyclic vector $x_\rho$, such that 
	\[ 
		\rho(a)= \langle \pi_\rho (a) x_\rho, x_\rho \rangle, ~~~~ \forall a \in A.
	\]
\end{thm}
\begin{proof}
	We will construct from $\rho$ the space $\Hr$,  representation $\pi_\rho$, 
	and vector $x_\rho$, and demonstrate the required properties.
	
	Consider the \emph{left kernel} of $\rho$:
	\[
		L_\rho = \{a \in A ~|~ \rho (a ^\ast a) = 0 \}.
	\]	
	For $a,b \in A$, define $\langle a , b \rangle_0 = \rho(b^\ast a)$, then 
	$\langle \cdot, \cdot \rangle_0$ satisfies
	\begin{enumerate}
	  \item Linearity in 1st argument: for $a,b\in A$, $\alpha, \beta \in \mathbb{C}$:
		\begin{align*}
		   \langle \alpha a + \beta b, c \rangle_0 
		&= \rho (c^\ast(\alpha a + \beta b)   								\\
		&= \rho (\alpha c^\ast a + \beta c^\ast b)  						\\
		&= \alpha \rho (c^\ast a) + \beta \rho (c^\ast b)					\\
		&= \alpha \langle a , c \rangle_0 + \beta \langle b, c \rangle_0.
		\end{align*}
	  \item Conjugate symmetric: for $a,b \in A$:
	  	\begin{align*}
	  	   \langle b,a \rangle _0 
	  	&= \rho (a^\ast b)													\\
	  	&= \rho ((b^\ast a)^\ast)											\\
	  	&= \overline{\rho (b^\ast a)}										\\
	  	&= \overline{\langle a,b \rangle _0 }.
	  	\end{align*}
	  \item Positive semi-definite: for $a \in A$:
	  	\begin{align*}
	  	&=	\langle a,a\rangle_0 = \rho(a^\ast a) \geq 0,
	  	\end{align*}
	  	since $a^\ast a$ is positive for all $a$.
	\end{enumerate}
	Note that $\langle \cdot, \cdot \rangle$ is not necessarily positive definite on 
	$A$ -- the left kernel is exactly where this fails.
	
	To see that $L_\rho$ is a linear subspace of $A$, consider 
	\[
		L= \{t \in A  ~|~ \langle t,a \rangle _0 = 0, ~\forall a \in A \}\subseteq L_\rho.
	\]
	For $t \in L_\rho$, by the Cauchy-Schwarz inequality we have 
	\[ 
		|\langle t,a \rangle_0|^2 \leq \langle t,t \rangle_0 \langle a,a\rangle_0,~~ \forall a \in A;
	\]
	that is,
	\[
		\langle t,a \rangle _0 = 0, ~~~ \forall a \in A,
	\]
	so $t\in L$ and $L_\rho =L$.
	Now, for $a,b \in L$, $\alpha \in \mathbb{C}$ and $c \in A$:
	\[
		\langle \alpha a + b, c \rangle _0 = \alpha \langle a,c \rangle _0 + 
						\langle b,c\rangle _0 = 0,
	\]
	so $\alpha a +b \in L$; also, $\langle 0,c\rangle _0 = 0$ so $0 \in L$. 
	Hence, $L_\rho  $ is a linear subspace of $A$.
	
	For $s \in A$, $t\in L_\rho$, by the Cauchy Schwarz inequality we have 
	\begin{align*}
		|\rho (s^\ast t) |^2 
		&= 		|\langle t,s\rangle_0 |^2 									\\
		&\leq 	\langle t,t\rangle_0 \cdot \langle s,s\rangle_0  			\\
		&= 		\rho (t^\ast t) \cdot \rho (s^\ast s)						\\
		&=		0,
	\end{align*}
	so $\rho (s^\ast t) = 0$. Letting $s = a^\ast a t$ for $a \in A$, then
	\begin{align*}
		\rho ((at)^\ast at) 
		&= 		\rho (at^\ast a^\ast at) 									\\
		&= 		\rho ((a^\ast at)^\ast t) 									\\
		&= 		\rho (s^\ast t) 											\\
		&=		0,
	\end{align*}
	so that $at \in L_\rho$, for all $a \in A$ and $t \in L_\rho$; 
	we conclude that $L_\rho$ is a left ideal in $A$. $L_\rho$ is closed as the preimage in $A$ of $
	\{0\}$ under the continuous map $a \mapsto \rho (a^\ast a)$. 
	
	Consider now the quotient space $V_\rho = A / L_\rho$, with $\langle \cdot,\cdot\rangle$ defined by 
	\[
		\langle a+L_\rho,b+L_\rho \rangle = \langle a,b\rangle_0, ~~~~ 
		\mbox{for}~~a+L_\rho,b+L_\rho \in V_\rho.
	\]
	It follows from properties $i),~ii)$ and $iii)$ of $\langle \cdot,\cdot  \rangle _0$ that
	$\langle \cdot,\cdot  \rangle$ is an inner product on $V_\rho$ -- with
	\begin{align*}
				\langle a + L^\rho, a + L^\rho \rangle = 0
		&\iff 	\langle a,a\rangle_0=0										\\
		&\iff 	a \in L_\rho												\\
		&\iff 	a+L_\rho = 0+L_\rho
	\end{align*}
	giving positive definiteness.
	The completion of $V_\rho$, with respect to the norm $\|\cdot\|$ induced by $\langle\cdot,\cdot
	\rangle$, is a Hilbert space -- this is the Hilbert space $\Hr$ we're looking for.
	
	Now we fix $a \in A$, and consider the map 
	\[
		\pi_a : V_\rho \to V_\rho; b+L_\rho \mapsto ab+L_\rho.
	\]
	Let $b_1, b_2 \in A$ be such that $b_1+L_\rho = b_2+L_\rho$. Then:
	\begin{align*}
		&~~~~~~~~	b_1-b_2 \in L_\rho										\\	
		&\implies	a(b_1-b_2) \in L_\rho									\\	
		&\implies	ab_1-ab_2 \in L_\rho									\\	
		&\implies	ab_1 + L_\rho = ab_2+L_\rho								\\	
		&\implies	\pi_a(b_1+L_\rho) = \pi_a(b_2+L_\rho).
	\end{align*}
	Hence $\pi_a$ defines a linear operator on $V_\rho$.
	For $b+L_\rho \in V_\rho$:
	\begin{align*}
				\|a\|^2 \cdot \|b+L_\rho\| - \| \pi_a(b+L_\rho) \|
		&=		\|a\|^2 \cdot \|b+L_\rho\| - \| ab+L_\rho \|				\\
		&=		\|a\|^2 \cdot \langle b+L_\rho,b+L_\rho \rangle - 
									\langle ab+L_\rho,ab+L_\rho \rangle		\\
		&=		\|a\|^2 \cdot \rho(b^\ast b) - \rho ((ab)^\ast ab)			\\
		&= 		\rho (\|a\|^2 b^\ast b-b^\ast a^\ast ab)					\\
		&=		\rho (b^\ast (\|a\|^{2} \mathbbm{1} - a ^\ast a)b)			\\
		&\geq 	0.
	\end{align*}
	Thus $\pi_a$ is a bounded operator on $\Vr$, with $\|\pi_a\| \leq \|a\|$. By continuity,
	$\pi_a$ extends to a bounded operator $\pi_\rho(a):\Hr \to \Hr$,
	such that \[ \pi_\rho(a)(v) = \pi_a(v) \] for $v\in \Vr$. 
	Then $\pi_\rho(a) \in \B(\Hr)$ for each $a \in A$, so $\pi_\rho$ defines a map 
	$A \to \B(\Hr)$ such that $ a \mapsto \pi_\rho(a)$. This will be our representation.
	
	Now, for $a,b \in A$, $c+ L_\rho \in \Vr$ and $\alpha \in \mathbb{C}$:
	\begin{align*}
				\pi_{\alpha a+b}(c+\Lr)
		&=		(\alpha a+b) (c+\Lr)										\\
		&=		(\alpha ac +\Lr) + (bc+\Lr)									\\
		&=		\alpha \pi_a (c+\Lr) + \pi_b(c+\Lr),
	\end{align*}
	so that $\pi_{\alpha a + b} = \alpha \pi_a +\pi_b$ on $\Vr$.\\
	For $a,b \in A$ and $c+ L_\rho \in \Vr$:
	\begin{align*}
		\pi_{ab}(c+\Lr)
		&=		abc+\Lr														\\
		&=		\pi_a (bc+\Lr)												\\
		&=		\pi_a (\pi_b (c+\Lr))										\\
		&=		(\pi_a \cdot \pi_b) (c+\Lr),
	\end{align*}
	so that $\pi_{a b} = \pi_a \cdot \pi_b$ on $\Vr$.\\
	For $a\in A$ and $b+ L_\rho,~c+ L_\rho \in \Vr$:
	\begin{align*}
				\langle b+\Lr, \pi_a^\ast (c+\Lr) \rangle 
		&=		\langle \pi_a (b+\Lr), c+\Lr \rangle						\\
		&=		\langle ab +Lr, c+\Lr \rangle								\\
		&=		\rho(c^\ast ab)												\\
		&=		\rho((a^\ast c)^\ast b)										\\
		&=		\langle b+\Lr, a^\ast c+\Lr \rangle							\\
		&=		\langle b+\Lr, \pi_{a^\ast}(c+\Lr)\rangle,
	\end{align*}
	so that $ \pi_a^\ast = \pi_{a^\ast}$ on $\Vr$.\\
	The subset $\Vr$ is dense in $\Hr$, so the three properties above hold on $\Hr$ by continuity of 
	$\pi_\rho$. Hence, $\pi_\rho: A \to \B(\Hr)$ is a representation of $A$.
	As to the unit vector, consider $\xr = \1 + \Lr \in \Vr$. Then for $a \in A$,
	\begin{align*}
				\langle \pi_\rho(a)\xr ,\xr \rangle 
		&=		\langle \pi_a(\1+\Lr), \1+\Lr \rangle						\\
		&=		\langle a+\Lr, \1+\Lr \rangle								\\
		&=		\rho(a);
	\end{align*}
	and in particular, $\langle \xr,\xr \rangle = \rho (\1) = 1$, so $\xr$ is a unit vector in $\Hr$. 
	The set $\{\pi_\rho(a)x_\rho ~|~ a\in A\}$ is the dense subset $\Vr$ of $\Hr$, so $x_\rho$ is a 
	cyclic vector for $\rho$.
\end{proof}

We know from Section \ref{section:CX} that all functionals of the form $\rho_x(f) = f(x)$, for $x$ in 
$X$, are pure states on the $C^\ast$-algebra $\CX$. So, what happens if we apply the GNS construction to 
$\rho_x$?

Fix $x$ in $X$. If $\rho_x(\bar f f)=0$, then $(\bar f f)(x)=0$ and $f(x)=0$. Then the left kernel
$L_{\rho_x}$ reduces to the kernel (in the usual linear sense) of $\rho_x$:
\begin{align*}
	L_{\rho_x} = \{f\in\CX ~|~ f(x) = 0\} = \ker(\rho_x).
\end{align*}
The kernel of a non-zero linear functional is a proper ideal, with codimension $1$, so when we take the 
quotient we find that 
\begin{align*}
	V_{\rho_x} = \CX/L_{\rho_x} \cong \C,
\end{align*}
which we identify with $\C$ in the natural way: $f+L_{\rho_x} \mapsto f(x)$.
Then for $g$ in $\CX$, the map 
\begin{align*}
	\pi_g:\C\to\C:z\mapsto g(x)z
\end{align*} 
is a linear operator on $\C$, and the map 
\begin{align*}
	\pi_{\rho_x}:\CX\to\B(\C):g\mapsto\pi_g
\end{align*} 
is a representation of $\CX$ on $\C$.
The unit in $\CX$ is the constant function $\1$, which in $V_{\rho_x}$ is identified with $\1(x)=1$; so 
we have that $1$ is a cyclic vector for the representation $\pi_{\rho_x}$.

\section{The Gelfand-Naimark theorem}\label{section:gn}
We finish this chapter with the big result; that every $C^\ast$-algebra can be isometrically embedded 
as a $C^\ast$-subalgebra of $\BH$, the $C^\ast$-algebra of bounded linear operators on a Hilbert space 
$\H$. First, we need another definition.
\begin{defn}
	Let $A$ be a $C^\ast$-algebra. Suppose we have a collection $\{\phi_i ~|~i\in I\}$ of 
	representations of $A$ on Hilbert spaces $\{\H_i ~|~ i\in I\}$.
	For $a$ in $A$, we have $\|\phi_i(a)\|\leq \|a\|$ (by Proposition \ref{prop:homo}, as each $\phi_i$ 
	is a $\ast$-homomorphism), so we have a bounded operator $\oplus\phi_i(a)$ on $\oplus \H_i$ by 
	section \ref{section:hscons}. By the properties of $\oplus\phi_i(a)$ stated therein, the map 
	\[
		\phi:A\to \B(\oplus\H_i):a\mapsto\oplus\phi_i(a)
	\]
	is a $\ast$-homomorphism, and so is a representation of $A$ on $\oplus\H_i$. Call $\phi$ the 
	\emph{direct sum} of the collection $\{\phi_i ~|~i\in I\}$, denoted $\oplus\phi_i$.
\end{defn}


\begin{thm}[Gelfand-Naimark, {\cite[4.5.6]{kadison83}}] \label{thm:gn}
	Every $C^\ast$-algebra has a faithful representation.
\end{thm}
\begin{proof}
	With $\S_0$ any collection of states on $A$ containing $\P{A}$, let $\phi$ be the direct sum of
	the collection $\{\pi_\rho ~|~ \rho\in\S_0\}$ of representations as constructed by the GNS
	construction. We will show that $\phi$ is a faithful representation.
	
	Given $a$ in $A$, if $\phi(a)=0$ then $\pi_\rho(a)=0$ for all pure states $\rho$ on $A$. But then,
	since $\rho(a)=\langle \pi_\rho (a) x_\rho, x_\rho \rangle$ by the GNS construction, we have 
	$\rho(a)=0$, and then by Lemma \ref{lemma:pure1}, $a=0$. Thus, $\phi$ is one-to-one and hence a 
	faithful representation of $A$.
\end{proof}

Note that the proof of the Gelfand-Naimark theorem depends on a choice of the set $\S_0$ of states on 
$A$, with the only restriction being that it contains the set of pure states $\P A$. If we take the 
direct sum over the representations associated with all of $\S (A)$, then $\phi$ is called \emph{the 
universal representation} of $A$.




\begin{thebibliography}{00}
\bibitem{blackadar06}
	Blackadar, B.,
	\emph{Operator Algebras: Theory of C*-Algebras and von Neumann Algebras.}
	Encyclopaedia of Mathematical Sciences, 122. Operator Algebras and Non-commutative Geometry, III. 
	Springer-Verlag, Berlin (2006).

\bibitem{davidson96} 
	Davidson, K.R.,
	\emph{C*-Algebras by example.}
	Fields Institute Monographs, 6. American Mathematical Society, Providence, RI (1996).

\bibitem{dixmier77}
	Dixmier, J.,
	\emph{C*-algebras.}
	Holland Mathematical Library, Vol. 15. North-Holland Publishing Co., Amsterdam -- New York -- Oxford 
	(1977).

\bibitem{dixmier81}
	Dixmier, J.,
	\emph{von Neumann algebras.}
	North-Holland Publishing Co., Amsterdam -- New York (1981).

\bibitem{gelfand43}
	Gelfand, I. \& Neumark, M.,
	\emph{On the imbedding of normed rings into the ring of operators in Hilbert space.}
	Rec. Math. [Mat. Sbornik] N.S. 12(54) (1943), pp.~197–-213.
	
\bibitem{kadison83}
	Kadison, R. V. \& Ringrose, J. R.,
	\emph{Fundamentals of the theory of operator algebras: Vol. I. Elementary theory.}
	Pure and Applied Mathematics, 100. Academic Press, Inc. [Harcourt Brace Jovanovich, Publishers], New 
	York (1983).

\bibitem{kadison86}	
	Kadison, R. V. \& Ringrose, J. R.,
	\emph{Fundamentals of the theory of operator algebras: Vol. II. Advanced theory.}
	Pure and Applied Mathematics, 100. Academic Press, Inc., Orlando, FL (1986).

\bibitem{rudin91}
	Rudin, W.,
	\emph{Functional Analysis (2nd Ed.).}
	McGraw-Hill (1991).

\bibitem{sakai71}
	Sakai, S.,
	\emph{C*-algebras and W*-algebras.}
	Ergebnisse der Mathematik und ihrer Grenzgebiete, Band 60. Springer-Verlag, New York-Heidelberg 
	(1971).
	
\bibitem{simmons83}
	Simmons, G.,
	\emph{Introduction to Topology and Modern Analysis.}
	Robert E. Krieger Publishing Co., Inc., Melbourne, Fl. (1983).

\bibitem{vonneumann32}
	von Neumann, J.,
	\emph{Mathematical Foundations of Quantum Mechanics.}
	(as translated by Beyer, R.T.)
	Princeton University Press (1955).
	
\end{thebibliography}

\end{document}