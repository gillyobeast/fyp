\documentclass[12pt,a4paper]{amsart}
\usepackage{graphicx}
\usepackage[color=purple!40]{todonotes} %% does the to do stuff
\usepackage{enumitem}

\theoremstyle{plain}

\newtheorem{thm}{Theorem}
\newtheorem*{thm*}{Theorem}
\theoremstyle{definition}
\newtheorem{defn}{Definition}
\newtheorem*{defn*}{Definition}

\author{Luke Armitage}
\title{C$^\ast$-Algebras, and the~Gelfand-Naimark~Theorem}

\begin{document}
\maketitle


\section{Preliminaries}
we will assume knowledge on... \\
brief(er than asst 3) history




\section{Basics}

\subsection{Definitions}
\begin{defn}[C*-algebra]
	
\end{defn}

\todo[inline] {define C*algebras, states, representations, the weak* topology}





\section{Representations of C*-algebras}
\todo[inline]{to include all representation theory, including GNS, CGN and GN}


\begin{thm} [Gelfand- Naimark-Segal]
	If $\rho$ is a state on a $C^\ast$-algebra $A$, then there exists a cyclic representation 
	$\pi_\rho$ of $A$ on a Hilbert space ${H}_\rho$, with unit cyclic vector $x_\rho$, such that 
	\[ 
		\rho(a)~=~ \langle \pi_\rho (a) x_\rho, x_\rho \rangle, ~~~~ \forall a \in A.
	\]
\end{thm}
\begin{proof}
	We will construct from $\rho$ the space $H_rho$,  representation $\pi_\rho$, and vector $x_\rho$, and demonstrate the required properties.
	
	Consider the \emph{left kernel} of $\rho$:
	\[
		L_\rho := \{t \in A | \rho (t ^\ast t) = 0 \}.
	\]	
	For $a,b \in A$, define $\langle a , b \rangle_0 := \rho(b^\ast a)$.
	Then $L_\rho = \{t\in A | \langle t , t \rangle _0 = 0 \} $, and $\langle \cdot, \cdot \rangle_0$ satisfies
	\begin{enumerate}[label=(\roman*)]
	  \item Linearity in 1st argument: for $a,b\in A$, $\alpha, \beta \in \mathbb{C}$:
		\begin{align*}
		   \langle \alpha a + \beta b, c \rangle_0 
		&= \rho (c^\ast(\alpha a + \beta b)   								\\
		&= \rho (\alpha c^\ast a + \beta c^\ast b)  						\\
		&= \alpha \rho (c^\ast a) + \beta \rho (c^\ast b)					\\
		&= \alpha \langle a , c \rangle_0 + \beta \langle b, c \rangle_0.
		\end{align*}
	  \item Conjugate symmetric: for $a,b \in A$:
	  	\begin{align*}
	  	   \langle b,a \rangle _0 
	  	&= \rho (a^\ast b)													\\
	  	&= \rho ((b^\ast a)^\ast)											\\
	  	&= \overline{\rho (b^\ast a)}										\\
	  	&= \overline{\langle a,b \rangle _0 }.
	  	\end{align*}
	  \item Positive semi-definite. \todo{why?}
	  	\begin{align*}
	  	\end{align*}
	\end{enumerate}
	Note that $\langle \cdot, \cdot \rangle$ is not necessarily positive definite on $A$ -- $L_\rho$ is exactly where this fails.
	
	$L_\rho$ is a linear subspace of $A$: Consider 
	\[
		L:= \{t \in A  | \langle t,a \rangle _0 = 0 \forall a \in A \}\subseteq L_\rho.
	\]
	
	For $t \in L_\rho$, by Cauchy-Schwarz we have 
	\[ 
		|\langle t,a \rangle_0|^2 \leq \langle t,t \rangle_0 \langle a,a\rangle_0,~~ \forall a \in A;
	\]
	that is,
	\[
		\langle t,a \rangle _0 = 0, ~~~ \forall a \in A,
	\]
	so $t\in L$ and $L_\rho =L$.
\end{proof}




\end{document}