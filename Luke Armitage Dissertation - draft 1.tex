\documentclass[12pt,a4paper]{amsart}
\usepackage{graphicx}
\usepackage[color=purple!80]{todonotes} %% does the to do stuff
\usepackage{enumitem}
%%\usepackage{fullpage}
\usepackage{bbm}
\usepackage{hyperref}
\usepackage{mathrsfs}




\theoremstyle{plain}

\newtheorem{thm}{Theorem}
\newtheorem*{thm*}{Theorem}
\newtheorem*{claim}{Claim}
\newtheorem{prop}{Proposition}
\newtheorem*{prop*}{Proposition}
\theoremstyle{definition}
\newtheorem{defn}{Definition}
\newtheorem*{defn*}{Definition}

\renewcommand{\H}{\mathcal{H}}
\newcommand{\B}{\mathcal{B}}
\newcommand{\BH}{\mathcal{\B(\H)}}
\newcommand{\Hr}{\mathcal{H}_\rho}
\newcommand{\1}{\mathbbm{1}}
\newcommand{\C}{\mathbb{C}}
\newcommand{\R}{\mathbb{R}}
\newcommand{\N}{\mathbb{N}}
\newcommand{\Vr}{V_\rho}
\newcommand{\Lr}{L_\rho}
\newcommand{\xr}{x_\rho}
\renewcommand{\phi}{\varphi}
\newcommand{\CX}{C(X)}
\renewcommand{\P}[1]{\mathscr{P}(#1)}

\author{Luke Armitage}
\title{$C^\ast$-Algebras, and the~Gelfand-Naimark~Theorem}

\begin{document}
\maketitle
\makeatletter  %%todonotes magic
    \providecommand\@dotsep{5}
  \makeatother
\listoftodos\relax
\newcommand\Item[1][]{ %% enumerate magic
  \ifx\relax#1\relax  \item \else \item[#1] \fi
  \abovedisplayskip=0pt\abovedisplayshortskip=0pt~\vspace*{-\baselineskip}}

\section{Preliminaries}
we will assume knowledge on... (algebra homos map 1 to 1,hausdorff/compact spaces)\\
brief(er than asst 3) history \\
most texts start from $\BH$ to justify the whole thing. we're algebraists, who don't need no justification. we jump right in at the deep (abstract) end.
\todo[inline]{write this!} %%


\section{Definitions}

\subsection*{}
\begin{defn}
	A \emph{Banach algebra} is a complex Banach space $(A,\|\cdot\|)$ which forms an 
	algebra, such that 
	\begin{align*}
		\|ab\| \leq \|a\| \|b\| \mbox{ for all } a,b \in A.
	\end{align*}
	A \emph{$\ast$-algebra} is an algebra $A$ with an \emph{involution} map 
	$a \mapsto a^\ast$ on $A$ such that, for all $a,b \in A$ and for $\alpha \in \C$,
	\renewcommand{\labelenumi}{(\roman{enumi})}
	\begin{enumerate}
		\item $a^{\ast\ast} = (a^\ast)^\ast = a$,
		\item $(\alpha a+b)^\ast = \overline{\alpha} a^\ast + b^\ast$,
		\item $(ab)^\ast = b^\ast a^\ast$.
	\end{enumerate}
	The element $a^\ast$ is referred to as the \emph{adjoint} of $a$. 		\\
	A \emph{C*-algebra} is a Banach algebra $(A, \| \cdot \|)$ with involution 
	map $a \mapsto a^\ast$, with the condition
	\begin{align*}
		\|a ^\ast a\| = \|a\|^2 \mbox{ for all } a \in A.
	\end{align*}
	This condition is known as the \emph{$C^\ast$ axiom}. There is a weaker, but 
	ultimately equivalent, axiom called the \emph{$B^\ast$ axiom}.\todo{state and prove} %%
\end{defn}
Here we consider complex $C^\ast$-algebras. The theory of real $C^\ast$-algebras has advanced....
\todo[inline]{remark on work in real $C^\ast$-algebras} %%

Unless specified otherwise, by an \emph{ideal} of a Banach algebra, we mean a two-sided ideal.
A $\ast$-ideal in a Banach $\ast$-algebra is a $\ast$-closed ideal. 
For $C^\ast$-algebras, it turns out that any ideal is automatically a $\ast$-ideal\todo{prove}.%%

\subsection{Unitization}
If a $C^\ast$-algebra $A$ contains an identity element $\1$ such that 
$a\cdot \1 = a = \1 \cdot a$ for all $a \in A$, call $\1$ the \emph{unit} 
in $A$, and $A$ is then a \emph{unital} $C^\ast$-algebra.

\begin{prop}
	Any non-unital $C^\ast$-algebra $A$ can be isometrically embedded 
	in a unital $C^\ast$-algebra $\tilde{A}$ as a maximal ideal.
\end{prop}
\todo[inline]{tidy up proof}%%
\begin{proof}
	Let $\tilde{A} = A \oplus \C$ with pointwise addition, and define
	\begin{align*}
		(a,\lambda) (b,\mu) &:= (ab+\lambda b + \mu a, \lambda \mu),		\\
		(a,\lambda)^\ast &:= (a^\ast,\overline{\lambda}),					\\
		\|(a,\lambda)\| &:= \sup_{\|b\|=1}{\|ab+\lambda b\|}.
	\end{align*}
	Then $\tilde{A}$ is a $\ast$-algebra. The norm $\|(a,\lambda)\|$ is the norm 
	in $\B(A)$ of left-multiplication by $a$ on something iunno???
	Thus $\tilde{A}$ is a Banach $\ast$-algebra with unit $(0,1)$. 
	By design, $A$ is a maximal ideal of codimension 1. 
	The embedding $a\mapsto(a,\lambda)$ is isometric as 
	\[
		\|a\| = \|a\cdot\frac{a}{\|a\|}\| \leq \|(a,0)\| \leq \sup_{\|b\|=1}{\|ab\|} \leq \|a\|.
	\]
	It remains to verify the $C^\ast$-axiom:
	\begin{align*}
				\|(a,\lambda)\|^2 
		&=		\sup_{\|b\|=1}{\|ab+\lambda b\|^2}								\\
		&=		\sup_{\|b\|=1}{\|b^\ast a^\ast ab 
								+\lambda b^\ast a^\ast b
								+\overline{\lambda}b^\ast a b
								+|\lambda|^2 b^\ast b}							\\
		&\leq	\sup_{\|b\|=1}{\|a^\ast ab 
								+\lambda a^\ast b
								+\overline{\lambda}a b
								+|\lambda|^2 b\|}									\\
		&=		\|(a^\ast a + \lambda a^\ast +\overline{\lambda}a,|\lambda|^2)	\\
		&= 		\|(a,\lambda)^\ast(a,\lambda)\|									\\
		&\leq	\|(a,\lambda)^\ast\| \|(a,\lambda)\|.
	\end{align*}
	By symmetry of $\ast$, $\|(a,\lambda)^\ast\| = \|(a,\lambda)\|$. 
	Hence, the above inequality becomes equality and we have that
	\begin{align*}
		\|(a,\lambda)^\ast(a,\lambda)\| = \|(a,\lambda)\|^2. 
	\end{align*}		
	
\end{proof}

In light of this result, we take all $C^\ast$-algebras from here to be unital. 
For the results we will consider, we can simply consider the unital case. 
However, there are important circumstances in which we need to relax the unital condition, 
\todo[inline]{does the following theory carry from a unital C*alg to any non-unital ideal? a remark}%%

\subsection{the spectrum}
Given an element $a \in A$ of a $C^\ast$-algebra, define its spectrum sp$(a)$:
\[
	\mbox{spec}(a):=\{\lambda\in \C ~|~ a-\lambda \1 \mbox{ is not invertible in }A\}.
\]


\subsection{more definitions}
An element $a\in A$ of a $C^\ast$-algebra is called 
\begin{itemize}
	\item \emph{self-adjoint} if $a^\ast=a$;
	\item \emph{unitary} if $aa^\ast=a^\ast a = \1$;
	\item \emph{positive} if it is self-adjoint and sp$(a)$ $\subseteq \R^+$.
\end{itemize}
Denote the set of positive elements in $A$ by $A^+$.


	Given Banach $\ast$-algebras $A$ and $B$, a map $\phi:A\to B$ is a \emph{$\ast$-homomorphism} 
	if it is an algebra homomorphism for which $\phi(a^\ast) = \phi(a)^\ast$ for all $a \in A$.
	If a $\ast$-homomorphism $\phi$ is one-to-one, call it a \emph{$\ast$-isomorphism}.


\todo[inline]{norm preservation theorem here?}%%


	A \emph{linear functional} on a $C^\ast$-algebra $A$ is a linear operator $\rho:A \to \C$.
	A \emph{state} on $A$ is a linear functional $\rho$ 
	such that \todo{is this right? what of $\rho(\1)=1$?}$\|\rho\|=1$ and $\rho (a) \geq 0$  %%
	for all positive elements $a \in A^+$. 



\todo [inline]{normal(?), the weak* topology(?), cauchy-schwarz}%%

\subsection{$\B(\H)$ - an example.}
This section concerns the fundamental example of a $C^\ast$-algebra - the set $\B(\H)$ of bounded 
linear operators on a Hilbert space $\H$.  
Here we will demonstrate that $\B(\H)$ is a $C^\ast$-algebra and give some basic results.

\begin{claim} $\B(\H)$ is a $C^\ast$-algebra with the operator norm 
\[
	\|T\|:= \sup_{\|x\|=1}{\|Tx\|}
\]
and involution taking $T$ to its adjoint map $T^\ast$. 
The identity map $I:x\mapsto x$ is a unit for $\BH$
\end{claim}
\begin{proof}
	$\|\cdot\|$ is a norm on $\BH$. Let $\{T_n\}_{n\in\N}$ be a Cauchy sequence 
	in $\BH$. Then for any positive $\epsilon$, there is a positive integer $N$ such that 
	\begin{align*}
		\|T_m-T_n\| < \epsilon \mbox{ for all } m,n \geq N.
	\end{align*}
	Applying $T_m-T_n$ to $x \in \H$, we have 
	\begin{align}\label{cauchyTn}
		\|T_mx-T_nx\| \leq \|T_m-T_n\| \|x\| < \epsilon \|x\|,
	\end{align}
	so $\{T_nx\}_{n\in\N}$ is a Cauchy sequence in $\H$, converging to an element in $\H$.
	Define a linear operator $T:\H \to \H$ by 
	\begin{align*}
		Tx:= \lim_{n\to\infty}{T_nx} \mbox{ for } x \in \H.
	\end{align*}
	Taking limits as $m$ tends to infinity in equation \eqref{cauchyTn}, we obtain
	\begin{align*}
		\|Tx-T_nx\| < \epsilon \|x\| \mbox{ for all }n \geq N,
	\end{align*}
	and so we have that $T-T_n$ (and hence $T=(T-T_n)+T_n$) is a bounded operator and  
	\begin{align*}
		\|T-T_n\| <\epsilon \mbox{ for all }n \geq N.
	\end{align*}
	We conclude that $T_n \to T$, and so $\BH$ is complete.
	
	Since boundedness is equivalent to continuity on $\H$, given $S,T\in\BH$, the operator 
	$ST:\H \to \H; x \mapsto (S\circ T)(x)$ is bounded on $\H$.
	Given $x\in\H$ and $\lambda\in\C$, 
	\begin{align*}
			((\lambda S)T)(x)
		&=	((\lambda S)\circ T)(x)												\\
		&=	\lambda S(Tx)														\\
		&=	\lambda (S\circ T)(x)												\\
		&=	\lambda ST(x),
	\end{align*} 
	so that $(\lambda S)T = \lambda ST$ in $\BH$, whence $\BH$ is an algebra.
	We have 
	\begin{align*}
				\|ST\|
		&=		\sup_{\|x\|=1}{\|STx\|} 										\\
		&=		\sup_{\|x\|=1}{\|S(Tx)\|} 										\\
		&\leq	\|S\| \sup_{\|x\|=1}{\|Tx\|} 									\\
		&=		\|S\| \|T\|.
	\end{align*}
	
	To see that $^\ast$ is an involution, use the fact that the adjoint 
	operator is unique for each operator and the following equalities.
	\begin{enumerate}[label=(\roman*)]
	\Item	\begin{align*}
				\langle (\alpha T+S)^\ast x,y\rangle 
			&=	\langle x, \alpha T+S y \rangle									\\
			&=	\overline{\alpha}\langle x,Ty\rangle + \langle x, Sy \rangle	\\
			&=	\overline{\alpha}\langle T^\ast x,y\rangle +
										 \langle S^\ast x, y \rangle			\\
			&=	\langle(\overline{\alpha} T^\ast + S^\ast) x, y \rangle.
			\end{align*}
	\Item 	\begin{align*}
				\langle (T^\ast)^\ast x,y\rangle 
			&=	\langle x, T^\ast y \rangle										\\
			&=	\overline{\langle T^\ast y,x \rangle}							\\
			&=	\overline{\langle y, Tx \rangle}								\\
			&=	\langle Tx,y\rangle.
			\end{align*}
	\Item	\begin{align*}
				\langle (ST)^\ast x,y \rangle
			&=	\langle x, STy \rangle											\\
			&=	\langle	S^\ast x,Ty \rangle										\\
			&=	\langle T^\ast S^\ast x,y \rangle.
		\end{align*}
	\end{enumerate}
	It remains to demonstrate the $C^\ast$-axiom on $\BH$. For all $x\in\H$, we have
	\begin{align*}
		\|Tx\|^2 = \langle Tx,Tx\rangle = \langle T^\ast Tx,x\rangle \leq \|T^\ast T\| \|x\|^2,
	\end{align*}
	so that
	\begin{align*}
		\|T\|^2 \leq \|T^\ast T\| \leq \|T^\ast\| \|T\| = \|T\|^2.
	\end{align*}
	It is clear that $I$ is a unit.
	Hence, the claim.
\end{proof} 




\todo [inline]{example of $\B(\H)$ at end of section to retro-motivate notation. }%%

\subsection{$\CX$ - another example.}
Given a locally compact Hausdorff space $X$, let $\CX$ be the algebra of continuous functions
$f:X\to\C$, with addition and multiplication defined pointwise. Define $\|\cdot\|$ on $\CX$ by
\[
	\|f\|:= \sup_{x\in X}{|f(x)|},
\]
that is the norm inherited from the Banach space $\ell^2(X,\C)$.





\todo [inline]{example of $C(X)$?} %%


\section{Representations of C*-algebras}
\todo[inline]{to include all representation theory, including GNS, CGN and GN}%%

\begin{defn}
	Given a $C^\ast$-algebra $A$, a \emph{representation of $A$ on a Hilbert space $\H$} 
	is a $\ast$-homomorphism $\phi: A \to \B(\H)$. 
	An isomorphic representation is called \emph{faithful}.
\end{defn}
	
\begin{thm}[Gelfand-Naimark for commutative algebras] \todo {find correct name - Gelfand Representation theorem?}%%
	Every commutative $C^\ast$-algebra $A$ is $\ast$-isomorphic to $C(\P{A})$,
	 the continuous functions on the topological space of pure states on $A$.
\end{thm}	


From [subsection on \CX], $C(\P{A})$ is unital if and only if $\P{A}$ is compact. In the case that 
$\P{A}$ is noncompact but locally compact, the unitization of $C(\P{A})$ corresponds to $C(\P{A}^\dagger)$ where $X^\dagger$ is the one-point compactification of the topological space $X$.
	\todo{remark about nonunital commutative.}%%
	
\begin{thm} [Gelfand-Naimark-Segal]
	If $\rho$ is a state on a $C^\ast$-algebra $A$, then there exists a cyclic representation 
	$\pi_\rho$ of $A$ on a Hilbert space ${H}_\rho$, with unit cyclic vector $x_\rho$, such that 
	\[ 
		\rho(a)~=~ \langle \pi_\rho (a) x_\rho, x_\rho \rangle, ~~~~ \forall a \in A.
	\]
\end{thm}
\begin{proof}
	We will construct from $\rho$ the space $\Hr$,  representation $\pi_\rho$, 
	and vector $x_\rho$, and demonstrate the required properties.
	
	Consider the \emph{left kernel} of $\rho$:
	\[
		L_\rho := \{t \in A ~|~ \rho (t ^\ast t) = 0 \}.
	\]	
	For $a,b \in A$, define $\langle a , b \rangle_0 := \rho(b^\ast a)$.
	Then $L_\rho = \{t\in A ~|~ \langle t , t \rangle _0 = 0 \} $, and 
	$\langle \cdot, \cdot \rangle_0$ satisfies
	\begin{enumerate}[label=(\roman*)]
	  \item Linearity in 1st argument: for $a,b\in A$, $\alpha, \beta \in \mathbb{C}$:
		\begin{align*}
		   \langle \alpha a + \beta b, c \rangle_0 
		&= \rho (c^\ast(\alpha a + \beta b)   								\\
		&= \rho (\alpha c^\ast a + \beta c^\ast b)  						\\
		&= \alpha \rho (c^\ast a) + \beta \rho (c^\ast b)					\\
		&= \alpha \langle a , c \rangle_0 + \beta \langle b, c \rangle_0.
		\end{align*}
	  \item Conjugate symmetric: for $a,b \in A$:
	  	\begin{align*}
	  	   \langle b,a \rangle _0 
	  	&= \rho (a^\ast b)													\\
	  	&= \rho ((b^\ast a)^\ast)											\\
	  	&= \overline{\rho (b^\ast a)}										\\
	  	&= \overline{\langle a,b \rangle _0 }.
	  	\end{align*}
	  \item Positive semi-definite. \todo{why?} %%
	  	\begin{align*}
	  	\end{align*}
	\end{enumerate}
	Note that $\langle \cdot, \cdot \rangle$ is not necessarily positive definite on 
	$A$ -- $L_\rho$ is exactly where this fails.
	
	$L_\rho$ is \todo{sentence} a linear subspace of $A$: Consider  %%
	\[
		L:= \{t \in A  ~|~ \langle t,a \rangle _0 = 0, ~\forall a \in A \}\subseteq L_\rho.
	\]
	
	For $t \in L_\rho$, by Cauchy-Schwarz we have 
	\[ 
		|\langle t,a \rangle_0|^2 \leq \langle t,t \rangle_0 \langle a,a\rangle_0,~~ \forall a \in A;
	\]
	that is,
	\[
		\langle t,a \rangle _0 = 0, ~~~ \forall a \in A,
	\]
	so $t\in L$ and $L_\rho =L$.
	
	Now, for $a,b \in L$, $\alpha \in \mathbb{C}$ and $c \in A$:
	\[
		\langle \alpha a + b, c \rangle _0 = \alpha \langle a,c \rangle _0 + \langle b,c\rangle _0 = 0,
	\]
	so $\alpha a +b \in L$; also, $\langle 0,c\rangle _0 = 0$ so $0 \in L$. 
	Hence, $L ( =L_\rho ) $ is a linear subspace of $A$.
	
	For $s \in A$, $t\in L_\rho$, by the Cauchy Schwarz inequality [ref] we have 
	\begin{align*}
		|\rho (s^\ast t) |^2 
		&= 		|\langle t,s\rangle_0 |^2 									\\
		&\leq 	\langle t,t\rangle_0 \cdot \langle s,s\rangle_0  			\\
		&= 		\rho (t^\ast t) \cdot \rho (s^\ast s)						\\
		&=		0,
	\end{align*}
	so $\rho (s^\ast t) = 0$. Letting $s = a^\ast a t$ for $a \in A$, then
	\begin{align*}
		\rho ((at)^\ast at) 
		&= 		\rho (at^\ast a^\ast at) 									\\
		&= 		\rho ((a^\ast at)^\ast t) 									\\
		&= 		\rho (s^\ast t) 											\\
		&=		0,
	\end{align*}
	so that $at \in L_\rho$, for all $a \in A$ and $t \in L_\rho$; 
	we conclude that $L_\rho$ is a left ideal in $A$.
	$L_\rho$ is \todo{start sentence properly} the preimage in $A$ of $\{0\}$ under the continuous map%%
	$t \mapsto \rho (t^\ast t)$, so is closed.
	
	Consider now $V_\rho := A / L_\rho$, with $\langle \cdot,\cdot\rangle$ defined by 
	\[
		\langle a+L_\rho,b+L_\rho \rangle := \langle a,b\rangle_0, ~~~~ 
		\mbox{for}~~a+L_\rho,b+L_\rho \in V_\rho.
	\]
	It follows from properties $i),ii)$ and $iii)$ of $\langle \cdot,\cdot  \rangle _0$ that
	$\langle \cdot,\cdot  \rangle$ is an inner product on $V_\rho$ -- with
	\begin{align*}
				\langle a + L^\rho, a + L^\rho \rangle = 0
		&\iff 	\langle a,a\rangle=0										\\
		&\iff 	a \in L_\rho												\\
		&\iff 	a+L_\rho = 0+L_\rho
	\end{align*}
	giving positive definiteness.
	The completion of $V_\rho$ with respect to $\langle \cdot,\cdot  \rangle$ is a Hilbert space -
	this is the Hilbert space $\Hr$ we're looking for.
	
	Now we fix $a \in A$, and consider the map 
	\[
		\pi_a : V_\rho \to V_\rho; b+L_\rho \mapsto ab+L_\rho.
	\]
	Let $b_1, b_2 \in A$ be such that $b_1+L_\rho = b_2+L_\rho$. Then:
	\begin{align*}
		&\implies	b_1-b_2 \in L_\rho										\\	
		&\implies	a(b_1-b_2) \in L_\rho									\\	
		&\implies	ab_1-ab_2 \in L_\rho									\\	
		&\implies	ab_1 + L_\rho = ab_2+L_\rho								\\	
		&\implies	\pi_a(b_1+L_\rho) = \pi_a(b_2+L_\rho).
	\end{align*}
	Hence $\pi_a$ defines a linear operator on $V_\rho$.
	
	For $b+L_\rho \in V_\rho$:
	\begin{align*}
				\|a\|^2 \cdot \|b+L_\rho\| - \| \pi_a(b+L_\rho) \|
		&=		\|a\|^2 \cdot \|b+L_\rho\| - \| ab+L_\rho \|				\\
		&=		\|a\|^2 \cdot \langle b+L_\rho,b+L_\rho \rangle - 
									\langle ab+L_\rho,ab+L_\rho \rangle		\\
		&=		\|a\|^2 \cdot \rho(b^\ast b) - \rho ((ab)^\ast ab)			\\
		&= 		\rho (\|a\|^2 b^\ast b-b^\ast a^\ast ab)					\\
		&=		\rho (b^\ast (\|a\|^{2} \mathbbm{1} - a ^\ast a)b)			\\
		&\geq 	0.
	\end{align*}
	
	Thus $\pi_a$ is a bounded operator, with $\|\pi_a\| \leq \|a\|$. By continuity, \todo {cont of what?} %%
	$\pi_a$ extends to a bounded operator on $\Hr$ -- say $\pi_\rho(a):\Hr \to \Hr$ 
	such that \[ \pi_\rho(a)(v) = \pi_a(v) \] for $v\in \Vr$. 
	Then $\pi_\rho(a) \in \B(\Hr)$ for each $a \in A$, so $\pi_\rho$ defines a map 
	$A \to \B(\Hr)$ such that $ a \mapsto \pi_\rho(a)$. This will be our representation.
	
	Now, for $a,b \in A$, $c+ L_\rho \in \Vr$ and $\alpha \in \mathbb{C}$:
	\begin{align*}
				\pi_{\alpha a+b}(c+\Lr)
		&=		(\alpha a+b) (c+\Lr)										\\
		&=		(\alpha ac +\Lr) + (bc+\Lr)									\\
		&=		\alpha \pi_a (c+\Lr) + \pi_b(c+\Lr),
	\end{align*}
	so that $\pi_{\alpha a + b} = \alpha \pi_a +\pi_b$ on $\Vr$.
	
	For $a,b \in A$ and $c+ L_\rho \in \Vr$:
	\begin{align*}
		\pi_{ab}(c+\Lr)
		&=		abc+\Lr														\\
		&=		\pi_a (bc+\Lr)												\\
		&=		\pi_a (\pi_b (c+\Lr))										\\
		&=		(\pi_a \cdot \pi_b) (c+\Lr),
	\end{align*}
	so that $\pi_{a b} = \pi_a \cdot \pi_b$ on $\Vr$.
	
	
	For $a\in A$ and $b+ L_\rho,~c+ L_\rho \in \Vr$:
	\begin{align*}
				\langle b+\Lr, \pi_a^\ast (c+\Lr) \rangle 
		&=		\langle \pi_a (b+\Lr), c+\Lr \rangle						\\
		&=		\langle ab +Lr, c+\Lr \rangle								\\
		&=		\rho(c^\ast ab)												\\
		&=		\rho((a^\ast c)^\ast b)										\\
		&=		\langle b+\Lr, a^\ast c+\Lr \rangle							\\
		&=		\langle b+\Lr, \pi_{a^\ast}(c+\Lr),
	\end{align*}
	so that $ \pi_a^\ast = \pi_{a^\ast}$ on $\Vr$.
	
	$\Vr \subset \Hr$ is a dense subset, so the three properties above hold on 
	$\Hr$ by continuity. \todo {cont of what?} %%
	Hence, $\pi_\rho: A \to \B(\Hr)$ is a representation of $A$.
	As to the unit vector, consider $\xr := \1 + \Lr \in \Vr$. Then for $a \in A$,
	\begin{align*}
				\langle \pi_\rho(a)\xr ,\xr \rangle 
		&=		\langle \pi_a(\1+\Lr), \1+\Lr \rangle						\\
		&=		\langle a+\Lr \1+\Lr \rangle								\\
		&=		\rho(a);
	\end{align*}
	in particular, $\langle \xr,\xr \rangle = \rho (\1) = 1$, so $\xr$ is a unit vector in $\Hr$.	
\end{proof}


\begin{thm}[Gelfand-Naimark]
	Every $C^\ast$-algebra has a faithful representation.
\end{thm}
\begin{proof}
	for this we just take the direct sum representation of the representations given from GNS
	by some set of states containing all pure states.
\end{proof}

\todo[inline]{references!!!!}

\end{document}